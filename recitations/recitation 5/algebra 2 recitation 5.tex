\documentclass[a4paper,10pt,oneside,openany]{article}

\usepackage[lang=hebrew]{maths}
\usepackage{hebrewdoc}
\usepackage{stylish}
\usepackage{lipsum}
\let\bs\blacksquare

\title{
אלגברה ב' (104168) \textenglish{---} אביב 2020-2021
\\
תרגול 5 \textenglish{---} נילפוטנטיות
}
\author{אלעד צורני}
\date{\today}

\begin{document}
\maketitle

\section{נילפוטנטיות}

\subsection{חזרה}

\begin{definition}[העתקה נילפוטנטית]
העתקה
$T \in \endo_{\mbb{F}}\prs{V}$
נקראת
\emph{נילפוטנטית}
אם יש
$k \in \mbb{N}_+$
עבורו
$T^k = 0$.
ה־%
$k$
המינימלי הזה נקרא
\emph{אינדקס הנילפוטנטיות של
$T$}.
\end{definition}

\subsection{תרגילים}

\begin{remark}
נניח לאורך התרגול כי
$V$
מרחב וקטורי סוף־מימדי.
\end{remark}

\begin{exercise}
יהי
$V$
מרחב וקטורי ממימד
$n \in \mbb{N}_+$
ותהי
$T \in \endo_{\mbb{F}}\prs{V}$
נילפוטנטית מאינדקס
$k$.
הראו כי
$k \leq n$.
\end{exercise}

\begin{solution}
ראינו כי
$\ker\prs{T^k} \leq \ker\prs{T^n}$,
לכן אם
$T^k = 0$
נקבל
$\ker\prs{T^k} = V \leq \ker\prs{T^n}$
כלומר
$\ker\prs{T^n} = V$
כלומר
$T^n = 0$.
ממינימליות
$k$
נקבל
$k \leq n$.
\end{solution}

\begin{exercise}
יהי
$\mbb{F}$
סגור אלגברית ותהי
$T \in \endo_{\mbb{F}}\prs{V}$.
הראו ש־%
$T$
 נילפוטנטית אם ורק אם הערך העצמי היחיד שלה הוא
$0$.
מצאו דוגמא נגדית כאשר
$\mbb{F}$
אינו סגור אלגברית.
\end{exercise}

\begin{solution}
נניח כי
$\lambda \in \mbb{F}$
ערך עצמי של
$T \in \endo_{\mbb{F}}\prs{V}$
נילפוטנטית.
יש
$k \in \mbb{N}_+$
עבורו
$T^k = 0$.
יהי
$v \in V$
וקטור עצמי של
$T$
עם ערך עצמי
$\lambda$.
אז
$T^k v = \lambda^k v = 0$
לכן
$\lambda = 0$.

נניח כי
$S \in \endo_{\mbb{F}}\prs{V}$
עם ערך עצמי יחיד
$0$.
כיוןן ש־%
$\mbb{F}$
סגור־אלגברית,
יש בסיס
$B$
כך ש־%
$\brs{S}_B$
משולשת עליונה, ונקבל שיש על האלכסון שלה אפסים כי
$0$
ערך עצמי יחיד של
$S$.
נכתוב
$B = \prs{v_1, \ldots, v_n}$
וגם
$V_i \ceq \spn\prs{v_1, \ldots, v_i}$
לכל
$i \in [n]$.
נסיק כי
$S V_i \subseteq V_{i-1}$
לכל
$i \in [n]$,
וגם
$S V_1 = 0$.
אז
\[S^n V = S^n V_n = S^{n-1} V_{n-1} = \ldots = S V_1 = 0\]
כלומר
$S^n = 0$.

עבור
$\mbb{F} = \mbb{R}$
ניקח את
$T = L_A$
עבור
\[A = \pmat{0 & 1 & 0 \\ -1 & 0 & 0 \\ 0 & 0 & 0}\]
שהערך העצמי הממשי היחיד שלה הוא
$0$
(כי
$\pm i \notin \mbb{R}$)
אבל
$A^4 = \pmat{1 & 0 & 0 \\ 0 & 1 & 0 \\ 0 & 0 & 0}$
לכן
$T^4 = e_1 = e_1$.
כלומר
$T^4 \neq 0$
ולכן
$T$
אינה נילפוטנטית.
\end{solution}

\begin{examples}
\begin{description}
\item[העתקות נילפוטנטיות:]
ההעתקה
$L_A$
נילפוטנטית עבור כל אחת מהמטריצות הבאות.
\begin{itemize}
\item $A = \pmat{0 & 1 & 0 \\ 0 & 0 & 1 \\ 0 & 0 & 0}$.
\item כל $A$
משולשת עליונה עם
$0$
על כל האלכסון הראשי, כפי שראינו בתרגיל האחרון.
\item $A = \pmat{0 & 0 & 0 \\ 0 & 0 & 1 \\ 1 & 0 & 0}$.
\end{itemize}
\item[העתקות שאינן נילפוטנטיות:]
עבור המטריצות הבאות,
$L_A$
אינה נילפוטנטית.
\begin{itemize}
\item $A = \pmat{1 & 0 & 0 \\ 0 & 1 & 0 \\ 0 & 0 & 0}$.
מתקיים
$Ae_1 = e_1$
ולכן
$1$
ערך עצמי ולא יתכן כי
$L_A$
נילפוטנטית.
\item $A = \pmat{1 & 1 & 1 \\ 1 & 1 & 1 \\ 1 & 1 & 1}$.
מתקיים
$A\prs{e_1 + e_2 + e_3} = 3 \prs{e_1 + e_2 + e_3}$
לכן
$3$
ערך עצמי, ולא יתכן כי
$L_A$
נילפוטנטית.
\item $A = \pmat{0 & 1 & 0 \\ -1 & 0 & 0 \\ 0 & 0 & 0} \in M_3\prs{\mbb{R}}$.
מתקיים
$A^4 = \pmat{1 & 0 & 0 \\ 0 & 1 & 0 \\ 0 & 0 & 0} \neq 0$
לכן
$L_A^4 \neq 0$
וראינו שאינדקס הנילפוטנטיות חייב להיות לכל היותר
$3$
במקרה זה. לכן
$L_A$
אינה נילפוטנטית.
\end{itemize}
\end{description}
\end{examples}

\begin{exercise}
הראו כי
$T \in \endo_{\mbb{F}}\prs{V}$
נילפוטנטית אם ורק אם לכל
$v \in V$
יש
$k \in \mbb{N}_+$
עבורו
$T^k v = 0$.
\end{exercise}

\begin{solution}
נניח כי
$T$
נילפוטנטית מאינדקס
$k$.
אז
$T^k v = 0 v = 0$
לכל
$v \in V$.

נניח להיפך כי לכל
$v \in V$
יש
$k \in \mbb{N}_+$
עבורו
$T^k v = 0$.
אז
$v \in \ker\prs{T^k}$.
אבל, ראינו כי
\[\ker\prs{T} \leq \ker\prs{T^2} \leq \ker\prs{T^3} \leq \ldots\]
מתייצבת, לכן יש
$m \in \mbb{N}_+$
עבורו
$\ker\prs{T^k} \subseteq \ker\prs{T^m}$
לכל
$k \in \mbb{N}_+$.
נקבל
$v \in \ker\prs{T^m}$
לכל
$v \in V$,
לכן
$T^m = 0$.
\end{solution}

\begin{exercise}
הראו שצירוף לינארי של העתקות נילפוטנטיות מתחלפות הוא נילפוטנטי.
מצאו דוגמה נגדית עבור העתקות נילפוטנטיות שאינן מתחלפות.
\end{exercise}

\begin{solution}
יהיו
$T_1. T_2 \in \endo_{\mbb{F}}\prs{V}$
נילפוטנטיות מאינדקסים
$k_1, k_2$
בהתאמה וכך שמתקיים
$T_1 T_2 = T_2 T_1$.
יהי
$\alpha \in \mbb{F}$.

מתקיים
\begin{align*}
\prs{\alpha T_1}^{k_1} = \alpha^{k_1} T_1^{k_1} = \alpha^{k_1} 0 = 0
\end{align*}
לכן
$\alpha T_1$
נילפוטנטית מאינדקס קטן או שווה ל־%
$k$.
כעת
\begin{align*}
\text{.} \prs{T_1 + T_2}^{k_1 + k_2 + 1} &= \sum_{i=0}^{k_1 + k_2} \binom{k_1 + k_2}{i} T_1^i T_2^{k_1 + k_2 - i}
\end{align*}
כאשר
$i \geq k_1$
מתקיים
$T_1^i = 0$
וכאשר
$i < k_1$
מתקיים
$k_1 + k_2 - i \geq k_2$
לכן
$T_2^{k_1 + k_2 - i} = 0$.
נקבל בסך הכל כי
$T_1 + T_2$
נילפוטנטית מאינדקס קטן או שווה
$k_1 + k_2$.

בלי ההנחה שההעתקות מתחלפות, אפשר לקחת
\begin{align*}
T_1 &= \pmat{0 & 1 \\ 0 & 0}, \\
T_2 &= \pmat{0 & 0 \\ 1 & 0}
\end{align*}
כאשר שתיהן נילפוטנטיות מאינדקס 2, אבל
\[T_1 + T_2 = \pmat{0 & 1 \\ 1 & 0}\]
הפיכה ולכן אינה נילפוטנטית (אין לה ערך עצמי
$0$).
\end{solution}

\begin{exercise}
תהי
$T \in \endo_{\mbb{F}}\prs{V}$
נילפוטנטית.
הראו כי
$\det\prs{T} = \tr\prs{T} = 0$.
\end{exercise}

\begin{solution}
ראינו בהרצאה שיש בסיס
$B$
של
$V$
כך ש־%
$\brs{T}_B$
משולשת עליונה.
כיוון שכל הערכים העצמיים של
$T$
שווים לאפס, נקבל שיש
$0$
על האלכסון.
אז
$\det\prs{T} = \det\prs{\brs{T}_B} = 0$
כמכפלת איברי האלכסון וגם
$\tr\prs{T} = \tr\prs{\brs{T}_B} = 0$
כסכום איברי האלכסון.
\end{solution}

\begin{exercise}
תהי
$T$
נילפוטנטית מאינדס
$k$.
הראו שהעתקות
$\prs{\id_V \pm T}$
הפיכות ומצאו את ההופכיות שלהן.
\end{exercise}

\begin{solution}
אנו יודעים כי
\[\sum_{k \in \mbb{N}} r^k = \frac{1}{1-r}\]
עבור
$r < 0$.
נרצה אם כן שההופכית של
$\id_V - T$
תהיה
$\id_V + T + \ldots + T^{k-1}$.
אכן,
\begin{align*}
\prs{\id_V - T}\prs{\id_V + T + \ldots + T^{k-1}} &= \sum_{i = 0}^{k-1} T^{i} - \sum_{i = 1}^k T^^i
\\&= \id_V - T^k
\\&= \id_V - 0
\\ \text{.} \hphantom{\prs{\id_V - T}\prs{\id_V + T + \ldots + T^{k-1}}} &= \id_V
\end{align*}

כעת, אם
$T$
נילפוטנטית מאינדקס
$k$
גם
$-T$
נילפוטנטית מאינדקס
$k$,
לכן ההופכית של
$\id_V + T = \id_V - \prs{-T}$
היא
\[\text{.} \id_V - T + T^2 - T^3 + \ldots + \prs{-1}^{k-1} T^{k-1}\]
\end{solution}

\begin{exercise}
הראו כי
\begin{align*}
D \colon \mbb{F}_n\brs{x} &\to \mbb{F}_n\brs{x} \\
p &\mapsto p'
\end{align*}
נילפוטנטית.
\end{exercise}

\begin{solution}
לכל
$p \in \mbb{F}_n\brs{x}$
מתקיים
$\deg_{\mbb{F}}\prs{p} \leq n$
לכן
$D^{n-1}\prs{p}$
פולינום קבוע, ולכן
$D^n p = 0$.
\end{solution}

\begin{exercise}
הראו כי
\begin{align*}
D \colon \mbb{F}\brs{x} &\to \mbb{F}\brs{x} \\
p &\mapsto p'
\end{align*}
אינה נילפוטנטית.
\end{exercise}

\begin{solution}
נניח בשלילה כי
$D$
נילפוטנטית מאינדקס
$k$.
אז
$D^k x^{k+1} = 0$,
אבל
\begin{align*}
D^k x^{k+1} &= \prs{k+1} D^{k-1} x^k
\\&= \ldots
\\&= \prs{k+1}!
\\&\neq 0
\end{align*}
בסתירה.
\end{solution}

\begin{exercise}
יהי
$V$
מרחב וקטורי ממימד
$n \in \mbb{N}_+$
ותהי
$T \in \endo_{\mbb{F}}\prs{V}$
נילפוטנטית מאינדקס
$k$.
לכל
$i \in [k]$
נסמן
$n_i \ceq \dim \ker\prs{T^i}$.
הראו שמתקיים
\begin{align*}
\text{.} 0 < n_1 < n_2 < \ldots < n_{k-1} < n_k = n
\end{align*}
\end{exercise}

\begin{solution}
מתקיים
$\ker\prs{T^k} = \ker\prs{0} = V$
לכן
$n_k = \dim\prs{V} = n$.
אם
$n_1 = 0$
נקבל
$\dim \ker\prs{T} = 0$
לכן
$T$
חד־חד ערכית ולכן הפיכה, בסתירה לנילפוטנטיות.

אם
$n_i = n_{i+1}$
עבור
$i \in [k-1]$
נקבל
$\ker\prs{T^i} = \ker\prs{T^{i+1}}$.
אבל, ראינו שבמקרה זה
$\ker\prs{T^i} = \ker\prs{T^j}$
לכל
$j \geq i$.
בפרט
$\ker\prs{T^i} = \ker\prs{T^k} = V$,
כלומר
$i = k$
בסתירה להנחה
$i \in \brs{k-1}$.
\end{solution}

\begin{exercise}
יהי
$V$
מרחב וקטורי ממימד
$n \in \mbb{N}_+$
עם בסיס
$B \prs{v_1, \ldots, v_n}$.
תהי
\begin{align*}
T \colon V &\to V \\
v_1 &\mapsto 0 \\
\text{.} \forall i > 1: v_i &\mapsto v_{i-1}
\end{align*}
הראו כי
$T$
נילפוטנטית מאינדקס
$n$
וכיתבו את
$\brs{T}_B$.
\end{exercise}

\begin{solution}
מתקיים
\[\brs{T}_B = \pmat{0 & 1 & 0 & \ldots & 0 \\ 0 & 0 & 1 & \ddots & \vdots \\ \vdots & \ddots & 0 & \ddots & 0 \\ 0 & 0 & \ddots & 0 & 1 \\ 0 & 0 & \cdots & 0 & 0}\]
וזאת משולשת עליונה עם
$0$
על האלכסון, לכן כפי שתיארנו מקודם
$T^n = 0$.
מאידך,
$T^{n-1}v_n = v_1 \neq 0$
לכן האינדקס של
$T$
שווה
$n$.
\end{solution}

\begin{definition}[מטריצה נילפוטנטית]
תהי
$A \in M_n\prs{\mbb{F}}$.
נאמר כי
$A$
נילפוטנטית אם
$L_A$
נילפוטנטית ונגדיר את אינדקס הנילפוטנטיות של
$A$
להיות זה של
$L_A$.
\end{definition}

\begin{remark}
לחלופין,
$A$
נילפוטנטית אם יש
$k \in \mbb{N}_+$
עבורו
$A^k = 0$.
הערך המינימלי של
$k$
כזה הוא אינדקס הנילפוטנטיות של
$A$.
\end{remark}

\begin{exercise}
תהי
$A \in M_n\prs{\mbb{F}}$
אלכסונית בלוקים
$\prs{n_1, \ldots, n_k}$
כאשר כל בלוק
$A_i$
נילפוטנטי מאינדקס
$k_i$.
הראו כי
$A$
נילפוטנטית מאינדקס
$\max_{i \in [k]} n_i$.
\end{exercise}

\begin{solution}
יהי
$n' \ceq \max_{i \in [k]} n_i$.
מתקיים
\[A^{n'} = \pmat{A_1^{n'} & \cdots & 0 \\ 0 & \ddots & 0 \\ 0 & 0 & A_{k}^{n'}} = 0\]
כי
$n' \geq n_i$
לכל
$i \in [k]$.
אם
$n'' < n'$
יש
$i \in [k]$
עבורו
$n_i > n''$.
אז
$A_i^{n''} \neq 0$
ממינימליות
$n_i$.
אז
$A^{n''} \neq 0$.
לכן
$n'$
אינדקס הנילפוטנטיות של
$A$.
\end{solution}

\end{document}