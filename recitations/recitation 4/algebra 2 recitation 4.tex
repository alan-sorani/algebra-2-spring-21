\documentclass[a4paper,10pt,oneside,openany]{article}

\usepackage[lang=hebrew]{maths}
\usepackage{hebrewdoc}
\usepackage{stylish}
\usepackage{lipsum}
\let\bs\blacksquare

\title{
אלגברה ב' (104168) \textenglish{---} אביב 2020-2021
\\
תרגול 4 \textenglish{---} הטלות ומרחבים עצמיים מוכללים
}
\author{אלעד צורני}
\date{\today}

\begin{document}
\maketitle

\section{הטלות}

\subsection{חזרה}

\begin{definition}[הטלה במקביל לתת־מרחב]
יהיו
$V$
מרחב וקטורי מעל
$\mbb{F}$
ו־%
$U,W \leq V$
עבורם
$U \oplus W = V$.
נגדיר את
\emph{ההטלה על
$U$
במקביל ל־%
$W$}
להיות ההעתקה
\begin{align*}
P_U \colon V &\to V \\
\text{.} \hphantom{lala} u+w &\mapsto u
\end{align*}
\end{definition}

\begin{remark}
ההטלה
$P_U$
תלויה ב־%
$W$.
\end{remark}

\begin{example}
יהי
$V = \mbb{R}^2$
ויהיו
\begin{align*}
U &= \spn\prs{e_1} \\
W_1 &= \spn\prs{e_2} \\
W_2 &= \spn\prs{e_1 + e_2}
\end{align*}
ויהיו
$P_1, P_2$
ההטלות על
$U$
במקביל ל־%
$W_1, W_2$
בהתאמה.

מתקיים
\[P_1 \prs{e_1 + e_2} = e_1\]
אבל
\[\text{.} P_2 \prs{e_1 + e_2} = e_1 + e_2\]
\end{example}

\begin{definition}[הטלה]
העתקה
$P \in \endo_{\mbb{F}}\prs{V}$
נקראת הטלה אם קיימים
$U,W \leq V$
עבורם
$P = P_U$.
\end{definition}

\begin{fact}
העתקה
$P \in \endo_{\mbb{F}}\prs{V}$
היא הטלה אם ורק אם
$P^2 = P$.
\end{fact}

\subsection{תרגילים}

\begin{exercise}
מצאו את כל הערכים העצמיים האפשריים של הטלה.
\end{exercise}

\begin{solution}
יהי
$\lambda$
ערך עצמי של הטלה
$P$
עם וקטור עצמי
$v$.
אז
\[\lambda^2 v = P^2 v = P v = \lambda v\]
לכן
$\lambda \in \set{0,1}$.

אפשר לקחת
$P = \id_V$
או
$P = 0_V$
ולקבל את שתי האופציות האלו.
\end{solution}

\begin{exercise}
תהי
$P \in \endo_{\mbb{F}}\prs{V}$
הטלה.
הראו כי
$P$
הטלה על
$U \leq V$
במקביל ל־%
$W \leq V$
אם ורק אם
$U = \im P$
וגם
$W = \ker P$.
\end{exercise}

\begin{solution}
\begin{itemize}
\item נניח כי
$P$
הטלה על
$U$
במקביל ל־%
$W$.
לכל
$u \in U$
מתקיים
\[P\prs{u} = P\prs{u+0} = u\]
לכן
$U \subseteq \im P$.
להיפך, אם
$v \in \im P$
יש
$u \in U, w \in W$
עבורם
$v = P\prs{u+w} = u \in U$
לכן
$\im P = U$.

יהי
$w \in W$.
מתקיים
\[P\prs{w} = P\prs{0+w} = 0\]
לכן
$W \subseteq \ker\prs{P}$.
להיפך, נניח כי
$v \in \ker P$
ונכתוב
$v = u + w$.
אז
\[0 = Pv = u\]
לכן
$v = w \in W$,
לכן
$\ker P = \im P$.

\item נניח כי
$\ker P = W$
וכי
$\im P = U$.
נניח כי
$v \in U \cap W$.
ראינו כי
$\rest{P}{\im P} = \id_{\im P}$
לכן
$Pv = v$.
אבל גם
$Pv = 0$,
לכן
$v = 0$
ולכן
$U,W$
זרים.
ממשפטי המימדים מתקיים
\begin{align*}
\dim \prs{\im P + \ker P} &=
\dim\prs{\im P} + \dim\prs{\ker P} - \dim\prs{\im P \cap \ker P}
\\&=
\dim\prs{\im P} + \dim\prs{\ker P}
\\ \text{.} \hphantom{\dim \prs{\im P + \ker P}} &= V
\end{align*}
לכן
$V = U \oplus W$.
עבור
$u \in U, w \in W$
נכתוב
$u = P\prs{u'}$.
אז
\begin{align*}
P\prs{u + w} &= P\prs{u} + P\prs{w}
\\&= P\prs{u}
\\&= u
\end{align*}
כאשר השוויון האחרון נכון כי
$\rest{P}{\im P} = \id_{\im P}$.
לכן
$P$
ההטלה על
$\im P$
בניצב ל־%
$\ker P$.
\end{itemize}
\end{solution}

\begin{remark}
בכיוון השני בפתרון הראינו שעבור הטלה
$P$
מתקיים
$V = \im P \oplus \ker P$.
\end{remark}

\begin{exercise}
תהי
$P \in \endo_{\mbb{F}}\prs{V}$
הטלה.
הראו כי
$P$
לכסינה.
\end{exercise}

\begin{solution}
יהי
$B$
בסיס עבור
$\im P$
ויהי
$C$
בסיס עבור
$\ker P$.
ראינו כי
$\rest{P}{\im P} = \id$
וכי
$\rest{P}{\ker P} = 0$
לכן המטריצה המייצגת של
$P$
בבסיס
$B * C$
היא
\begin{align*}
\text{.} \pmat{1 & 0 & & \cdots & & 0 \\ 0 & \ddots & 0 & \cdots & & 0 \\ & & 1 & 0 & \cdots & 0 \\ \vdots & & & 0 & & \vdots \\ & & & & \ddots & \\ 0 & & \cdots & & & 0}
\end{align*}
\end{solution}

\begin{exercise}
תהי
$P \in \endo_{\mbb{F}}\prs{V}$
הטלה ותהי
$T \in \endo_{\mbb{F}}\prs{V}$
כלשהי.
הראו כי
$\ker P$
הוא
$T$%
־שמור אם ורק אם יש
$\hat{T} \colon \ker P \to \ker P$
לינארית המקיימת
\[\text{.} \hat{T} \circ P = P \circ T\]
כלומר, כך שהדיאגרמה
\[\begin{tikzcd}
V \arrow[r, "T"] \arrow[d, swap, "P"] & V \arrow[d, "P"] \\
\im P \arrow[r, swap, "\hat{T}"] & \im P
\end{tikzcd}
\]
מתחלפת.
\end{exercise}

\begin{solution}
\begin{itemize}
\item נניח כי
$W$
הוא
$T$%
־שמור ויהי
$u \in \im P$.
יש
$v \in V$
עבורו
$P\prs{v} = u$,
ואז
$P\prs{u} = P^2\prs{u} = P\prs{v} = u$.
נגדיר
\[\text{.} \hat{T}\prs{u} = P \circ T\prs{u}\]

יהי
$v \in V$,
צריך להראות שמתקיים
\[\text{.} P \circ T \prs{v} = \hat{T} \circ P\prs{v}\]
נכתוב
$v = u+w$
כאשר
$u \in \im P, w \in \ker P$.
אז
\[\hat{T} \circ P \prs{v} = \hat{T} \prs{u} = P \circ T \prs{u}\]
וגם
\[T\prs{w} \in \ker P\]
לכן
\[\text{.} P \circ T \prs{u} = P \circ T \prs{u} + P \circ T \prs{w} = P \circ T \prs{u+w} = P \circ T\prs{v}\]
בסך הכל
\[\text{,} \hat{T} \circ P \prs{v} = P \circ T \prs{u} = P \circ T \prs{v}\]
כנדרש.

\item נניח כי יש העתקה
$\hat{T}$
כמתואר ויהי
$w \in \ker P$.
אז
\[PT\prs{w} = \hat{T}P\prs{w} = \hat{T}\prs{0} = 0\]
ולכן
$T\prs{w} \in \ker P$.
נקבל
$T\prs{\ker P} \subseteq \ker P$.
\end{itemize}
\end{solution}

\section{מרחבים עצמיים מוכללים והפולינום האופייני}

\subsection{תזכורת}

\begin{definition}[מרחב עצמי מוכלל]
תהי
$T \in \endo_{\mbb{F}}\prs{V}$
ונסמן
$n \ceq \dim_{\mbb{F}}\prs{V}$.
\emph{המרחב העצמי המוכלל של
$T$
עבור ערך עצמי
$\lambda$}
הוא
\[\text{.} V_{\lambda}' \ceq \ker \prs{\prs{T-\lambda \id_V}^{n}}\]
\emph{הריבוי האלגברי של
$\lambda$}
הוא
\[\text{.} r_{a,T}\prs{\lambda} \ceq \dim V_{\lambda}'\]
\end{definition}

\begin{definition}[פולינום אופייני]
יהי
$V$
מרחב וקטורי סוף־מימדי.
\emph{הפולינום האופייני}
של
$T \in \endo_{\mbb{F}}\prs{V}$
שערכיה העצמיים השונים הם
$\prs{\lambda_i}{i \in [k]}$
הוא
\begin{align*}
\text{.} p_T\prs{x} \ceq \sum_{i \in [k]} \prs{x-\lambda_i}^{r_{a,T}\prs{\lambda_i}}
\end{align*}
\end{definition}

\begin{theorem}[קיילי־המילטון]
לכל
$T \in \endo_{\mbb{F}}\prs{V}$
מתקיים
\[\text{.} p_T\prs{T} = 0\]
\end{theorem}

\subsection{תרגילים}

\begin{exercise}
תהי
\[\text{.} A = \pmat{1 & 1 & 0 \\ 0 & 1 & 2 \\ 0 & 0 & 3}\]
מצאו את המרחבים העצמיים ואת המרחבים העצמיים המוכללים של
$A$.
\end{exercise}

\begin{solution}
הערכים העצמיים של
$A$
הם
$1$
מריבוי אלגברי
$2$
ו־%
$3$
מריבוי אלגברי
$1$.
נחשב.
\begin{align*}
\ker\prs{A-I} &= \ker\pmat{0 & 1 & 0 \\ 0 & 0 & 2 \\ 0 & 0 & 2}
\\&= \spn\set{e_1} \\
\ker\prs{A-I}^2 &= \ker\pmat{0 & 0 & 4 \\ 0 & 0 & 4 \\ 0 & 0 & 4}
\\&= \spn\set{e_1, e_2} \\
\ker\prs{A-3I} &= \ker\pmat{-2 & 1 & 0 \\ 0 & -2 & 2 \\ 0 & 0 & 0}
\\&= \ker\pmat{2 & -1 & 0 \\ 0 & 1 & -1 \\ 0 & 0 & 0}
\\&= \ker\pmat{2 & 0 & -1 \\ 0 & 1 & - 1 \\ 0 & 0 & 0}
\\&= \spn\set{\pmat{1 \\ 2 \\ 2}}
\end{align*}
בסך הכל
\begin{align*}
V_1 &= \spn\set{e_1} \\
V_1' &= \spn\set{e_1, e_2} \\
\text{.} V_3 = V_3' &= \spn\set{\pmat{1\\2\\2}}
\end{align*}
\end{solution}

\begin{exercise}
תהי
$T \in \endo_{\mbb{C}}\prs{V}$
ויהי
$\lambda$
ערך עצמי של
$T$.
יהי
$\prs{v_1, \ldots, v_r}$
בסיס של
$V_{\lambda}'$
כאשר
$\prs{v_1, \ldots, v_s}$
בסיס של
$V_\lambda$.
הראו כי
$\hat{V}_\lambda \ceq \spn\set{v_{s+1}, v_{s+2}, \ldots, v_r}$
אינו
$T$%
־שמור.
\end{exercise}

\begin{solution}
אם
$\hat{V}_\lambda$
מרחב
$T$%
־שמור, לההעתקה
$\rest{T}{\hat{V}_\lambda}$
יש ערך עצמי
$\mu$.
אבל, אז
$\mu$
ערך עצמי של
$T$.
אם
$\mu \neq \lambda$
נקבל
$V'_\mu \cap V'_\lambda \neq \set{0}$
בסתירה.
אחרת, יש ל־%
$\rest{T}{\hat{V}_\lambda}$
וקטור עצמי עם ערך עצמי
$\lambda$,
כלומר
$\hat{V}_\lambda \cap V_\lambda \neq \set{0}$
בסתירה להגדרת
$\hat{V}_\lambda$.
\end{solution}

\begin{exercise}
יהי
$V$
מרחב וקטורי מעל שדה
$\mbb{F}$
כך שמתקיים
$1_{\mbb{F}} + 1_{\mbb{F}} \neq 0$,
ויהיו
$P_1, P_2 \colon V \to V$
הטלות.
\begin{enumerate}
\item הוכיחו כי
$\id_V + P_1$
הפיכה ומצאו את
$\prs{\id_V + P_1}^{-1}$.
\item נניח כי
$P_1 + P_2$
הטלה. הוכיחו כי
$P_1P_2 = 0$.
\item מצאו דוגמאות נגדיות לשני הסעיפים במקרה
$1_{\mbb{F}} + 1_{\mbb{F}} = 0$.
\end{enumerate}
\end{exercise}

\begin{solution}
\begin{enumerate}
\item
$P_1$
לכסינה עם ערכים עצמיים
$0,1$
בלבד, לכן הערכים העצמיים של
$I + P_1$
הם
$1,2$
בלבד. בפרט
$0$
אינו ערך עצמי ולכן
$P_1$
הפיכה.
יש מטריצה אלכסונית
$D$
ובסיס
$B$
כך ש־%
$\brs{T}_B$
אלכסונית עם איברי אלכסון בקבוצה
$\set{0,1}$.
אז
\[\brs{T + \id_V}_B = \brs{T}_B + \brs{\id_V}_B = D + I\]
עם איברי אלכסון בקבוצה
$\set{1,2}$.
אם זאת מטריצת היחידה,
$D = 0$
ואז
$P = 0$
ונקבל
$\prs{\id_V+P}^{-1} = \id_V$.
אם
$P = \id_V$
נקבל
$\prs{\id_V + P}^{-1} = \frac{1}{2} \id_V$.
אחרת,
\[D = \pmat{0 & 0 \\ 0 & 1}\]
ונקבל
\[\prs{I - \frac{1}{2} D} \prs{I + D} = I\]
לכן
\[\text{.} \prs{\id_V - \frac{1}{2} P}\prs{\id_V + P} = \id_V\]

נציג דרך נוספת.
מתקיים
$P^2 = P$
לכן
$P^2 - P = 0$.
אז
\[\text{.} \prs{P + \id_V - \id_V}^2 - \prs{P + \id_V - \id_V} = 0\]
נפתח את הביטוי, נכפול ב־%
$\prs{P + \id_V}^{-1}$
ונקבל את התוצאה.
\item מתקיים
\[P_1 + P_2 = \prs{P_1 + P_2}^2 = P_1^2 + P_1 P_2 + P_2 P_1 + P_2^2 = P_1 + P_1 P_2 + P_2 P_1 + P_2\]
לכן
\[\text{.} P_1 P_2 = - P_2 P_1\]
נניח בשלילה ש־%
$P_1 P_2 \neq 0$.
אז יש
$v \in V$
כך ש־%
$u \ceq P_2\prs{v} \notin \ker\prs{P_1}$.
אז
\[-P_2P_1\prs{v} = P_1 P_2 \prs{u} = - P_2 P_1\prs{u} = -P_2 P_1 P_2 \prs{v} = P_2^2 P_1\prs{v} = P_2 P_1\prs{v}\]
לכן
$P_2 P_1\prs{v} = 0$.
אז
\[P_1\prs{u} = P_1 P_2\prs{v} = -P_2 P_1\prs{v} = 0\]
כלומר
$u \in \ker\prs{P_1}$,
בסתירה.
\item
ניקח
$P_1 = P_2 = \id_V$.
אז
$P_1 + P_2 = 0$
הטלה אבל
$P_1 P_2 = \id_V \neq 0$.
כמו כן
$\id_V + P_1 = 2\id_V = 0$
אינה הפיכה.

\end{enumerate}
\end{solution}

\begin{exercise}
תהי
$T \in \endo_{\mbb{C}}\prs{V}$
ויהיו
$\prs{V_i}_{i \in [m]}$
תת־מרחבים
$T$%
־שמורים של
$V$.
לכל
$i \in [m]$
יהי
$p_i \ceq p_{\rest{T}{V_i}}$.
הוכיחו כי
\[\text{.} p_T = \prod_{i \in [m]} p_i\]
\end{exercise}

\begin{solution}
ראינו בהרצאה כי כל אנדומורפיזם של מרחב וקטורי יש בסיס לפיו הוא מיוצג כמטריצה משולשת עליונה. לכן יש בסיסים
$B_1, \ldots, B_m$
עבור
$V_1, \ldots, V_m$
כך ש־%
$\brs{\rest{T}{V_i}}{B_i}$
משולשות עליונות.
יהי
$B = B_1 * \ldots * B_m$,
ואז
$\brs{T}_B$
משולשת עליונה.
הריבוי האלגברי של
$\lambda$
הוא מספר הפעמים שהוא מופיע על האלכסון, לכן נקבל
\begin{align*}
r_{a,T}\prs{\lambda} = \sum_{i \in [m]} r_{a,\rest{T}{V_i}}\prs{\lambda}
\end{align*}
ולכן החזקה של
$x-\lambda$
ב־%
$p_T$
היא מכפלת החזקות ב־%
$p_i$.
אלגברית,
\begin{align*}
r_{a,T}\prs{\lambda} &= \dim \ker \prs{T - \lambda \id_V}
\\&= \dim \ker \prs{\bigoplus_{i \in [m]} \rest{T - \lambda \id_V}{V_i}}
\\&= \dim \bigoplus_{i \in [m]} \ker\prs{\rest{T - \lambda \id_V}{V_i}}
\\ \text{.} \hphantom{r_{a,T}\prs{\lambda}} &= \sum_{i \in [m]} r_{a,\rest{T}{V_i}}\prs{\lambda}
\end{align*}
\end{solution}

\end{document}