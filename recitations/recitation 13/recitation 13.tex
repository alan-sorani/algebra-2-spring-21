\documentclass[a4paper,10pt,oneside,openany]{article}

\usepackage[lang=hebrew]{maths}
\usepackage{hebrewdoc}
\usepackage{stylish}
\usepackage{lipsum}
\let\bs\blacksquare
\usepackage{ytableau}

\usepackage{graphicx, txfonts}

\newcommand{\heart}{\ensuremath\varheartsuit}

\title{
אלגברה ב' (104168) \textenglish{---} אביב 2020-2021
\\
תרגול 13 \textenglish{---}
קריטריון סילבסטר
ותרגילים ממבחנים
}
\author{אלעד צורני}
\date{\today}

\begin{document}
\maketitle

\section{קריטריון סילבסטר}

\begin{theorem}[קריטריון סילבסטר]
$A \in M_n\prs{\mbb{R}}$
מוגדרת חיובית אם ורק אם
\[\Delta_i \ceq \det \pmat{a_{1,1} & \cdots & a_{1,i} \\ \vdots & \ddots & \vdots \\ a_{i,1} & a_{i,i}} > 0\]
לכל
$i \in [n]$.
\end{theorem}

\begin{definition}[מינור ראשי]
$\Delta_i$
נקרא
\emph{המינור הראשי ה־%
$i$}.
\end{definition}

\begin{exercise}
מצאו תנאי על המינורים של
$A \in M_n\prs{\mbb{R}}$
על מנת ש־%
$A$
תהיה מוגדרת שלילית.
\end{exercise}

\begin{solution}
נרצה וריאציה על קריטריון סילבסטר עבור מטריצות מוגדרות שלילית.
נשים לב כי
$A$
מוגדרת שלילית אם ורק אם
$-A$
מוגדרת חיובית, כיוון ש־%
$\trs{Av, v} < 0$
אם ורק אם
$\trs{-Av, v} = -\trs{A,v} > 0$.
נסמן ב־%
$\Delta'_i$
את המינור הראשי ה־%
$i$
של
$-A$.
אז
\[\text{.} \Delta_i' = \det\prs{-\pmat{a_{1,1} & \cdots & a_{1,i} \\ \vdots & \ddots & \vdots \\ a_{i,1} & \cdots & a_{i,i}}} = \prs{-1}^i \Delta_i\]
כעת,
$A$
מוגדרת שלילית אם ורק אם
$-A$
מוגדרת חיובית, אם ורק אם
$\Delta_i' > 0$
לכל
$i \in \brs{n}$
אם ורק אם
$\prs{-1}^i \Delta_i > 0$
לכל
$i \in \brs{n}$.
\end{solution}

\begin{exercise}
הראו כי
\[A \ceq \pmat{-1 & 1 & 0 & 1 \\ 1 & -3 & 0 & 0 \\ 0 & 0 & -1 & -1 \\ 1 & 0 & -1 & -3}\]
מוגדרת שלילית.
\end{exercise}

\section{תרגילים ממבחנים}

\begin{exercise}%1
\begin{enumerate}
\item הראו כי ל־%
$A \ceq \pmat{0&1\\1&1}, B \ceq \pmat{1&1\\1&0} \in M_2\prs{\mbb{Z}_2}$
אין צורת ז'ורדן ב־%
$M_2\prs{\mbb{Z}_2}$.

\item הראו כי כל מטריצה אחרת ב־%
$M_2\prs{\mbb{Z}_2}$
דומה למטריצת ז'ורדן.
\end{enumerate}
\end{exercise}

\begin{solution}
\begin{enumerate}
\item
מתקיים
\begin{align*}
\det\prs{A}, \det\prs{B} &= -1 = 1 \\
\text{.} \tr\prs{A}, \tr\prs{B} &=  1
\end{align*}
מהשוויון הראשון הערך העצמי היחיד של
$A,B$
הוא
$1$.
אז אם יש צורת ז'ורדן היא
$\pmat{1 & 0 \\ 0 & 1}$
או
$\pmat{1 & 1 \\ 0 & 1}$,
אך בשני המקרים האלו העקבה היא
$2 = 0$,
בסתירה.

\item
תהי
$C \in M_2\prs{\mbb{Z}_2} \setminus \set{A,B}$.
אם
$\tr\prs{C} = 1$
נקבל
\[\text{.} C \in \set{\pmat{0 & 1 \\ 0 & 1}, \pmat{0 & 0 \\ 1 & 1}, \pmat{1 & 1 \\ 0 & 0}, \pmat{1 & 0 \\ 1 & 0}, \pmat{1 & 0 \\ 0 & 0}, \pmat{0 & 0 \\ 0 & 1}}\]
אז
\begin{enumerate}
\item אם
$C = \pmat{0 & 0 \\ 1 & 1}$
נכתוב
$B = \pmat{e_1 + e_2, e_2}$
ואז
\begin{align*}
C \prs{e_1 + e_2} &= C e_1 + C e_2 = e_2 + e_2 = 0 \\
C e_2 &= e_2
\end{align*}
לכן
$B$
בסיס ז'ורדן של
$C$
וצורת ז'ורדן לפיו היא
$\pmat{0 & 0 \\ 0 & 1}$.

\item אם
$C = \pmat{0 & 1 \\ 0 & 1}$
היא המטריצה המשוחלפת של
המטריצה הקודמת, ולכן לפי תרגיל משיעורי הבית יש לה צורת ז'ורדן והיא זהה לזאת הקודמת.

\item אם
$C = \pmat{1 & 1 \\ 0 & 0}$
נכתוב
$B \ceq \pmat{e_1 + e_2, e_1}$
ואז
$B$
בסיס ז'ורדן כמו במקרה הראשון.

\item אם
$C = \pmat{1 & 0 \\ 1 & 0}$
היא המטריצה המשוחלפת של המטריצה הקודמת, ולכן לפי תרגיל משיעורי הבית יש לה צורת ז'ורדן והיא זהה לזאת הקודמת.

\item שאר המטריצות כבר בצורת ז'ורדן.
\end{enumerate}

אם
$\tr\prs{C} = 0$
נקבל
\[\text{.} C \in \set{\pmat{0 & 1 \\ 1 & 0}, \pmat{1 & 1 \\ 1 & 1}, \pmat{0 & 0 \\ 1 & 0}, \pmat{1 & 0 \\ 1 & 1}, \pmat{0 & 1 \\ 0 & 0}, \pmat{1 & 1 \\ 0 & 1}, \pmat{0 & 0 \\ 0 & 0}, \pmat{1 & 0 \\ 0 & 1}}\]
אז
\begin{enumerate}
\item אם
$C = \pmat{0 & 1 \\ 1 & 0}$
נכתוב
$B \ceq \prs{e_1 + e_2, e_1}$
ואז
\begin{align*}
C\prs{e_1 + e_2} &= C e_1 + C e_2 = e_2 + e_1 = e_1 + e_2 \\
C e_1 &= e_2 = e_1 + e_2 - e_1 = \prs{e_1 + e_2} + e_1
\end{align*}
ולכן
$B$
בסיס ז'ורדן וצורת ז'ורדן היא
$\pmat{1 & 1 \\ 0 & 1}$.

\item אם
$C = \pmat{1 & 1 \\ 1 & 1}$
נכתוב
$B = \pmat{e_1 + e_2, e_1}$
ואז
\begin{align*}
C \prs{e_1 + e_2} &= 2\prs{e_1 + e_2} = 0 \\
C e_1 &= e_1 + e_2
\end{align*}
ולכן
$B$
בסיס ז'ורדן וצורת ז'ורדן היא
$\pmat{1 & 1 \\ 0 & 1}$.

\item אם
$C = \pmat{0 & 0 \\ 1 & 0}$
או
$C = \pmat{1 & 0 \\ 1 & 1}$
היא מטריצה משוכלפת של מטריצה ז'ורדן ולכן יש לה צורת ז'ורדן עם בסיס
$B = \pmat{e_2, e_1}$.

\item שאר האופציות כבר בצורת ז'ורדן.
\end{enumerate}
\end{enumerate}
\end{solution}

\begin{exercise}%2
יהי
$V$
מרחב מכפלה פנימית מרוכב סוף־מימדי.

\begin{enumerate}
\item

תהי
$T \in \endo_{\mbb{C}}\prs{V}$.
הראו ש־%
$T$
נורמלית אם ורק אם יש פולינום
$p \in \mbb{C}\brs{x}$
עבורו
$T^* = p\prs{T}$.

היעזרו בעובדה הבאה.

\begin{theorem}[אינטרפולציית לגרנג']
תהיינה
$x_1, \ldots, x_k, y_1, \ldots, y_k \in \mbb{C}$.
קיים פולינום
$p \in \mbb{C}\brs{x}$
ממעלה
$k+1$
המקיים
$p\prs{x_i} = y_i$
לכל
$i \in \brs{k}$.
\end{theorem}

\item

הוכיחו כי
$T_1, T_2$
מתחלפות אם ורק אם
$T_1^*, T_2$
מתחלפות.
\end{enumerate}
\end{exercise}

\begin{solution}
\begin{enumerate}
\item אם יש
$p \in \mbb{C}\brs{x}$
עבורו
$T^* = p\prs{T}$
אז
$T$
נורמלית כי העתקה מתחלפת עם פולינום בה.

בכיוון השני, נניח כי
$T$
נורמלית.
אז יש בסיס
$B$
אורתונורמלי עבורו
\[\brs{T}_B = \mrm{diag}\prs{\lambda_1, \ldots, \lambda_n}\]
עבור ערכים
$\lambda_i \in \mbb{C}$.
כיוון ש־%
$B$
אורתונורמלי מתקיים
\[\text{.}\brs{T^*}_B = \brs{T}_B^* = \mrm{diag}\prs{\bar{\lambda}_1, \ldots, \bar{\lambda}_n}\]
מאינטרפולציית לגרנג', קיים פולינום
$p \in \mbb{C}\brs{x}$
עבורו
$p\prs{\lambda_i} = \bar{\lambda}_i$
אז
\[\brs{T^*}_B = p\prs{\brs{T}_B} = \brs{p\prs{T}}_B\]
ולכן
$T^* = p\prs{T}$,
כנדרש.

\item נניח כי
$T_1 T_2 = T_2 T_1$
ונראה כי
$T_1^* T_2 = T_2 T_1^*$.
הכיוון השני ינבע על ידי שימוש באותה תכונה עם
$T_1^*$
במקום
$T_1$.

$T_1$
נורמלית ולכן
$T_1 = p\prs{T_1}$
עבור \emph{איזשהו}
$p \in \mbb{C}\brs{x}$.
נכתוב
$p\prs{x} = \sum_{i \in [d]} a_i x^i$
ואז
\begin{align*}
T_1^* T_2 &= p\prs{T_1} T_2
\\&= \sum_{i \in [d]} a_i T_1^i T_2
\\&= \sum_{i \in [d]} a_i T_2 T_1^i
\\&= T_2 \sum_{i \in [d]} a_i T_1^i
\\&= T_2 p\prs{T_1}
\\&= T_2 T_1^*
\end{align*}
כנדרש.
\end{enumerate}
\end{solution}

\begin{exercise}%3
יהי
$V$
מרחב מכפלה פנימית מרוכב ממימד סופי ותהיינה
$S,T \in \endo_{\mbb{C}}\prs{V}$
יוניטריות.
הוכיחו או מצאו דוגמא נגדית עבור כל אחד מהסעיפים הבאים.

\begin{enumerate}
\item $S+T$
נורמלית.
\item $S \circ T$
נורמלית.
\item אם
$S,T$
מתחלפות אז
$S+T$
נורמלית.
\end{enumerate}
\end{exercise}

\begin{solution}
\begin{enumerate}
\item מתקיים
\begin{align*}
\prs{S+T}^* \circ \prs{S+T} &= \prs{S^* + T^*} \circ \prs{S+T}
\\&=
S^* \circ S + S^* \circ T + T^* \circ S + T^* \circ T
\\&=
2 \id_V + S^* \circ T + T^* \circ S
\end{align*}
וגם
\begin{align*}
\prs{S+T} \circ \prs{S+T}^* &= \prs{S + T} \circ \prs{S^* + T^*}
\\&= S \circ S^* + S \circ T^* + T \circ S^* + T \circ T^*
\\ \hphantom{\prs{S+T} \circ \prs{S+T}^*} &= 2\id_V + S \circ T^* + T \circ S^*
\end{align*}
לכן נרצה לדעת האם בהכרח
\[\text{.} S^* \circ T + T^* \circ S = S \circ T^* + T \circ S^*\]

כדי למצוא דוגמא נגדית, נצטרך שלפחות אחת מבין
$S,T$
לא תהיה צמודה לעצמה.
ניקח
\begin{align*}
S \colon \mbb{C}^2 &\to \mbb{C}^2 \\
\pmat{x \\ y} &\mapsto \pmat{y \\ x}
\end{align*}
וכן
\begin{align*}
T \colon \mbb{C}^2 &\to \mbb{C}^2 \\
\text{.} \pmat{x \\ y} &\mapsto \pmat{\frac{x+y}{\sqrt{2}} \\ \frac{x-y}{\sqrt{2}}}
\end{align*}
$E$
בסיס אורתונורמלי ולכן
\begin{align*}
\brs{S^*}_E &= \brs{S}_E^t = \pmat{0 & 1 \\ 1 & 0}^t = \pmat{0 & 1 \\ 1 & 0} = \brs{S}_E
\end{align*}
וגם
\begin{align*}
\text{.} \brs{T^*}_E &= \brs{T}_E^t = \pmat{1/\sqrt{2} & 1/\sqrt{2} \\ }^t = \pmat{ 0 & 1 \\ 0 & 1}
\end{align*}
כעת
\begin{align*}
\brs{S^* \circ T + T^* \circ S}_E &=
\brs{S^*}_E \brs{T}_E + \brs{T^*}_E \brs{S}_E
\\&=
\pmat{0 & 1 \\ 1 & 0} \pmat{0 & 0 \\ 1 & 1} + \pmat{0 & 1 \\ 0 & 1} \pmat{0 & 1 \\ 1 & 0}
\\&=
\pmat{1 & 1 \\ 0 & 0} + \pmat{1 & 0 \\ 1 & 0}
\end{align*}
אבל
\begin{align*}
\pmat{S \circ T^* + T \circ S^*}_E &= \brs{S}_E \brs{T^*}_E + \brs{T}_E \brs{S^*}_E
\\&= \pmat{0 & 1 \\ 1 & 0} \pmat{0 & 1 \\ 0 & 1} + \pmat{0 & 0 \\ 1 & 1} \pmat{0 & 1 \\ 1 & 0}
\\&= \pmat{0 & 0 \\ 1 & 1} + \pmat{1 & 0 \\ 1 & 0}
\end{align*}
ואלו מטריצות שונות.

\item מתקיים
\begin{align*}
\prs{S \circ T}^* \circ \prs{S \circ T} &= T^* \circ S^* \circ S \circ T
\\&=
T^* \circ S \circ S^* \circ T
\\&=
T ^* \circ T
\\&=
\id_V
\end{align*}
לכן
$S \circ T$
יוניטרית ובפרט נורמלית.

\begin{remark}
ראיתם בתרגול שהרכבה של העתקות יוניטריות היא יוניטרית, ולכן אין צורך בפירוט מעבר לכך כפי שהתרגיל מנוסח.
\end{remark}

\item 
כמו מקודם,
מתקיים
\begin{align*}
\prs{S+T}^* \circ \prs{S+T} &= \prs{S^* + T^*} \circ \prs{S+T}
\\&=
S^* \circ S + S^* \circ T + T^* \circ S + T^* \circ T
\\&=
2 \id_V + S^* \circ T + T^* \circ S
\end{align*}
וגם
\begin{align*}
\prs{S+T} \circ \prs{S+T}^* &= \prs{S + T} \circ \prs{S^* + T^*}
\\&= S \circ S^* + S \circ T^* + T \circ S^* + T \circ T^*
\\ \hphantom{\prs{S+T} \circ \prs{S+T}^*} &= 2\id_V + S \circ T^* + T \circ S^*
\end{align*}
כעת,
$S,T$
מתחלפות, לכן גם
$S^*, T$
מתחלפות וגם
$S, T^*$
מתחלפות, ונקבל שוויון.
\end{enumerate}
\end{solution}

\begin{exercise}
יהיו
$v_1, \ldots, v_n \in \mbb{R}^n$.
הראו כי
$B \ceq \prs{v_1, \ldots, v_n}$
בסיס של
$\mbb{R}^n$
אם ורק אם למטריצת
\textenglish{Gram}
של
$B$
\[\mrm{Gr}\prs{B} \ceq \pmat{\trs{v_1, v_1} & \cdots & \trs{v_1, v_n} \\ \vdots & \ddots & \vdots \\ \trs{v_n, v_1} & \cdots & \trs{v_n, v_n}} = \prs{\trs{v_i, v_j}}_{i,j \in \brs{n}}\]
יש דטרמיננטה חיובית.
\end{exercise}

\begin{solution}
נניח כי
$B$
בסיס.
אז
$\mrm{Gr}\prs{B}$
המטריצה המייצגת של המכפלה הפנימית הסטנדרטית לפי
$B$.
לכן קיים
$a \in \mbb{R}$
עבורו
\[\det\prs{\mrm{Gr}\prs{B}} = a^2 \det\prs{I_n} = a^2\]
(כי
$\mrm{Gr}\prs{B} = P^t I_n P$
וכי הדטרמיננטה כפלית).
כיוון ש־%
$\mrm{Gr}\prs{B}$
הפיכה, לא יתכן
$a = 0$
ולכן
$\det\prs{\mrm{Gr}\prs{B}} = a^2 > 0$.

בכיוון השני, נניח ש־%
$B$
אינו בסיס. אז יש סקלרים
$\alpha_1, \ldots, \alpha_{n-1} \in \mbb{R}$
עבורם
\[\text{.} v_n = \sum_{i \in [n-1]} \alpha_i v_i\]
נקבל
\begin{align*}
\text{.} \prs{\mrm{Gr}\prs{B}} &= \pmat{\trs{v_1, v_1} & \cdots & \trs{v_1, \sum_{i \in [n-1]} \alpha_i v_i} \\ \vdots & \ddots & \vdots \\ \trs{v_n, v_1} & \cdots & \trs{v_i, \sum_{i \in [n-1]} \alpha_i v_i}}
=
\pmat{\trs{v_1, v_1} & \cdots & \sum_{i \in [n-1]} \alpha_i \trs{v_1, v_i} \\ \vdots & \ddots & \vdots \\ \trs{v_n, v_1} & \cdots & \sum_{i \in [n-1]} \alpha_i \trs{v_i, v_i}}
\end{align*}
לכן, העמודה הימנית של
$\mrm{Gr}\prs{B}$
היא צירוף לינארי של שאר העמודות, ולכן
$\det\prs{\mrm{Gr}\prs{B}} = 0$.
\end{solution}

\begin{exercise}
יהי
$V$
מרחב וקטורי סוף־מימדי מעל
$\mbb{R}$.
תהי
$g$
מכפלה פנימית על
$V$
ותהי
$h$
תבנית בילינארית סימטרית על
$V$.
הוכיחו שקיים בסיס
$B$
של
$V$
עבורו
$\brs{h}_B, \brs{g}_B$
שתיהן אלכסוניות.
\end{exercise}

\begin{solution}
נסמן
$n \ceq \dim_{\mbb{R}}\prs{V}$.
יהי
$E$
בסיס אורתונורמלי של
$V$
ביחס ל־%
$g$.
אז
$\brs{g}_E = I_n$.
כעת,
$\brs{h}_E$
סימטרית כי
$h$
סימטרית, ולכן יש מטריצה
$Q \in M_n\prs{\mbb{R}}$
אורתוגונלית עבורה
$Q^t \brs{h}_E Q$
אלכסונית.
נרצה
$Q = P^B_E$
ולכן נגדיר
$B \ceq \prs{Q^{-1} v_1, \ldots, Q^{-1} v_n}$
עבור
$E \ceq \prs{v_1, \ldots, v_n}$.
אז
\begin{align*}
\brs{g}_B &= \prs{P^B_E}^t \brs{g}_E P^B_E = Q^t \brs{g}_E Q = Q^t I_n Q = Q^t Q = I_n \\
\brs{h}_B &= \prs{P^B_E}^t \brs{h}_E P^B_E = Q^t \brs{h}_E Q
\end{align*}
שתיהן אלכסוניות, כנדרש.
\end{solution}

\begin{exercise}
יהי
$V$
מרחב מכפלה פנימית ממימד
$n \in \mbb{N}$
מעל
$\mbb{R}$.
יהיו
$U,W \leq V$
עבורם
$V = U \oplus W$.
עבור
$v = u + w$
כאשר
$u \in U, w \in W$
נגדיר
$T\prs{v} = u - w$.
הניחו כי
$T$
לינארית והראו כי
$T$
צמודה לעצמה אם ורק אם
$U \perp W$.
\end{exercise}

\begin{solution}
נניח כי
$U \perp W$.
יהי
$B$
בסיס אורתונורמלי ל־%
$U$
ויהי
$C$
בסיס אורתונורמלי ל־%
$W$.
אז
$D \ceq B * C$
בסיס אורתונורמלי ל־%
$V$.
נקבל כי בבסיס זה
\[\brs{T}_D = \pmat{\brs{\rest{T}{U}}_B & 0 \\ 0 & \brs{\rest{T}{W}}_C} = \pmat{I_k & 0 \\ 0 & - I_\ell}\]
עבור
$k \ceq \dim U, \ell \ceq \dim W$.
כיוון ש־%
$D$
אורתונורמלי, מתקיים
\begin{align*}
\text{.} \brs{T^*}_D &= \brs{T}_D^t = \pmat{I_k & 0 \\ 0 & - I_{\ell}}^t = \pmat{I_k & 0 \\ 0 & - I_{\ell}} = \brs{T}_D
\end{align*}
לכן,
$T^* = T$,
כנדרש.

נניח כעת כי
$T^* = T$
ויהיו
$u \in U$
ו־%
$w \in W$.
מתקיים
\begin{align*}
\trs{u,w} &= \trs{Tu, w}
\\&= \trs{u, T^*  w}
\\&= \trs{u, T w}
\\&= \trs{u, -w}
\\&= -\trs{u,w}
\end{align*}
לכן
$\trs{u,w} = 0$
ולכן
$u \perp w$.
נקבל כי
$U \perp W$.
\end{solution}

\end{document}