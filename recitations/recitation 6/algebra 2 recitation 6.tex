\documentclass[a4paper,10pt,oneside,openany]{article}

\usepackage[lang=hebrew]{maths}
\usepackage{hebrewdoc}
\usepackage{stylish}
\usepackage{lipsum}
\let\bs\blacksquare
\usepackage{ytableau}

\usepackage{graphicx, txfonts}

\newcommand{\heart}{\ensuremath\varheartsuit}

\title{
אלגברה ב' (104168) \textenglish{---} אביב 2020-2021
\\
תרגול 6 \textenglish{---} צורת ובסיס ז'ורדן
}
\author{אלעד צורני}
\date{\today}

\begin{document}
\maketitle

\section{צורת ז'ורדן}

\subsection{חזרה}

\begin{definition}[בלוק ז'ורדן]
לכל
$\lambda \in \mbb{F}$
ולכל
$m \in \mbb{F}$
נסמן
\[J_m\prs{0} \ceq \pmat{\lambda & 1 & \cdots & & 0 \\ & \lambda & 1 & & \\ \vdots & & \ddots & \ddots & \vdots \\ & & & \lambda & 1 \\ 0 & & \cdots & & \lambda} \in M_m\prs{\mbb{F}}\]
ונקרא למטריצה זאת
\emph{בלוק ז'ורדן מגודל
$n$
עם ערך עצמי
$\lambda$}.
\end{definition}

\begin{definition}[מטריצת ז'ורדן]
מטריצה
$A$
נקראת
\emph{מטריצת ז'ורדן}
אם
$A$
מטריצת בלוקים עם בלוקים
$A_1, \ldots, A_k$
כך שכל
$A_i$
בלוק ז'ורדן.
\end{definition}

\begin{definition}[צורת ובסיס ז'ורדן של העתקה]
תהי
$T \in \endo_{\mbb{F}}\prs{V}$.
\emph{צורת ז'ורדן של
$T$}
היא מטריצה מייצגת
$\brs{T}_B$
של
$T$
שהיא מטריצת ז'ורדן.

בסיס
$B$
עבורו
$\brs{T}_B$
מטריצת ז'ורדן נקרא
\emph{בסיס ז'ורדן עבור
$T$}.
\end{definition}

\begin{theorem}[משפט ז'ורדן]
יהי
$\mbb{F}$
שדה סגור אלגברית ויהי
$V$
מרחב וקטורי מעל
$\mbb{F}$.
לכל
$T \in \endo_{\mbb{F}}\prs{V}$
יש צורת ז'ורדן, וצורת הז'ורדן של
$T$
יחידה עד כדי שינוי סדר הבלוקים.
\end{theorem}

\begin{definition}[צורת ז'ורדן של מטריצה]
תהי
$A \in M_n\prs{\mbb{F}}$.
\emph{צורת ז'ורדן של
$A$}
היא מטריצת ז'ורדן
$J$
שדומה ל־%
$A$.
\end{definition}

\begin{remark}
משפט ז'ורדן אומר שאם
$\mbb{F}$
סגור אלגברית, לכל
$A \in M_n\prs{\mbb{F}}$
יש צורת ז'ורדן. אכן, ל־%
$L_A$
יש צורת ז'ורדן, והמטריצה
$\brs{L_A}_B$
דומה ל־%
$A = \brs{L_A}_E$
כאשר
$B$
בסיס ז'ורדן.
\end{remark}

\subsection{מציאת צורת ז'ורדן}

תהי
$T \in \endo_{\mbb{F}}\prs{V}$
עבור
$\mbb{F}$
סגור אלגברית.
יהי
$\lambda \in \mbb{F}$
ערך עצמי של
$T$
ונסתכל על
$\rest{T}{V_{\lambda}'}$.
נתאר את צורת ז'ורדן של
$\rest{T}{V_{\lambda}'}$
באמצעות דיאגרמת יאנג, כאשר אורכי השורות הם גדלי הבלוקים.

\begin{example}
נסתכל על המטריצה
\begin{align*}
\text{.} A = \pmat{0 & 1 & 0 & 0 & 0 \\ 0 & 0 & 1 & 0 & 0 \\ 0 & 0 & 0 & 0 & 0 \\ 0 & 0 & 0 & 0 & 1 \\ 0 & 0 & 0 & 0 & 0} = \pmat{J_3\prs{0} & 0 \\ 0 & J_2\prs{0}}
\end{align*}
מתאימה לה הדיאגרמה הבאה.

\begin{center}
\begin{english}
\ydiagram{3,2}
\end{english}
\end{center}
\end{example}

נסמן ב־%
$b_i$
את אורך העמודה ה־%
$i$.
אז
$b_1$
הוא מספר הבלוקים שגודלם לפחות 1. זה בדיוק מספר הוקטורים העצמיים של הערך העצמי
$0$
כי כל בלוק מתאים לשרשרת
\[\prs{\prs{T-\lambda \id_V}^{k-1}\prs{v} , \ldots, \prs{T-\lambda \id_V}\prs{v}, v}\]
כאשר
$\prs{T - \lambda \id_V}^k\prs{v} = 0$.

באופן כללי אפשר לראות כי
$b_i$
הוא מספר הבלוקים מגודל לפחות
$i$,
וכי
$\dim\ker\prs{T-\lambda \id_V}^i$
הוא הסכום
$b_1 + \ldots + b_i$.

מספר הבלוקים מגודל
$j$
הוא מספר הבלוקים מגודל לכל הפחות
$j$
פחות מספר הבלוקים מגודל גדול מ־%
$j$.
מספר זה שווה
$b_j - b_{j+1}$.
אבל, מהמשוואה
\[\dim\ker\prs{T-\lambda \id_V}^i = b_1 + \ldots + b_i\]
נקבל
\[b_i = \dim\ker\prs{T-\lambda \id_V}^i - \dim\ker\prs{T-\lambda \id_V}^{i-1}\]
ולכן מספר הבלוקים מגודל
$j$
הוא
\begin{align*}
b_j - b_{j+1} &= \dim \ker\prs{T - \lambda \id_V}^j - \dim\ker\prs{T-\lambda \id_V}^{j-1} - \dim\ker\prs{T-\lambda \id_V}^{j+1} + \dim\ker\prs{T-\lambda \id_V}^{j}
\\\text{.} \hphantom{b_j - b_{j+1}} &= 2\dim \ker\prs{T - \lambda \id_V}^j - \dim\ker\prs{T-\lambda \id_V}^{j+1} - \dim\ker\prs{T-\lambda \id_V}^{j-1}
\end{align*}

\subsection{תרגילים}

\begin{exercise}
מצאו את צורת ז'ורדן של המטריצה
\[\text{.} A = \pmat{1 & 1 & \cdots & 1 & 1 \\ 0 & 1 & \cdots & 1 & 1 \\ 0 & 0 & \ddots & \vdots & \vdots \\ \vdots & \vdots & \ddots & 1 & 1 \\ 0 & 0 & \cdots & 0 & 1} \in M_n\prs{\mbb{C}}\]
\end{exercise}

\begin{solution}
המטריצה משולשת עליונה ולכן הערכים העצמיים של
$A$
על האלכסון. נקבל כי
$1$
הערך העצמי היחיד וכי הוא מריבוי
$n$.
הדרגה של
$A - I$
היא
$n-1$
ולכן הריבוי הגיאומטרי של
$1$
הוא
$1$.
לכן יש בלוק יחיד בצורת ז'ורדן, ונקבל כי צורת ז'ורדן היא
$J\prs{A} = J_n\prs{1}$.
\end{solution}

\begin{exercise}
חשבו את צורת ז'ורדן
$J\prs{A}$
של המטריצה הבאה,
\[A = \pmat{1 & 0 & 0 & 0 & 0 \\ 1 & - 1 & 0 & 0 & -1 \\ 1 & -1 & 0 & 0 & -1 \\ 0 & 0 & 0 & 0 & -1 \\ -1 & 1 & 0 & 0 & 1}\]
כאשר נתון כי
$p_A\prs{x} = x^4\prs{x-1}$.
\end{exercise}

\begin{solution}
לפי הפולינום האופייני,
$1$
ערך עצמי מריבוי אלגברי
$1$.
לכן גם הריבוי הגיאומטרי שלו הוא
$1$
ונקבל כי יש בלוק יחיד עם ערך עצמי
$1$,
ושגודלו
$1$.

כמו כן, אנו יודעים כי
$0$
ערך עצמי מריבוי אלגברי
$4$.
לכן יהיו בלוקי ז'ורדן עם ערך עצמי
$0$
שסכום הגדלים שלהם הוא
$4$.
ניתן לראות כי
$r\prs{A} = 3$,
ולכן יש
$5 - r\prs{A} = 2$
וקטורים עצמיים עם ערך עצמי
$0$.
אז, $J\prs{A}$
היא אחת מהמטריצות הבאות.
\begin{align*}
J_1 &= \pmat{1 & 0 & 0 & 0 & 0 \\ 0 & 0 & 0 & 0 & 0 \\ 0 & 0 & 0 & 1 & 0 \\ 0 & 0 & 0 & 0 & 1 \\ 0 & 0 & 0 & 0 & 0} = \mrm{diag}\prs{J_1\prs{1}, J_2\prs{0}, J_2\prs{0}} \\
J_2 &= \pmat{1 & 0 & 0 & 0 & 0 \\ 0 & 0 & 1 & 0 & 0 \\ 0 & 0 & 0 & 0 & 0 \\ 0 & 0 & 0 & 0 & 1 \\ 0 & 0 & 0 & 0 & 0} = \mrm{diag}\prs{J_1\prs{1}, J_1\prs{0}, J_3\prs{0}}
\end{align*}

כדי למצוא איזו מהמטריצות היא צורת ז'ורדן של
$A$
ניעזר בנוסחא לחישוב מספר הבלוקים מגודל נתון.
מספר הבלוקים מגודל לפחות
$2$
הוא
\[b_2 = \dim \ker\prs{{L_A}^2} - \dim \ker\prs{{L_A}^1}\]
כאשר ניתן לראות כי
\[\text{.} \dim \ker\prs{\prs{L_A}^1} = 2\]
מתקיים
\[A^2 = \pmat{1 & 0 & 0 & 0 & 0 \\ 1 & 0 & 0 & 0 & 0 \\ 1 & 0 & 0 & 0 & 0 \\ 1 & -1 & 0 & 0 & -1 \\ -1 & 0 & 0 & 0 & 0}\]
ולכן
\[\text{.} \dim \ker\prs{{L_A}^2} = 3\]
אז
$b_2 = 3 - 2 = 1$,
כלומר יש בלוק אחד עם ערך עצמי
$0$
ומגודל לפחות
$2$.
זה לא המקרה עבור
$J_2$
ולכן נקבל כי צורת ז'ורדן של
$A$
היא
$J_1$.
\end{solution}

\begin{exercise}
\begin{enumerate}
\item האם כל המטריצות
$A \in M_5\prs{\mbb{C}}$
המקיימות
$A^4 \neq 0$
וגם
$A^5 = 0$
הן דומות?

\item האם כל המטריצות
$A \in M_5\prs{\mbb{C}}$
המקיימות
$A^3 \neq 0$
וגם
$A^4 = 0$
הן דומות?

\item האם כל המטריצות
$A \in M_5\prs{\mbb{C}}$
המקיימות
$A^2 \neq 0$
וגם
$A^3 = 0$
הן דומות?
\end{enumerate}
\end{exercise}

\begin{solution}
\begin{enumerate}
\item התנאי
$A^4 \neq 0$
וגם
$A^5 = 0$
אומר כי
\[\text{,}\dim \ker\prs{L_A^5} - \dim \ker\prs{L_A^4} \geq 1\]
כלומר יש בלוק מגודל לפחות
$5$.
לכן
$A$
דומה ל־%
$J_5\prs{0}$,
ולכן התשובה היא כן.
\item התנאי
$A^3 \neq 0$
וגם
$A^4 = 0$
אומר כי
\[\text{,} \dim \ker\prs{L_A^4} - \dim \ker\prs{L_A^3} \geq 1\]
ולכן יש בלוק מגודל לפחות
$4$.
אבל,
$A^4 = 0$
ולכן לא יתכן שיש בלוק ז'ורדן מגודל
$5$.
לכן כל
$A$
כזאת דומה ל-%
$\mrm{diag}\prs{J_4\prs{0}, J_1\prs{0}}$,
והתשובה היא כן.
\item התנאים
$A^2 \neq 0, A^3 = 0$
מראים כי
\[\text{,} \dim \ker\prs{L_A^3} - \dim \ker\prs{L_A^2} \geq 1\]
ולכן יש בלוק מגודל לפחות
$3$.
אבל,
\begin{align*}
\mrm{diag}\prs{J_3\prs{0}, J_2\prs{0}} \\
\mrm{diag}\prs{J_3\prs{0}, J_1\prs{0}, J_1\prs{0}}
\end{align*}
שתיהן עומדות בתנאים, ואינן דומות.
\end{enumerate}
\end{solution}

\section{בסיס ז'ורדן}

\subsection{מציאת בסיס ז'ורדן}

נציג שתי דרכים למציאת בסיס ז'ורדן. כל אחת שמה דגש אחר על מבנה הבסיס, ולכן חשוב להבין את שתיהן.
האלכוריתם מבוסס על ההנחה שקיימת צורת ז'ורדן, ועל האופן בו צריכים להיראות הוקטורים בבסיס ז'ורדן.

\emph{רעיון:}

ננסה להבין בסיס למרחב ציקלי
$W$.
אם
$B = \prs{v_1, \ldots, v_k}$
וגם
\[\brs{\rest{T}{W}}_B = J_k\prs{\lambda}\]
נקבל כי בהכרח
\[\prs{\rest{T}{W} - \lambda \id_V} v_i = v_{i-1}\]
לכל
$i > 1$
וגם
\[\text{.} \prs{\rest{T}{W} - \lambda \id_V} v_1 = 0\]

מפני שמטריצת ז'ורדן היא מטריצה אלכסונית בלוקים של בלוקי ז'ורדן, כאשר כל בלוק מייצג צמצום לתת־מרחב ציקלי, נקבל כי בסיס ז'ורדן מורכב משרשראות כאלו של וקטורים.
נציג זאת בדוגמא.

\begin{example}
תהי
\[\text{.} J \ceq \mrm{diag}\prs{J_4\prs{1}, J_2\prs{1}, J_2\prs{1}, J_3\prs{0}, J_1\prs{0}}\]
נסתכל על כל ערך עצמי בנפרד.
עבור הערך העצמי
$1$
נשים לב מתקיים
\begin{align*}
\prs{J-I} e_1 &= \prs{J-I} e_5 = \prs{J-I} e_7 = 0 \\
\text{.} \forall i \in [8] \setminus \set{1,5,7}\colon \prs{J-I} e_i &= e_{i-1}
\end{align*}
אפשר לתאר זאת בדיאגרמת יאנג באופן הבא, כאשר נחשוב על כל ריבוע כוקטור שכשכופלים אותו ב־%
$\prs{J-I}$
(משמאל)
מקבלים את הוקטור שמשמאלו.
\begin{center}
\begin{english}
\ydiagram{3,2,2}
\end{english}
\end{center}

עבור הערך העצמי
$0$
נקבל באופן דומה את הדיאגרמה הבאה.
\begin{center}
\begin{english}
\ydiagram{3,1}
\end{english}
\end{center}
\end{example}

בכל אחת מהדרכים הבאות, נמצא קודם כל את הערכים העצמיים של
$T \in \endo_{\mbb{F}}\prs{V}$,
ולאחר מכן נתייחס לכל ערך עצמי
$\lambda \in \mbb{F}$
בנפרד.

\subsubsection{דרך 1: השלמה של בסיס - משמאל לימין}

\begin{enumerate}
\item נמצא את המרחב העצמי
$V_\lambda$
עבור
$\lambda$.
\[V_\lambda = \ker\prs{T-\lambda \id_V}\]
נמצא לו בסיס
$B_1 \ceq \prs{u_1, \ldots, u_m}$.

\item לכל
$i \in [m]$
נפתור את מערכת המשוואות
$\prs{T-\lambda \id_V} v_i = u_i$.
הוקטור
$v_i$
יהיה מימין לוקטור
$u_i$
בשרשרת המתאימה.

\begin{remark}
שימו
\textenglish{\heart}
שיתכן שאין וקטור
$v_i$
כזה.

הדרך הזאת לא תמיד עובדת. כי צריכים לקחת בחירה מתאימה של בסיס כדי שיהיו
$v_i$
כאלו.
\end{remark}

\item נחזור על השלב הקודם עם וקטורים
$w_i$
עבורם
$\prs{T-\lambda \id_V} w_i = v_i$
ונמשיך כך עד שלאף אחת מהמערכות לא יהיה פתרון.

אם הגענו לאוסף שרשראות מאורך כולל
$n$,
הן מרכיבות בסיס ז'ורדן.
\end{enumerate}

\begin{remark}
החיסרון בדרך הזאת היא שהיא לא תמיד עובדת. צריך לבחור בסיס מתאים ל־%
$\ker\prs{T-\lambda \id_V}$
כדי שהיא תעבוד. קיימת צורת ז'ורדן עבור
$T$
ולכן קיים בסיס כזה, אבל לא תמיד נמצא אותו.
\end{remark}

\begin{example}
נסתכל על
$A = \mrm{diag}\prs{J_3\prs{0}, J_1\prs{0}}$.
נקבל
$\ker L_A = \spn\set{e_1, e_4}$.
אז
$B \ceq \prs{e_1 + e_4, e_4}$
בסיס לגרעין, אבל אין פתרון ל־%
$Av = e_4$
או ל־%
$Av = e_1 + e_4$.

אם היינו לוקחים את
$e_1, e_4$
כבסיס ל־%
$\ker\prs{L_A}$
היה אפשר להשלים אותו לבסיס ז'ורדן
$E = \prs{e_1, \ldots, e_4}$.
\end{example}

\begin{remark}
סדר הכתיבה של הבסיס חשוב.
\end{remark}

\subsubsection{דרך 2: בניית השרשראות מימין לשמאל}

הדרך השנייה שנציג יסודית יותר וכוללת חישובים של כל הוקטורים העצמיים המוכללים. היא מתבססת על אופן הוכחת משפט ז'ורדן כפי שמופיעה בהרצאה.

נמצא וקטורים
$v$
עבורם
\[\prs{\prs{T - \lambda \id_V}^{k-1} v, \ldots, \prs{T-\lambda \id_V} v, v}\]
שרשראות ז'ורדן מקסימליות, עבור ערכים מתאימים של
$k$,
כאשר נחפש את השרשראות מהארוכה לקצרה.

\begin{enumerate}
\item נמצא את
$k$%
- אורך השרשרת המקסימלית, בעזרת מימד הגרעין.
זהו הערך המינימלי עבורו
\[\text{.} \dim \ker \prs{\prs{T-\lambda \id_V}^k} = r_a\prs{\lambda}\]

\item נמצא וקטורים
$v_1, \ldots, v_r$
שמשלימים בסיס של
$\ker\prs{\prs{T-\lambda \id_V}^{k-1}}$
לבסיס של
$\ker\prs{\prs{T-\lambda \id_V}^k}$.
לכל
$i \in [r]$
נוסיף את
\[\prs{\prs{T-\lambda \id_V}^{k-1} v_i, \prs{T-\lambda \id_V}^{k-2} v_i, \ldots, \prs{T-\lambda\id_V} v_i, v_i}\]
לבסיס ז'ורדן שאנו מרכיבים.
נסמן
\begin{align*}
\text{.} B_k \ceq \prs{\prs{T-\lambda \id_V}^{k-1} v_1, \ldots, v_1, \ldots, \prs{T-\lambda \id_V}^{k-1} v_r, \ldots, v_r}
\end{align*}

\begin{remark}
שימו
\textenglish{\heart}
שיש חשיבות לסדר הוקטורים בבסיס.
\end{remark}

\item נמצא את אורך השרשרת השנייה הכי ארוכה, באמצעות הסתכלות על מימדי הגרעינים.
נחפש את ה-%
$k$
הבא (הקטן יותר) עבורו יש שרשרת ז'ורדן מגודל
$k$,
כלומר זה שעבורו מתקיים
\[\text{.} \dim \ker\prs{\prs{T-\lambda \id_V}^k} - \dim \ker\prs{\prs{T - \lambda \id_V}^{k - 1}} > \dim \ker\prs{\prs{T-\lambda \id_V}^{k+1}} - \dim \ker\prs{\prs{T - \lambda \id_V}^{k}}\]

\item נשלים בסיס של
$\ker\prs{\prs{T-\lambda \id_V}^{k-1}}$
לבסיס של
$\ker\prs{\prs{T-\lambda \id_V}^k}$
שנסמנו
$\tilde{B}_k$
וניקח תת־קבוצה
$\set{v_1, \ldots, v_r} \subseteq \tilde{B}_k$
מקסימלית כך ש־%
\[\set{v_1 \ldots, v_r} \cup B_{k+1}\]
בלתי־תלויה לינארית,
כאשר
$B_{k+1}$
אוסף הוקטורים מהשלב הקודם.

לכל
$i \in [r]$
נוסיף את השרשרת
\[\prs{\prs{T-\lambda \id_V}^{k-1} v_i, \prs{T-\lambda \id_V}^{k-2} v_i, \ldots, \prs{T-\lambda\id_V} v_i, v_i}\]
לבסיס ז'ורדן. נסמן
\[\text{.} B_k \ceq B_{k+1} * \prs{\prs{T-\lambda \id_V}^{k-1} v_1, \ldots, v_1, \ldots, \prs{T-\lambda \id_V}^{k-1} v_r, \ldots, v_r}\]

\item נחזור על שני השלבים הקודמים עד שנמצא
$r_a\prs{\lambda}$
וקטורים עצמיים מוכללים בלתי־תלויים
$B_0 \ceq B_0\prs{\lambda}$
לכל ערך עצמי
$\lambda$.
אז
\[B \ceq B_0\prs{\lambda_1} * \ldots B_0\prs{\lambda_m}\]
בסיס ז'ורדן כאשר
$\prs{\lambda_i}_{i \in [m]}$
הערכים העצמיים השונים של
$T$.
\end{enumerate}

\begin{remark}
בהרבה מהמקרים, נחפש צורת ז'ורדן עבור העתקות ומטריצות יחסית פשוטות. במקרה זה יכול ללעזור לנו למצוא קודם כל את צורת ז'ורדן בעזרת גדלי הבלוקים, ורק אז את בסיס ז'ורדן.
\end{remark}

\subsection{תרגילים}

\begin{exercise}
מצאו צורת ובסיס ז'ורדן עבור
\begin{align*}
D \colon \mbb{C}_n\brs{x} &\to \mbb{C}_n\brs{x} \\
\text{.} \hphantom{lalalala} p &\mapsto p'
\end{align*}
\end{exercise}

\begin{solution}
נשים לב כי
\[\text{.} \ker\prs{D^i} = \spn\prs{1, x, \ldots, x^{i-1}} = \mbb{C}_{i-1}\brs{x}\]
לכן נסתכל על
$k = n+1$.
נרצה להשלים בסיס של
$\ker\prs{D^{n}}$
לבסיס של
$\ker\prs{D^{n+1}}$.
מתקיים
\[\ker\prs{D^n} = \spn\set{1, x, x^2, \ldots, x^{n-1}}\]
לכן נשלים את הבסיס
$\prs{1, \ldots, x^{n-1}}$
לבסיס
$\prs{1, \ldots, x^{n-1}, x^n}$.
נקבל בסיס ז'ורדן
\[\text{.} \prs{D^n\prs{x^n}, D^{n-1}\prs{x^n}, \ldots, D\prs{x^n}, x^n} = \prs{n!, n! x, \frac{n!}{2!} x^2, \ldots, frac{n!}{\prs{n-1}!} x^{n-1}, x^n}\]
\end{solution}

\begin{exercise}
נתונה המטריצה הנילפוטנטית
\[\text{.} A = \pmat{21 & -7 & 8 \\ 60 & -20 & 23 \\ -3 & 1 & -1}\]
מצאו צורת ובסיס ז'ורדן עבור
$A$.
\end{exercise}

\begin{solution}
חישוב ישיר מראה כי
$A^2 \neq 0$.
$A$
נילפוטנטית ולכן
$A^3 = 0$
ונקבל
$\dim \ker \prs{L_A^3} = 3 = r_a\prs{0}$.
אז אורך השרשרת המקסימלית הוא
$3$
ונקבל כי
$J\prs{A} = J_3\prs{0}$.
ניתן לראות שהעמודה הראשונה והשנייה של
$A$
תלויות לינארית, ושהן בלתי תלויות בשלישית, לכן
$r\prs{A} = 2$.
אז הריבוי הגיאומטרי של
$0$
הוא
$1$
(יכולנו לדעת את זה גם לפי מספר הבלוקים) ונקבל כי
\[\text{.} \ker\prs{L_A} = \spn\set{e_1 + 3 e_2}\]
כעת נחשב את
$\ker\prs{L_A^2}$.
מתקיים
\[A^2 = \pmat{-3 & 1 & -1 \\ -9 & 3 & - 3 \\ 0 & 0 & 0}\]
לכן
$\ker\prs{L_A^2}$,
ששווה למרחב הפתרונות של המערכת ההומוגונית, הוא
\[\text{.} \ker\prs{A^2} = \spn\set{e_1 + 3 e_2, e_2 + e_3}\]

\begin{remark}
קיבלנו בינתיים גרעין דו־מימדי עבור
$L_A^2$,
מה שמסתדר עם צורת ז'ורדן של
$A$
שגילינו.
\end{remark}

כעת, נשלים את הבסיס של
$\ker\prs{L_A^2}$
לבסיס של
$\ker\prs{L_A^3}$,
למשל על ידי הוספת
$e_1$:
\[\text{.} \ker\prs{L_A^3} = \spn\set{e_1 + 3 e_2, e_2 + e_3, e_1}\]
אז שרשרת ז'ורדן תהיה
\[\text{.} \prs{A^2 e_1, A e_1, e_1} = \pmat{\pmat{-3 \\ -9 \\ 0}, \pmat{21 \\ 60 \\ -3}, e_1}\]

\begin{remark}
\emph{אין קשר}
בין בסיס ז'ורדן לבין הבסיסים שמצאנו עבור הגרעינים השונים תוך כדי מציאת הוקטורים בראש השרשרת.
\end{remark}
\end{solution}

\end{document}