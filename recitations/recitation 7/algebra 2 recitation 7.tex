\documentclass[a4paper,10pt,oneside,openany]{article}

\usepackage[lang=hebrew]{maths}
\usepackage{hebrewdoc}
\usepackage{stylish}
\usepackage{lipsum}
\let\bs\blacksquare
\usepackage{ytableau}

\usepackage{graphicx, txfonts}

\newcommand{\heart}{\ensuremath\varheartsuit}

\title{
אלגברה ב' (104168) \textenglish{---} אביב 2020-2021
\\
תרגול 6 \textenglish{---} צורת ובסיס ז'ורדן
}
\author{אלעד צורני}
\date{\today}

\begin{document}
\maketitle

\section{בסיס ז'ורדן}

\subsection{חזרה}

\begin{algorithm}
כדי למצוא צורת ז'ורדן נסתכל על כל ערך עצמי
$\lambda$
בנפרד.
נסמן
$N \ceq T - \lambda \id_V$,
נחפש וקטורים בלתי־תלויים ב־%
$\ker\prs{N^k} \setminus \ker\prs{N^{k-1}}$
שיפתחו שרשראות ז'ורדן שירכיבו קבוצה בלתי־תלויה
$B_k$.
נעשה זאת עבור ערכי
$k$
הולכים וקטנים ונקבל בסוף בסיס ז'ורדן
$B_0$.
\end{algorithm}

\subsection{תרגילים}

\begin{exercise}
ידוע כי כל הערכים העצמיים של
\[A \ceq \pmat{1 & 1 & 0 & 0 & 0 & 0 \\
-1 & 3 & 0 & 0 & 0 & 0 \\
-1 & 0 & 2 & 1 & 0 & 0 \\
-1 & 0 & -1 & 3 & 1 & 0 \\
-1 & 0 & -1 & 0 & 4 & 0 \\
-1 & 0 & -1 & 0 & 2 & 2} \in M_6\prs{\mbb{C}}\]
רציונליים.
מצאו צורת ובסיס ז'ורדן עבור
$A$.
\end{exercise}

\begin{solution}
נסמן
$V = \mbb{C}^6$.

חישוב ישיר נותן
\[\text{.} p_A\prs{x} = \det\prs{xI - A} = x^6 - 15x^5 + 93x^4 - 305x^3 + 558x^2 - 540x + 216\]
ממשפט ניחוש השורש הרציונלי אפשר למצוא את השורשים ולקבל
\[\text{.} p_A\prs{x} = \prs{x-2}^3 \prs{x-3}^3\]

נסתכל על הערכים העצמיים
$2,3$
בנפרד.

\begin{description}
\item[$\lambda = 3$:]
מתקיים
\[\prs{A-3I} = \pmat{-2 & 1 & 0 & 0 & 0 & 0 \\
-1 & - & 0 & 0 & 0 & 0 \\
-1 & 0 & -1 & 1 & 0 & 0 \\
-1 & 0 & -1 & 0 & 1 & 0 \\
-1 & 0 & -1 & 0 & 1 & 0 \\
-1 & 0 & -1 & 0 & 2 & -1}\]
ומרחב הפתרונות של המערכת ההומוגנית הוא
\[\text{.} \ker\prs{L_A - 3\id_V} = \spn\set{\pmat{0\\0\\1\\1\\1\\1}}\]
חישוב ישיר נותן כי
\[\ker\prs{\prs{L_A - 3\id_V}^2} = \spn\set{\pmat{0\\0\\1\\1\\1\\1}, e_3}\]
וגם
\[\text{.} \ker\prs{\prs{L_A - 3\id_V}^3} = \spn\set{\pmat{0\\0\\1\\1\\1\\1}, e_3, e_4}\]
אז,
$e_4$
פותח שרשרת ז'ורדן
\[\text{.} \pmat{\prs{A-3I}^2 e_4, \prs{A-3I}e_4, e_4} = \pmat{\pmat{0\\0\\-1\\-1\\-1\\-1}, e_3, e_4}\]

\item[$\lambda = 2$:]
מתקיים
\[\ker\prs{L_A - 2\id_V} = \spn\set{e_6, \pmat{1\\1\\1\\1\\1\\0}}\]
ולכן יש ל־%
$2$
ריבוי גיאומטרי
$2$.
אז יש שני בלוקי ז'ורדן עבור הערך העצמי
$2$,
ולכן השרשרת המקסימלית מגודל
$2$.
אפשר לראות כי סכום העמודות של
\[A - 2I = \pmat{-1 & 1 & 0 & 0 & 0 & 0 \\
-1 & 1 & 0 & 0 & 0 & 0 \\
-1 & 0 & 0 & 1 & 0 & 0 \\
-1 & 0 & -1 & 1 & 1 & 0 \\
-1 & 0 & -1 & 0 & 2 & 0 \\
-1 & 0 & -1 & 0 & 2 & 0}\]
שווה
$0$,
ולכן
$e_1 \in \ker\prs{\prs{L_A - \id_V}^2}$
ונקבל
\[\text{.} \ker\prs{\prs{L_A - \id_V}^2} = \spn\set{e_6, \pmat{1\\1\\1\\1\\1\\0}, e_1}\]
אז
$e_1$
מתחיל שרשרת ז'ורדן
\[\text{.} \prs{\prs{A-2I} e_1, e_1} = \prs{\pmat{-1\\-1\\-1\\-1\\-1\\-1}, e_1}\]

חסרה שרשרת ז'ורדן מאורך
$1$
עבור
$\lambda = 2$.
היא וקטור עצמי של
$2$
שאינו תלוי ב־%
$\prs{\prs{A-2I} e_1, e_1}$.
למשל,
$e_6$
הוא כזה וקטור עצמי.

\item[סיכום:]
נסדר את השרשראות השונות בבסיס ונקבל בסיס ז'ורדן
\[
B = \pmat{\pmat{0\\0\\-1\\-1\\-1\\-1}, e_3, e_4, \pmat{-1\\-1\\-1\\-1\\-1\\-1}, e_1, e_6}
\]
לפיו
\[\text{.} \brs{L_A}_B = \mrm{diag}\prs{J_3\prs{3}, J_2\prs{2}, J_1\prs{2}}\]
\end{description}

\end{solution}

\begin{exercise}
יהי
$\mbb{F} = \mbb{C}$.
\begin{enumerate}
\item
חשבו את
$J_m\prs{0}^k$
לכל
$m,k \in \mbb{N}_+$.

\item
נשים לב שמתקיים
$J_m\prs{\lambda} = J_m\prs{0} + \lambda I_m$
וכי
$J_m\prs{0}, I_m$
מטריצות מתחלפות. השתמשו בכך כדי לחשב את
$J_m\prs{\lambda}^k$
לכל
$m,k \in \mbb{N}_+$
ולכל
$\lambda \in \mbb{C}$.

\item תהי
\[\text{.} A \ceq \pmat{2 & 4 & 0 \\ -1 & -2 & 0 \\ 8 & 7 & 9} \in M_3\prs{\mbb{C}}\]
חשבו את
$A^{2021}$.
\end{enumerate}
\end{exercise}

\begin{solution}
\begin{enumerate}
\item%1
נשים לב כי
$J_{m}\prs{0}\prs{e_i} = e_{i-1}$
לכל
$i > 1$
וכי
$J_m\prs{0} e_1 = 0$.
אז
\begin{align*}
J_m\prs{0}^k e_i = e_{i-k}
\end{align*}
לכל
$i > k$
וגם
$J_m\prs{0}^k e_j = 0$
לכל
$i \leq k$.
נקבל כי
$J_m\prs{0}^k$
מטריצה עם
$1$
באלכסון ה־%
$k$
מעל האלכסון הראשי, ו־%
$0$
בכל שאר הכניסות.
\item%2
כיוון ש־%
$J_m\prs{0}, \lambda I_m$
מתחלפות, אפשר לחשב את
\[J_m\prs{\lambda}^k = \prs{\lambda I_m + J_m\prs{0}}^k\]
בעזרת הבינום.
מתקיים
\begin{align*}
\text{.} J_m\prs{\lambda}^k &= \sum_{i = 0}^{k} \binom{k}{i} J_m\prs{0}^i \lambda^{k-i}
\end{align*}
לפי מה שראינו עבור
$J_m\prs{0}^i$,
זאת מטריצה שבה על הלאכסון הראשי כתוב
$\lambda^k$,
על האלכסון מעליו
$k \lambda^{k-1}$
ועל האלכסון ה־%
$i$
מעל האלכסון הראשי
$\binom{k}{i} \lambda^{k-i}$.
\item%3
נסמן
$V = \mbb{C}^3$
ונמצא בסיס ז'ורדן
$B$
עבור
$J_A$.
אז נקבל
$P \in M_3\prs{\mbb{C}}$
עבורה
$J \ceq P A P^{-1}$
מטריצת ז'ורדן, ואז
\begin{align*}
A^{2021} &= \prs{P^{-1}JP}^{2021}
= P^{-1}J^{2021}P
\end{align*}
כאשר מהחישוב הנ"ל נדע לחשב את
$J^{2021}$.

\begin{description}
\item[ערכים עצמיים:]
ניתן לראות כי
$9$
ערך עצמי של
$A$
כי
$A e_3 = 9 e_3$.
נסמן ב־%
$\lambda_1, \lambda_2$
את הערכים העצמיים הנוספים ונקבל
\begin{align*}
\lambda_1 + \lambda_2 + 9 = \tr\prs{A} = 9 \\
9 \lambda_1 \lambda_2 = \det\prs{A} = 9\prs{-4+4} = 0
\end{align*}
לכן
$\lambda_1 = \lambda_2 = 0$.

נקבל כי
$9$
ערך עצמי מריבוי אלגברי
$1$
וכי
$0$
ערך עצמי מריבוי אלגברי
$2$.
בפרט,
$\prs{e_3}$
שרשרת ז'ורדן עבור הערך העצמי
$9$.
\item[שרשרת ז'ורדן עבור
$\lambda = 0$:]
ניתן לראות כי
$r\prs{A} = 2$
ולכן
$\dim \ker \prs{L_A} = 1$.
נשים לב כי
\[2 e_1 - e_2 - e_3 \in \ker\prs{L_A}\]
ולכן
\[\text{.} \ker\prs{L_A} = \spn\prs{2 e_1 - e_2 - e_3}\]
מתקיים
\[A^2 = \pmat{0 & 0 & 0 \\ 0 & 0 & 0 \\ 81 & 81 & 81}\]
לכן נוכל להשלים את
$\prs{2 e_1 - e_2 - e_3}$
לבסיס
\[\prs{2 e_1 - e_2 - e_3, e_1 - e_3}\]
של
$\ker\prs{L_A^2}$.
מתקיים
\[A \prs{e_1 - e_3} = 2e_1 - e_2 - e_3\]
לכן נקבל שרשרת ז'ורדן
\[\text{.} \prs{A\prs{e_1 - e_3}, e_1 - e_3} = \prs{2e_1 - e_2 - e_3, e_1 - e_3}\]

\item[מסקנה:]
קיבלנו
\[B \ceq \prs{2e_1 - e_2 - e_3, e_1 - e_3, e_3}\]
עבורו
\[\brs{L_A}_B = J \ceq \pmat{0 & 1 & 0 \\ 0 & 0 & 0 \\ 0 & 0 & 9}\]
ולכן
\begin{align*}
\text{.} A &= \brs{L_A}_E
= \brs{\id_V}^{B}_E \brs{L_A}_B \brs{\id_V}^E_B
\end{align*}
נסמן
\[P \ceq \brs{\id_V}^B_E = \pmat{2 & 1 & 0 \\ -1 & 0 & 0 \\ -1 & -1 & 1}\]
שעמודותיה הן וקטורי הבסיס
$B$,
ונקבל
\[\text{.} A = P J P^{-1}\]
אז
\begin{align*}
A^{2021} &= P J^{2021} P^{-1}
\\&= \pmat{2 & 1 & 0 \\ -1 & 0 & 0 \\ -1 & -1 & 1} \pmat{0 & 0 & 0 \\ 0 & 0 & 0 \\ 0 & 0 & 9^{2021}} \pmat{0 & -1 & 0 \\ 1 & 2 & 0 \\ 1 & 1 & 1}
\\ \text{.} \hphantom{A^{2021}} &= \pmat{0 & 0 & 0 \\ 0 & 0 & 0 \\ 9^{2021} & 9^{2021} & 9^{2021}}
\end{align*}
\end{description}
\end{enumerate}
\end{solution}

\section{פולינומים מינימליים}

\subsection{חזרה}

\begin{definition}[פולינום מינימלי]
עבור
$T \in \endo_{\mbb{F}}\prs{V}$,
\emph{הפולינום המינימלי של
$T$},
שנסמנו
$m_T \in \mbb{F}\brs{x}$,
הוא הפולינום המתוקן מהדרגה המינימלית עבורו
$m_T\prs{T} = 0$.
\end{definition}

\begin{proposition}
אם
$p \in \mbb{F}\brs{x}$
מקיים
$p\prs{T} = 0$
אז
$m_T \mid p$.
\end{proposition}

\begin{corollary}
הפולינום המינימלי של
$T$
מחלק את הפולינום האופייני שלה.
\end{corollary}

\begin{fact}
הריבוי של ערך עצמי
$\lambda$
בפולינום המינימלי
$m_T$
היא הגודל של בלוק ז'ורדן המקסימלי עבור הערך העצמי
$\lambda$.
\end{fact}

\begin{corollary}
העתקה
$T \in \endo_{\mbb{F}}\prs{V}$
לכסינה אם ורק אם הפולינום המינימלי שלה מתפרק לגורמים לינאריים זרים.
\end{corollary}

\subsection{תרגילים}

\begin{exercise}
יהי
$\mbb{F} = \mbb{C}$
ותהי
$T \in \endo_{\mbb{F}}\prs{V}$.
יהי
$m \in \mbb{N}_+$
עבורו
$T^m = \id_V$.
הראו כי
$T$
לכסינה.
\end{exercise}

\begin{solution}
כדי להראות ש־%
$T$
לכסינה מספיק להראות שכל שורשי
$m_T$
הינם מריבוי
$1$.
מההנחה, מתקיים
$m_T \mid \prs{x^m - 1}$
ולכן די להראות שכל שורשי
$x^m - 1$
הם מריבוי
$1$.
אכן, יש לפולינום זה
$m$
שורשים שונים
\[\text{.} \set{e^{\frac{2 \pi i k}{m}}}{k \in [m]} = \set{\mrm{cis}\prs{\frac{2 \pi k}{m}}}{k \in [m]}\]
\end{solution}

\begin{exercise}
\begin{enumerate}
\item
יהיו
$A,B \in M_6\prs{\mbb{C}}$
המקיימות
\begin{enumerate}[label = (\roman*)]
\item $p_A = p_B$.
\item $m_A = m_B$
וזהו פולינום ממעלה
$5$.
\end{enumerate}
הראו כי
$A \sim B$.

\item
מצאו
$A,B \in M_6\prs{\mbb{C}}$
שאינן דומות וכך שמתקיים
\begin{enumerate}[label = (\roman*)]
\item $p_A = p_B$.
\item $m_A = m_B$
וזהו פולינום ממעלה
$4$.
\end{enumerate}
\end{enumerate}
\end{exercise}

\begin{solution}
\begin{enumerate}
\item נתון
$\deg m_A = 5$,
וזהו סכום גדלי הבלוקים המקסימליים של הערכים העצמיים השונים של
$A$.
לכן יש ערך עצמי
$\lambda$
מריבוי גיאומטרי
$2$
ובלוק מגודל
$1$,
וכל ערך עצמי אחר הוא מריבוי גיאומטרי
$1$.
אז
\[\text{.} A \sim \pmat{\lambda & 0 & \cdots & & 0 \\ 0 & J_m\prs{\lambda_1} & \ddots & & \\ \vdots & \ddots & \ddots & & \vdots \\ & & & \ddots & 0 \\ 0 & & \cdots & 0 & J_{m_r}\prs{\lambda_r}}\]
נתון
$m_A = m_B$
לכן באותו אופן
\[B \sim \pmat{\lambda & 0 & \cdots & & 0 \\ 0 & J_m\prs{\lambda_1} & \ddots & & \\ \vdots & \ddots & \ddots & & \vdots \\ & & & \ddots & 0 \\ 0 & & \cdots & 0 & J_{m_r}\prs{\lambda_r}}\]
ונסיק כי
$A \sim B$.

\item נסתכל על
\begin{align*}
A &\ceq \pmat{J_1\prs{0} & & \\  & J_1\prs{0} & \\ & & J_4\prs{0}} \\
B &\ceq \pmat{J_2\prs{0} & \\ & J_4\prs{0}}
\end{align*}
ונקבל
$m_A = m_B = x^4$
אבל
$A \not\sim B$
כי יש להן צורת ז'ורדן שונה.

\end{enumerate}
\end{solution}

\begin{exercise}
עבור פולינום מתוקן
$p \in \mbb{F}\brs{x}$
נכתוב
\[p\prs{x} = \sum_{i = 0}^n c_i x^i\]
כאשר
$c_n = 1$,
ונגדיר את
\emph{המטריצה המלווה של
$p$}
על ידי
\[\text{.} C\prs{p} \ceq \pmat{0 & 0 & \cdots & 0 & -c_0 \\ 1 & 0 & \cdots & 0 & -c_1 \\ 0 & 1 & \cdots & 0 & -c_2 \\ \vdots & \vdots & \ddots & \vdots & \vdots \\ 0 & 0 & \dots & 1 & -c_{n-1}} \in M_n\prs{\mbb{F}}\]

יהי
$p \in \mbb{F}\brs{x}$
ממעלה
$n$.
מצאו את הפולינום האופייני ואת הפולינום המינימלי של
$C\prs{p}$.
\end{exercise}

\begin{solution}
\begin{itemize}
\item
מתקיים
\begin{align*}
\text{.} p_{C\prs{p}} &= \det\prs{I - xC\prs{p}}
= \det\pmat{x & 0 & \cdots & 0 & c_0 \\ -1 & x & \cdots & 0 & c_1 \\ 0 & -1 & \cdots & 0 & -c_2 \\ \vdots & \vdots & \ddots & \ddots & \vdots \\ 0 & 0 & \dots & -1 & x + c_{n-1}}
\end{align*}
ניעזר בכך שהטרמיננטה אינווריאנטית תחת הוספת כפולה של שורה לשורה אחרת.
נוסיף את השורה האחרונה כפול
$x$
לזאת שלפניה. לאחר מכן נוסיף את השורה ה־%
$n-1$
כפול
$x$
לזאת שלפניה ונמשיך כך עד שנקבל מטריצה
\begin{align*}
\pmat{0 & 0 & \cdots & 0 & y \\ -1 & 0 & \cdots & 0 & * \\ 0 & -1 & \cdots & 0 & * \\ \vdots & \vdots & \ddots & \ddots & \vdots \\ 0 & 0 & \dots & -1 & *}
\end{align*}
כאשר
\[\text{.} y = a_0 + x\prs{a_1 + x\prs{a_2 + x\prs{\ldots \prs{a_{n-2} + x\prs{a_{n-1} + x}} \ldots}}} = \sum_{i=0}^n c_i x^i = f\]
אז
\[\text{.} p_{C\prs{p}} = \prs{-1}^{n-1} \det \pmat{-1 & x & \cdots & 0 \\ 0 & -1 & \cdots & 0 \\ \vdots & \vdots & \ddots & \vdots \\ 0 & 0 & \cdots & - 1} = \prs{-1}^{n-1} \prs{-1}^{n-1} f = f\]
\item
יהי
$g \in \mbb{F}_{n-1}\brs{x}$
ונכתוב
\[\text{.} g\prs{x} = \sum_{i = 0}^{n-1} b_i x^i\]
אז
\begin{align*}
g\prs{C\prs{p}}\prs{e_1} &= \sum_{i = 0}^{n-1} b_i C\prs{p}^i e_1
\\&= \sum_{i=0}^{n-1} b_i e_{1+i}
\end{align*}
וביטוי זה שונה מאפס כאשר
$g \neq 0$
כי אז יש
$b_i \neq 0$
עבור
$i$
כלשהו.

לכן
$\deg\prs{m_{C\prs{p}}} \geq n$
ולכן
$m_{C\prs{p}} = p_{C\prs{p}}$.
\end{itemize}
\end{solution}

\end{document}