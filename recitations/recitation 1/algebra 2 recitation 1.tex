\documentclass[a4paper,10pt,oneside,openany]{article}

\usepackage[lang=hebrew]{maths}
\usepackage{hebrewdoc}
\usepackage{stylish}
\usepackage{lipsum}
\let\bs\blacksquare

\title{
אלגברה ב' (104168) \textenglish{---} אביב 2020-2021
\\
תרגול 1 \textenglish{---} חזרה - מטריצות מייצגות
}
\author{אלעד צורני}
\date{\today}

\begin{document}
\maketitle

\setcounter{section}{1}

\section*{מטריצות מייצגות}

\subsection*{הגדרות וסימונים}

\begin{definition}[בסיס של מרחב וקטורי]
יהי
$V$
מרחב וקטורי מעל שדה
$\mbb{F}$.
\emph{בסיס של
$V$}
הוא קבוצה
\textbf{סדורה}
$B$
כך שכל
$v \in V$
ניתן לכתיבה באופן
\textbf{יחיד}
כצירוף לינארי של (מספר סופי של) איברי
$B$.
\end{definition}

\begin{notation}[מרחב מטריצות]
יהי
$\mbb{F}$
שדה ויהיו
$n,m \in \mbb{N}_+$.
נסמן ב־%
$M_{n,m}\prs{\mbb{F}}$
מטריצות מעל
$\mbb{F}$
עם
$n$
שורות ו־%
$m$
עמודות.
נסמן גם
$M_n\prs{\mbb{F}} \ceq M_{n,n}\prs{\mbb{F}}$.
\end{notation}

\begin{notation}[מרחב העתקות לינאריות]
יהיו
$V,W$
מרחבים וקטוריים מעל שדה
$\mbb{F}$.
נסמן ב־%
$\hom_{\mbb{F}}\prs{V,W}$
את מרחב ההעתקות הלינאריות
$V \to W$.

נסמן לפעמים
\[\endo_{\mbb{F}}\prs{V} \ceq \hom_{\mbb{F}}\prs{V,V}\]
ונקרא לאיברי מרחב זה
\emph{אנדומורפיזמים של
$V$}.
\end{notation}

\begin{notation}
נסמן
$\mbb{F}^n \ceq M_{n,1}\prs{\mbb{F}}$
וגם
$\prs{\mbb{F}^n}^\perp \ceq M_{1,n}\prs{\mbb{F}}$.
\end{notation}

\begin{definition}[מטריצה מייצגת]
יהיו
$V,W$
מרחבים וקטוריים מעל שדה
$\mbb{F}$
עם בסיסים
\begin{align*}
B &= \prs{v_1, \ldots, v_n}, \\
C &= \prs{w_1, \ldots, w_m}
\end{align*}
בהתאמה.
\emph{המטריצה המייצגת של
$T$
לפי הבסיסים
$B,C$}
היא המטריצה
$A \ceq \brs{T}_B^C \in M_{m,n}\prs{\mbb{F}}$
שמוגדרת על ידי
\begin{align*}
\text{.} \forall j \in \brs{n} \colon T v_j = \sum_{i \in [m]} A_{i,j} w_i
\end{align*}
\end{definition}

\begin{notation}
יהי
$V$
מרחב וקטורי סוף־מימדי עם בסיס
$B$.
נסמן
$\brs{T}_B \ceq \brs{T}^B_B$.
\end{notation}

\begin{definition}[וקטור קואורדינטות]
יהי
$V$
מרחב וקטורי מעל שדה
$\mbb{F}$
עם בסיס
$B = \prs{v_1, \ldots, v_n}$.
עבור
$u \in V$
נסמן
$\brs{u}_B \in \mbb{F}^n$
את
\emph{וקטור הקואורדינטות של
$u$
לפי הבסיס
$B$}
שמוגדר על ידי
\begin{align*}
\text{.} u &= \sum_{i \in [n]} \prs{\brs{u}_B}_i v_i
\end{align*}
\end{definition}

\begin{definition}[מטריצת מעבר]
יהיו
$V$
מרחב וקטורי
עם בסיסים
$B,C$.
נסמן
\[P^B_C \ceq \brs{\id_V}^B_C\]
ונקרא לה
\emph{מטריצת המעבר מ־%
$B$
ל־%
$C$}.
\end{definition}

\begin{definition}[העתקות המתאימות למטריצה]
יהי
$\mbb{F}$
שדה, יהי
$n \in \mbb{N}_+$
ותהי
$A \in M_n\prs{\mbb{F}}$.
נגדיר
\begin{align*}
L_A \colon \mbb{F}^n &\to \mbb{F}^n \\
v &\mapsto Av
\end{align*}
וגם
\begin{align*}
R_A \colon \prs{\mbb{F}^n}^\perp &\to \prs{\mbb{F}^n}^\perp \\
\text{.} v &\mapsto vA
\end{align*}
\end{definition}

\begin{definition}
יהי
$V$
מרחב וקטורי
$n$%
־מימדי מעל שדה
$\mbb{F}$.
נגדיר
\begin{align*}
\rho_B \colon V &\to \mbb{F}^n \\
\text{.} v &\mapsto \brs{v}_B
\end{align*}
\end{definition}

\begin{remark}
$\rho_B$
שולחת בסיס לבסיס, ולכן הינה איזומורפיזם.
\end{remark}

\begin{exercisex}
תהי
$T \colon V \to W$
העתקה בין מרחבים וקטוריים ויהיו
$B,C$
בסיסים עבור
$V,W$
בהתאמה.
לכל
$v \in V$
מתקיים
\[ \text{.} \brs{T}^B_C \brs{v}_B = \brs{Tv}_C\]
להיפך, אם
$A$
מטריצה המקיימת
\[  A \brs{v}_B = \brs{Tv}_C\]
לכל
$v \in V$,
אז
$A = \brs{T}_C^B$.
\end{exercisex}

\begin{corollary} \label{corollary:representing_composition}
יהיו
$U,V,W$
מרחבים וקטוריים מעל שדה
$\mbb{F}$
ותהיינה
\begin{align*}
T \colon U &\to V \\
S \colon V &\to W
\end{align*}
העתקות לינאריות.
יהיו
$B,C,D$
בסיסים של
$U,V,W$
בהתאמה.
אז
\begin{align*}
\text{.} \brs{S \circ T}^B_D = \brs{S}^C_D \brs{T}^B_C
\end{align*}
\end{corollary}

\begin{proof}
לכל
$v \in V$
מתקיים
\begin{align*}
\brs{S}^C_D \brs{T}^B_C \brs{v}_B &= \brs{S}^C_D \brs{Tv}_C
\\&= \brs{S \circ T v}_D
\end{align*}
לכן
\[\text{,}\brs{S}^C_D \brs{T}^B_C = \brs{S \circ T}^B_D\]
כנדרש.
\end{proof}

\subsection*{תרגילים}

\begin{exercise}
תהי
$A \in M_{n}\prs{\mbb{F}}$
הפיכה.
\begin{enumerate}
\item הראו שלכל בסיס
$B$
של
$\mbb{F}^n$
קייo בסיס
$C$
של
$\mbb{F}^n$
כך שמתקיים
\begin{align*}
\text{.}A &= \brs{\id_V}^B_C
\end{align*}
\item יהי
$V$
מרחב וקטורי
$n$%
־מימדי מעל
$\mbb{F}$
ויהי
$B$
בסיס של
$V$.
מצאו בסיס
$B'$
של
$V$
כך ש־%
$A = P^B_{B'}$.
\item יהי
$T \colon V \to W$
איזומורפיזם בין מרחבים וקטוריים ממימד
$n \in \mbb{N}_+$
מעל
$\mbb{F}$.
יהי
$B$
בסיס של
$V$.
מצאו בסיס
$C$
של
$W$
כך שמתקיים
\[\text{.} A = \brs{T}^B_{C}\]
\end{enumerate}
\end{exercise}

\begin{solution}
\begin{enumerate}
\item%1
$A$
מטריצה הפיכה, לכן
$L_A$
איזומורפיזם, ולכן שולחת בסיס לבסיס.
ידוע כי
$\rho_B$
איזומורפיזם ולכן גם היא שולחת בסיס לבסיס.
מתקיים
\begin{align*}
\brs{\id_V}^B_C &= \brs{\rho_B^{-1} \circ \rho_B}^B_C
\\&= \brs{\rho_B^{-1}}^E_C \brs{\rho_B}^B_E
\\&= \brs{\rho_B^{-1}}^E_C I_n
\\&= \brs{\rho_B^{-1}}^E_C
\end{align*}
לכן מספיק למצוא
$C$
עבורו
\[\text{.} \brs{\rho_B^{-1}}^E_C = A\]
יהי
\[\text{.} C \ceq \prs{\rho_B^{-1} \circ L_A^{-1}\prs{e_1}, \ldots, \rho_B^{-1} \circ L_A^{-1}\prs{e_n}}\]
לכל
$i \in [n]$
נקבל
\begin{align*}
\brs{\rho_B^{-1}}^E_C A^{-1} e_i &= \brs{\rho_B^{-1}}^E_C \brs{A^{-1} e_i}_E
\\&= \brs{\rho_B^{-1} A^{-1} e_i}_C
\\\text{.} \hphantom{\brs{\rho_B^{-1}}^E_C A^{-1} e_i} &= e_i
\end{align*}
לכן
$\brs{\rho_B^{-1}}^E_C A^{-1} e_i = e_i$
לכל
$i \in [n]$,
לכן
\[\text{.} \brs{\rho_B^{-1}}^E_C = \prs{A^{-1}}^{-1} = A\]

\textbf{מוטיבציה:}

מהסתכלות על המקרה
$B = E$
רואים שבמקרה זה עובדת הבחירה
\[\text{.} C \ceq \prs{L_A^{-1}\prs{e_1}, \ldots, L_A^{-1}\prs{e_n}}\]
באופן כללי, הדבר לא עובד. אבל כיוון שיש איזומורפיזם של
$\mbb{F}^n$
ששולח את הבסיס
$B$
לבסיס
$E$,
נוכל להיעזר בו כדי למצוא בסיס
$C$
מתאים, בעזרת הצמדה.
$C$
שבחרנו הוא בעצם
\[\text{.} C = \prs{\rho_B^{-1} \circ L_A^{-1} \circ \rho_B\prs{v_1}, \ldots, \rho_B^{-1} L_A^{-1} \rho_B\prs{v_n}}\]

בדרך כלל כאשר אנו יודעים משהו עבור אוביקט, ורוצים להראות אותו עבור אוביקט איזומורפי, נוכל להיעזר בהצמדה באופן דומה.

\item%2
יהי
$E$
הבסיס הסטנדרטי של
$\mbb{F}^n$
ויהי
$E' = \prs{u_1, \ldots, u_n}$
בסיס עבורו
$\brs{\id_{\mbb{F}^n}}^E_{E'} = A$,
שקיים לפי הסעיף הקודם.
יהי
\[B' = \prs{\rho_B^{-1} \prs{u_1}, \ldots, \rho_B^{-1} \prs{u_n}}\]
בסיס של
$V$
כי
$\rho_B^{-1}$
איזומופריזם.
\[
\begin{tikzcd}[row sep=large, column sep = large]
V \arrow[d,swap,"\rho_B"] \arrow[r, "\rho_B^{-1} \circ L_A \circ \rho_B"] & V \\
\mbb{F}^n \arrow[r, "L_A"] & \mbb{F}^n \arrow[u, swap, "\rho_B^{-1}"]
\end{tikzcd}
\]
אז מתקיים
\begin{align*}
\brs{\id_V}^B_{B'} &\underset{\hphantom{\eqref{corollary:representing_composition}}}{=} \brs{\rho_B^{-1} \circ \id_{\mbb{F}^n} \circ \rho_B}^B_{B'}
\\&\underset{\eqref{corollary:representing_composition}}{=} \brs{\rho_B^{-1}}^{E'}_{B'} \brs{\id_{\mbb{F}^n}}^E_{E'} \brs{\rho_B}^B_E
\\&\underset{\hphantom{\eqref{corollary:representing_composition}}}{=} \brs{\id}^E_{E'}
\\&\underset{\hphantom{\eqref{corollary:representing_composition}}}{=} P^E_{E'}
\\ \text{.} \hphantom{\brs{\id_V}^B_{B'} } &\underset{\hphantom{\eqref{corollary:representing_composition}}}{=} A
\end{align*}
\item%3
יהי
\[B' = \prs{u_1, \ldots, u_n}\]
הבסיס של
$V$
מהסעיף הקודם.
יהי
\[C = \prs{T\prs{u_1}, \ldots, T\prs{u_n}}\]
בסיס של
$W$
כי
$T$
איזומורפיזם.
מתקיים
\begin{align*}
\brs{T}^B_C &\underset{\hphantom{\eqref{corollary:representing_composition}}}{=} \brs{T \circ \id_V}^B_C
\\&\underset{\eqref{corollary:representing_composition}}{=} \brs{T}^{B'}_C \brs{\id_V}^B_{B'}
\\&\underset{\hphantom{\eqref{corollary:representing_composition}}}{=} I_n A
\\ \text{.} \hphantom{\brs{T}^B_C} &\underset{\hphantom{\eqref{corollary:representing_composition}}}{=} A
\end{align*}
\end{enumerate}
\end{solution}

\begin{exercise}
תהי
$D \in \hom\prs{\mbb{R}_3\brs{x}, \mbb{R}_2\brs{x}}$
העתקת הגזירה, המוגדרת על ידי
$Dp = p'$.
מצאו בסיסים של
$\mbb{R}_3\brs{x}, \mbb{R}_2\brs{x}$
לפיהם המטריצה המייצגת של
$T$
היא
\[\text{.} \pmat{1 & 0 & 0 & 0 \\ 0 & 1 & 0 & 0 \\ 0 & 0 & 1 & 0}\]
\end{exercise}

\begin{solution}
יהי
$U \ceq \mbb{R} \leq \mbb{R}_3\brs{x}$
התת־מרחב של הפולינומים הקבועים.
מתקיים
$U = \ker D$.
יהי
\[W = \spn\set{x, x^2, x^3} \leq \mbb{R}_3\brs{x}\]
כך שמתקיים
$U \oplus W = \mbb{R}_3\brs{x}$.
נגדיר
\begin{align*}
T \ceq \rest{D}{W} \colon W &\to \mbb{R}_3\brs{x}
\end{align*}
וזה איזומורפיזם כי
$\dim_{\mbb{R}} W = \dim_{\mbb{R}} \mbb{R}_2\brs{x} = 3$
וכי
$\ker T = \set{0}$.

תהי
$A = I_3 \in M_3\prs{\mbb{R}}$.
מהתרגיל הקודם, יש בסיסים
$\tilde{B}, C$
של
$U, \mbb{R}_2\brs{x}$
בהתאמה כך ש־%
$\brs{T}^{\tilde{B}}_C = A$.
נכתוב
\[\tilde{B} = \prs{v_1, v_2, v_3}\]
ואז
\[B \ceq \prs{v_1, v_2, v_3, 1}\]
נותן את המבוקש.
\end{solution}

\begin{exercise}
יהי
$V$
מרחב וקטורי סוף מימדי ויהי
$v \in V$.
הראו שקיים בסיס
$B$
של
$V$
עבורו
$\brs{v}_B = \pmat{1 \\ \vdots \\ 1}$.
\end{exercise}

\begin{solution}
נשלים את
$\prs{v}$
לבסיס
\[B_0 = \prs{v_1, \ldots, v_n}\]
של
$V$,
עם
$v_1 = v$.
תהי
\begin{align*}
\text{.} A \ceq \pmat{1 & & & & \\
1 & 1 & & 0 & \\
\vdots & & \ddots & & \\
1 & & & 1 & \\
1 & 0 & \cdots & 0 & 1} \in M_n\prs{\mbb{F}}
\end{align*}
$A$
הפיכה, ולכן מתרגיל קודם קיים בסיס
$B$
של
$V$
עבורו
$\brs{\id_V}^{B_0}_B = A$.
נקבל
\begin{align*}
\brs{v}_B &= \brs{\id_V v}_B
\\&=
\brs{\id_V}^{B_0}_{B} \brs{w}_{B_0}
\\&= A \pmat{1 \\ 0 \\ \vdots \\ 0}
\\&= \pmat{1 \\ \vdots \\ 1}
\end{align*}
לכן
$B$
הבסיס שחיפשנו.
\end{solution}

\begin{exercise}
יהיו
$V,W$
מרחבים וקטוריים סוף־מימדיים מעל שדה
$\mbb{F}$,
ותהי
$T \in \hom_{\mbb{F}}\prs{V,W}$.
הראו כי
$\rank T = 1$
אם ורק אם יש בסיסים
$B,C$
ל־%
$V,W$
בהתאמה כך שכל מקדמי
$\brs{T}^B_C$
הם
$1$.
\end{exercise}

\begin{solution}
נניח כי יש בסיסים
$B,C$
כמתואר. אז
$\rank T = \rank \brs{T}^B_C = 1$.

בכיוון השני, נניח כי
$\rank T = 1$.
כלומר,
$\dim \im T = 1$.
ממשפט המימדים מתקיים
$\dim V = \dim \ker T + \dim \im T$,
לכן
\[\text{.} \dim \ker T = \dim V - \dim \im T = \dim V - 1\]
יהי
$n \ceq \dim V$
ויהי
\[\tilde{B} \ceq \prs{u_1, \ldots, u_{n-1}}\]
בסיס של
$\ker T$.

יהי
$w$
וקטור פורש של
$\im T$
ויהי
$C$
בסיס של
$W$
כך שמתקיים
$\brs{w}_C = \pmat{1 \\ \vdots \\ 1}$,
שקיים לפי התרגיל הקודם.
יהי
$v \ceq T^{-1}\prs{w}$,
ואז
$\prs{v, u_1, \ldots, u_{n-1}}$
בלתי־תלויים לינארית, כי
$v \notin \ker T$.
לכן זה בסיס של
$V$.
אז גם
$B \ceq \prs{v, v + u_1, \ldots, v + u_{n-1}}$
בסיס של
$V$
כי
המטריצה
\[\pmat{\vert & & \vert \\ \brs{v}_{\tilde{B}} & \cdots & \brs{v + u_{n-1}}_{\tilde{B}} \\ \vert & & \vert} = \pmat{1 & 1 & \cdots & 1 & 1 \\ 0 & 1 & & 0 & 0 \\ \vdots & & \ddots & & \vdots \\& 0 & & 1 & 0 \\
0 & & \cdots & 0 & 1}\]
הפיכה.

נסמן
$C = \prs{w_1, \ldots, w_m}$.
מתקיים
\[T\prs{v} = w = w_1 + \ldots + w_m\]
ולכל
$i \in [n-1]$
מתקיים
\begin{align*}
T \prs{v + u_i} &= T\prs{v} + T\prs{u_i}
\\&= T\prs{v} + 0
\\&= T\prs{v}
\\&= w_1 + \ldots + w_m
\end{align*}
לכן
$
\brs{T}^B_C
$
מטריצה שכל מקדמיה הם
$1$.
\end{solution}

\end{document}