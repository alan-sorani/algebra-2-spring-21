\documentclass[a4paper,10pt,oneside,openany]{article}

\usepackage[lang=hebrew]{maths}
\usepackage{hebrewdoc}
\usepackage{stylish}
\usepackage{lipsum}
\let\bs\blacksquare

\title{
אלגברה ב' (104168) \textenglish{---} אביב 2020-2021
\\
תרגול 3 \textenglish{---} סכומים ישרים, מרחבים שמורים והטלות
}
\author{אלעד צורני}
\date{\today}

\begin{document}
\maketitle

\section{פולינומים}

\begin{definition}[הצבת מטריצה בפולינום]
עבור
$T \in \endo_{\mbb{F}}\prs{V}$.
נגדיר העתקה
\begin{align*}
\ev_T \colon \mbb{F}\brs{x} &\mapsto \endo_{\mbb{F}}\prs{V} \\
\text{.} \hphantom{lalalalal} p &\mapsto p\prs{T}
\end{align*}
\end{definition}

\begin{definition}[חוג]
\emph{חוג}
$\prs{R, +, \cdot}$
הוא חבורה אבלית
$\prs{R, +}$
עם העתקה
$\cdot \colon R \to R$
אסוציאטיבית כך שמתקיים
\[x \prs{y+z} = xy + xz\]
וגם
\[\text{.} \prs{y+z}x = yx + zx\]
\end{definition}

\begin{definition}[יחידה]
יהי
$R$
חוג.
איבר
$e \in R$
נקרא
\emph{יחידה}
אם
\[ex = xe = x\]
לכל
$x \in R$.

היחידה בדרך כלל מסומנת
$1$.
\end{definition}

\begin{example}
$\mbb{F}\brs{x}$
הוא חוג עם יחידה לפי חיבור וכפל פולינומים.
\end{example}

\begin{example}

\end{example}

\begin{definition}[אידאל]
יהי
$R$
חוג.

תת־חבורה
$I \leq \prs{R, +}$
נקראת
\emph{אידאל}
אם לכל
$x \in R$
מתקיים
$xI, Ix \subseteq I$.
\end{definition}

\begin{example}
יהי
$R = \mbb{Z}$
חוג עם פעילות החיבור והכפל הרגילות.
התת־חבורה
$2 \mbb{Z}$
היא אידאל ב־%
$R$
כי
\[\text{.} x \cdot 2 \mbb{Z} = \set{x \cdot 2 n}{n \in \mbb{Z}} = \set{2xn}{n \in \mbb{Z}} \subseteq 2\mbb{Z}\]
\end{example}

\begin{definition}[הומומורפיזם של חוגים]
יהיו
$R,S$
שני חוגים.
\emph{הומומורפיזם}
$\phi \colon R \to S$
הוא העתקה המקיימת
\begin{align*}
\phi\prs{x+y} &= \phi\prs{x} + \phi\prs{y} \\
\text{.} \phi\prs{xy} &= \phi\prs{x}\phi\prs{y}
\end{align*}
\end{definition}

\begin{definition}[גרעין]
יהי
$\phi \colon R \to S$
הומומורפיזם של חוגים.
נגדיר את
\emph{הגרעין של
$\phi$}
להיות
\[\text{.} \ker\phi \ceq \set{x \in R}{\phi\prs{x} = 0}\]
\end{definition}

\begin{exercise}
יהי
$\phi \colon R \to S$
הומומורפיזם של חוגים.
הראו כי
$\ker \phi \leq R$
אידאל.
\end{exercise}

\begin{solution}
יהי
$x \in R$
ויהי
$a \in \ker \phi$.
צריך להראות כי
$xa,ax \in \ker \phi$.
אכן
\begin{align*}
\phi\prs{xa} &= \phi\prs{x}\phi\prs{a} = \phi\prs{x} \cdot 0 = 0 \\
\text{.} \phi\prs{ax} &= \phi\prs{a} \phi\prs{x} = 0 \cdot \phi\prs{a} = 0
\end{align*}
\end{solution}

\end{document}