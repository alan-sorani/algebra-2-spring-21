\documentclass[a4paper,10pt,oneside,openany]{article}

\usepackage[lang=hebrew]{maths}
\usepackage{hebrewdoc}
\usepackage{stylish}
\usepackage{lipsum}
\let\bs\blacksquare

\title{
אלגברה ב' (104168) \textenglish{---} אביב 2020-2021
\\
תרגול 2 \textenglish{---} וקטורים עצמיים, שקילות בין העתקות וסכומים ישרים
}
\author{אלעד צורני}
\date{\today}

\begin{document}
\maketitle

\section{ערכים ווקטורים עצמיים}

\begin{definition}[ערך ווקטור עצמיים]
יהי
$V$
מרחב וקטורי ותהי
$T \in \endo_{\mbb{F}}\prs{V}$.
$\lambda \in \mbb{F}$
נקרא
\emph{ערך עצמי של
$T$}
אם קיים
$v \in V \setminus \set{0}$
עבורו
$Tv = \lambda v$.
$v$
כזה נקרא
\emph{וקטור עצמי
(של $T$)
עבור הערך העצמי
$\lambda$}.
\end{definition}

\begin{definition}[מרחב עצמי]
יהי
$\lambda$
ערך עצמי של העתקה
$T \in \endo\prs{V}$.
אוסף הוקטורים העצמיים של
$T$
עם ערך עצמי
$\lambda$
הוא תת־מרחב וקטורי של
$V$
שנקרא
\emph{המרחב העצמי של הערך העצמי
$\lambda$
(עבור $T$)}.
נסמנו בדרך כלל
$V_\lambda$.
\end{definition}

\begin{remark}
בסימון הנ"ל לא מצוין מה ההעתקה
$T$.
נבהיר מה ההעתקה כשהדבר אינו ברור מהקונטקסט.
\end{remark}

\begin{exercise}
תהיינה
$T,S,P \in \endo_{\mbb{F}}\prs{V}$
כך שמתקיים
\[T = P^{-1}SP\]
הוכיחו כי
$v \in V$
וקטור עצמי של
$T$
עם ערך עצמי
$\lambda$
אם ורק אם
$Pv$
וקטור עצמי של
$S$
עם ערך עצמי
$\lambda$.
\end{exercise}

\begin{solution}
יהי
$v \in V$
וקטור עצמי של
$T$
עם ערך עצמי
$\lambda$.
מתקיים
\begin{align*}
\text{.} SPv = \prs{PTP^{-1}}Pv = PTv = P \lambda v = \lambda \prs{Pv} 
\end{align*}

יהי
$v \in V$
כך ש־%
$Pv$
וקטור עצמי של
$S$
עם ערך עצמי
$\lambda$.
מתקיים
\begin{align*}
\text{.} Tv = P^{-1}SPv = P^{-1}\lambda Pv = \lambda P^{-1} P v = \lambda v
\end{align*}
\end{solution}

\begin{definition}[שדה סגור אלגברית]
שדה
$\mbb{F}$
נקרא
\emph{סגור אלגברית}
אם לכל פולינום
$p \in \mbb{F}\brs{x} \setminus \set{0}$
יש
$\deg\prs{p}$
שורשים מעל
$\mbb{F}$.
\end{definition}

\begin{remark}
באופן שקול,
$\mbb{F}$
סגור אלגברית אם לכל פולינום שאינו קבוע יש שורש מעל
$\mbb{F}$.
\end{remark}

\begin{fact}
יהי
$\mbb{F}$
שדה סגור אלגברי ויהי
$V$
מרחב וקטורי מעל
$\mbb{F}$.
לכל
$T \in \endo_{\mbb{F}}\prs{V}$
יש ערך עצמי.
\end{fact}

\begin{exercise}
יהי
$V$
מרחב וקטורי סוף־מימדי ממימד
$n > 1$
מעל שדה סגור אלגברית
$\mbb{F}$
ותהי
$T \in \endo\prs{V}$.
הראו כי
\[\text{.} \set{p\prs{T}}{p \in \mbb{F}\brs{x}} \neq \endo\prs{V}\]
\end{exercise}

\begin{solution}
\begin{itemize}
\item אם ל־%
$T$
יש ערך עצמי יחיד
$\lambda$
מריבוי גיאומטרי מלא, יש בסיס
$B$
של
$V$
בו
$\brs{T}_B = \lambda I$.
אז
\[\brs{p\prs{T}}_B = p\prs{\brs{T}_B} = p\prs{\lambda I} = p\prs{\lambda} I\]
מטריצה סקלארית.
אבל, לא כל אנדומורפיזם מיוצג על ידי מטריצה סקלארית, לכן אין שוויון.

\item
יהי
$\lambda \in \mbb{F}$
ערך עצמי של
$T$
ויהי
$v \in V$
וקטור עצמי עבור
$\lambda$.
מהרדוקציה הקודמת, הריבוי הגיאומטרי של
$\lambda$
אינו
$n$.
נראה שיש העתקה
$S \in \endo_{\mbb{F}}\prs{V}$
שאינה מתחלפת עם
$T$,
וכיוון שכל פולינום ב־%
$T$
מתחלף עם
$T$
נקבל
\[\text{.} S \notin \set{p\prs{T}}{p \in \mbb{F}\brs{x}}\]
$\lambda$
מריבוי גיאומטרי קטן מ־%
$n$
לכן יש
$w \in V \setminus V_\lambda$.
אז
$\prs{v, w}$
בלתי־תלויה לינארית ולכן ניתן להשלים אותה לבסיס
\[\text{.} B = \prs{v,w, u_3, \ldots, u_n}\]
תהי
\begin{align*}
S \colon V &\to V \\
v &\mapsto w \\
w &\mapsto v \\
\text{.} \hphantom{lala} u_i &\mapsto u_i
\end{align*}
אכן
\begin{align*}
ST\prs{v} &= S\lambda v = \lambda Sv = \lambda w \\
TS\prs{v} &= Tw \neq \lambda w
\end{align*}
כאשר האי־שוויון נכון כי
$w \notin V_\lambda$.
לכן
$S$
אינה פולינום ב־%
$T$.
\end{itemize}
\end{solution}

\section{שקילות בין העתקות}

\begin{exercise}
נאמר שהעתקות
$T \in \endo_{\mbb{F}}\prs{V}$
ו־%
$S \in \endo_{\mbb{F}}\prs{W}$
\emph{צמודות}
ונסמן
$T \sim S$
אם קיים איזומורפיזם
$P \colon V \to W$
עבורו
$T = P^{-1} S P$.

\begin{enumerate}
\item הראו כי
$\sim$
יחס שקילות.
\item ראיתם בהרצאה שמטריצות מייצגות של העתקות צמודות הן צמודות.
הסיקו שכל שתי מטריצות מייצגות של אותה העתקה הן צמודות.
\end{enumerate}
\end{exercise}

\begin{solution}
\begin{enumerate}
\item
\begin{description}
\item[רפלקסיביות:]
תהי
$T \in \endo_{\mbb{F}}\prs{V}$.
מתקיים
\[T = \id_V^{-1} \circ T \circ \id_V\]
לכן
$T \sim T$.
\item[סימטריות:]
תהיינה
$T \in \endo_{\mbb{F}}\prs{V}$
ו־%
$S \in \endo_{\mbb{F}}\prs{W}$
עבורן
$T \sim S$.
קיים איזומורפיזם
$P \colon V \to W$
עבורו
$T = P^{-1} S P$.
אז
$S = P T P^{-1}$.
נסמן
$Q = P^{-1}$
ונקבל
\[\text{.} S = Q^{-1} T Q\]
\item[טרנזיטיביות:]
תהיינה
\begin{align*}
T &\in \endo_{\mbb{F}}\prs{U} \\
S &\in \endo_{\mbb{F}}\prs{V} \\
R &\in \endo_{\mbb{F}}\prs{W}
\end{align*}
המקיימות
$T \sim S$
וגם
$S \sim R$.
קיימים איזומורפיזמים
$P \colon U \to V$
ו־%
$Q \colon V \to W$
עבורם
$T = P^{-1} S P$
וגם
$S = Q^{-1} R Q$.
אז
\[T = P^{-1}\prs{Q^{-1} R Q}P = \prs{QP}^{-1} R \prs{QP}\]
כאשר
$QP$
איזומורפיזם כהרכבת איזומורפיזמים.
לכן
$T \sim R$.
\end{description}

\item
מרפלקסיביות נקבל
$T \sim T$.
לכן, מהטענה בכיתה, כל שתי מטריצות מייצגות של
$T$
צמודות.
\end{enumerate}
\end{solution}

\begin{exercise}
מטריצות
$A_1, A_2 \in M_n\prs{\mbb{F}}$
נקראות
\emph{צמודות}
אם יש
$P \in M_n\prs{\mbb{F}}$
הפיכה עבורה
$A_1 = P^{-1} A_2 P$.
במקרה זה נסמן
$A_1 \sim A_2$.

יהיו
$A_1, A_2 \in M_n\prs{\mbb{F}}$
צמודות. הראו שיש העתקה
$T \colon \mbb{F}^n \to \mbb{F}^n$
ובסיסים
$B,C$
של
$\mbb{F}^n$
עבורם
$\brs{T}_B = A_1, \brs{T}_C = A_2$.
\end{exercise}

\begin{solution}
נתון
$A_1 \sim A_2$
לכן קיימת
$P$
הפיכה עבורה
$A_1 = P A_2 P^{-1}$.
תהי
$T = L_{A_1}$
ויהי
$B = E$
הבסיס הסטנדרטי של
$\mbb{F}^n$.
אז
\[\text{.} \brs{T}_B = \brs{L_{A_1}}_E = A_1\]
$P$
הפיכה לכן יש בסיס
$C$
עבורו
$P = P^C_E$.
אז מתקיים
\begin{align*}
A_2 \brs{v}_C &= P^{-1} A_1 P \brs{v}_C
\\&= P^E_C A_1 P^C_E \brs{v}_C
\\&= P^E_C \brs{L_{A_1}}_E \brs{v}_E
\\&= P^E_C \brs{L_{A_1} v}_E
\\&= \brs{L_{A_1} v}_C
\\&= \brs{T v}_C
\end{align*}
לכן
$A_2 = \brs{T}_C$.
\end{solution}

\begin{exercise}
תהיינה
$T \in \endo_{\mbb{F}}\prs{V}$
ו־%
$S \in \endo_{\mbb{F}}\prs{W}$
ויהי
$B$
בסיס של
$V$.
הראו שמתקיים
$T \sim S$
אם ורק אם יש בסיס
$C$
של
$W$
עבורו
\[\text{.} \brs{T}_B = \brs{S}_C\]
\end{exercise}

\begin{solution}
\begin{itemize}
\item נניח כי
$T \sim S$
ויהי
$P \colon V \riso W$
המקיים
$T = P^{-1} S P$.
כיוון ש־%
$I$
הפיכה ו־%
$P$
איזומורפיזם,
ראינו שיש בסיס
$C$
של
$W$
עבורו
$\brs{P}^B_C = I$.
אז מתקיים
\begin{align*}
\brs{T}_B &= \brs{P^{-1} S P}_B
\\&= \brs{P^{-1}}_B^C \brs{S}_C \brs{P}^B_C
\\&= I^{-1} \brs{S}_C I
\\&= \brs{S}_C
\end{align*}
כנדרש.
\item
נניח כי יש בסיס
$C$
של
$W$
עבורו
$\brs{T}_B = \brs{S}_C$.
נכתוב
\begin{align*}
B &= \prs{v_1, \ldots, v_n} \\
C &= \prs{w_1, \ldots, w_n}
\end{align*}
ותהי
\begin{align*}
P \colon V &\to W \\
\text{.} v_i &\mapsto w_i
\end{align*}
אז
$P$
איזומורפיזם כי היא שולחת בסיס לבסיס.
מתקיים
\begin{align*}
\brs{P^{-1} S P}_B &= \brs{P^{-1}}^C_B \brs{S}_C \brs{P}^B_C
\\&= I \brs{S}_C I
\\&= \brs{S}_C
\\&= \brs{T}_B
\end{align*}
לכן לפי תרגיל משיעורי הבית
\[\text{.} P^{-1} S P = T\]
\end{itemize}
\end{solution}

\section{סכומים ישרים}

\begin{definition}
יהיו
$B_1 \ceq \prs{v_1, \ldots, v_m}$
ו־%
$B_2 \ceq \prs{w_1, \ldots, w_n}$
קבוצות סדורות.
נגדיר את
\emph{השרשור שלהן}
להיות
\[\text{.} B_1 * B_2 = \prs{v_1, \ldots, v_m, w_1, \ldots, w_m}\]
\end{definition}

\begin{definition}[סכום ישר]
יהי
$V$
מרחב וקטורי ויהיו
$V_1, V_2 \leq V$.
אם
$V_1 \cap V_2 = \set{0}$
נקרא לסכום
$V_1 + V_2$
\emph{סכום ישר}
ונסמנו
$V_1 \oplus V_2$.
\end{definition}

\begin{exercise}
יהי
$V$
מרחב וקטורי ויהיו
$V_1, \ldots, V_n \leq V$
עבורם
\[\text{.} V = \bigoplus_{i \in [n]} V_i \ceq V_1 \oplus \ldots \oplus V_n\]
לכל
$i \in [n]$
יהי
$B_i$
בסיס של
$V_i$.
הראו כי
\[B \ceq B_1 * \ldots * B_n\]
בסיס של
$V$.
\end{exercise}

\begin{solution}
יהי
$v \in V$.
יש
$v_1, \ldots, v_n \in V$
כך ש־%
$v_i \in V_i$
ויש
$\prs{\alpha_i}_{i \in [n]} \in \mbb{F}$
כך שמתקיים
\[\text{.} v = \sum_{i \in [n]} \alpha_i v_i\]
נכתוב
\[\text{.} B_i = \prs{u_{i,1}, \ldots, u_{i, n_i}}\]
לכל
$i \in [n]$
אפשר כתוב
\[\text{.} v_i = \sum_{j \in [n_i]} \beta_j u_{i,j}\]
אז
\[\text{.} v = \sum_{i \in [n]} \sum_{j \in [n_i]} \alpha_i \beta_j u_{i,j}\]
לכן
$\spn B = V$.

נראה כי
$B$
בלתי־תלויה לינארית.
נניח כי יש
$\alpha_{i,j}$
כך ש־%
\begin{align*}
\text{.} \sum_{i \in [n]} \sum_{j \in [n_i]} \alpha_{i,j} u_{i,j} = 0
\end{align*}
מהגדרת הסכום הישר, לכל
$i \in [n]$
מתקיים
\begin{align*}
\text{.} \sum_{j \in [n_i]} \alpha_{i,j} u_{i,j} = 0
\end{align*}
כיוון ש־%
$B_i$
בסיס נקבל מכך
$\alpha_{i,j} = 0$
לכל
$j \in [n_i]$.
לכן כל המקדמים שווים
$0$,
ולכן
$B$
בלתי־תלויה לינארית.
\end{solution}

\begin{exercise}
יהי
$V$
מרחב וקטורי ויהיו
$W,U \leq V$
עם בסיסים
$B,C$
בהתאמה.
נניח כי
$B * C$
בסיס של
$V$.
הראו כי
\[\text{.} V = W \oplus U\]
\end{exercise}

\begin{solution}
נסמן
\begin{align*}
B &= \prs{w_1, \ldots, w_k} \\
\text{.} C &= \prs{u_1, \ldots, u_\ell}
\end{align*}
יהי
$v \in V$.
אפשר לכתוב
\[v = \sum_{i \in [k]} \alpha_i w_i + \sum_{i \in [\ell]} \beta_i u_i \in W + U\]
לכן
$V = W + U$.

אם
$v \in W \cap U$
נוכל לכתוב
\[v = \sum_{i\in[k]} \alpha_i w_i = \sum_{i \in [\ell]} \beta_i u_i\]
ואז
\[\text{.} 0 = \sum_{i \in [k]} \alpha_i w_i - \sum_{i \in [\ell]} \beta_i u_i \]
כיוון ש־%
$B * C$
בסיס של
$V$
זאת קבוצה בלתי־תלויה לינארית ולכן נקבל
$\alpha_i, \beta_i = 0$
לכל
$i$.
לכן
$v = 0$,
כלומר
$W \cap U = \set{0}$.
לכן הסכום ישר.
\end{solution}

\begin{exercise}
יהי
$V$
מרחב וקטורי ויהי
$W \leq V$.
הראו שקיים
$U \leq V$
עבורו
$V = W \oplus U$.
\end{exercise}

\begin{solution}
יהי
$B$
בסיס של
$W$.
אז
$B$
קבוצה בת"ל ב־%
$V$
וראינו באלגברה ב' שאפשר להשלים קבוצה כזאת לבסיס של
$V$.
לכן יש
$B'$
בת"ל כך ש־%
$B * B'$
בסיס של
$V$.
מהתרגיל הקודם נקבל
\[\text{.} V = W \oplus \spn B'\]
\end{solution}

\end{document}