\documentclass[a4paper,10pt,oneside,openany]{article}

\usepackage[lang=hebrew]{maths}
\usepackage{hebrewdoc}
\usepackage{stylish}
\usepackage{lipsum}
\let\bs\blacksquare
\usepackage{ytableau}

\usepackage{graphicx, txfonts}

\newcommand{\heart}{\ensuremath\varheartsuit}

\title{
אלגברה ב' (104168) \textenglish{---} אביב 2020-2021
\\
תרגול 10 \textenglish{---}
תכונות של העתקות, פירוק לערכים סינגולריים,
פירוק פולארי ותבניות בילינאריות
}
\author{אלעד צורני}
\date{\today}

\begin{document}
\maketitle

\section{העתקות נורמליות וכו'}

\subsection{חזרה}

\begin{definition}
יהי
$V$
מרחב מכפלה פנימית מעל
$F = \mbb{R}$
($\mbb{F} = \mbb{C}$).
$T \in \endo_{\mbb{F}}\prs{V}$
תיקרא
\begin{itemize}
\item \emph{נורמלית}
אם
$T^* T = T T^*$.
\item \emph{צמודה לעצמה (הרמיטית)}
אם
$T = T^*$.
\item \emph{אורתוגונלית (יוניטרית)}
אם
$T^* = T^{-1}$.
\end{itemize}
\end{definition}

\begin{theorem}[משפט הפירוק הספקטרלי]
$T \in \endo_{\mbb{C}}\prs{V}$
נורמלית אם ורק אם קיים בסיס אורתונורמלי לפיו היא אלכסונית.
\end{theorem}

\subsection{תרגילים}

\begin{exercise}
בתרגיל זה נראה שכל העתקה היא צירוף לינארי של
$4$
העתקות יוניטריות.

יהי
$V$
מרחב מכפלה פנימית מעל
$\mbb{C}$
ותהי
$T \in \endo_{\mbb{C}}\prs{V}$.
\begin{enumerate}
\item נניח כי
$T$
הרמיטית ועם ערכים עצמיים אי־שליליים.
הראו כי יש
$S \in \endo_{\mbb{C}}\prs{V}$
הרמיטית עם ערכים עצמיים אי־שליליים עבורה
$S^2 = T$.
נסמנה
$\sqrt{T}$.

\item
נניח כי
$T$
הרמיטית. הראו כי הערכים העצמיים של
$T$
ממשיים.

\item נסמן ב־%
$\sigma\prs{T}$
את אוסף הערכים העצמיים של
$T$
ונסמן
\[\text{.} c \prs{T} \ceq \max\set{\abs{\lambda}}{\lambda \in \sigma\prs{T}}\]
תהי
\[\text{.} \tilde{T} \ceq \frac{T}{c\prs{T}}\]
הראו כי
$\id_V - \tilde{T}$
הרמיטית עם ערכים עצמיים אי־שליליים.

\item הראו כי
$T$
צירוף לינארי של שתי העתקות הרמיטיות.

\item 
נניח ש־%
$T$
הרמיטית.
הראו ש־%
$T$
צירוף לינארי של שתי העתקות יוניטריות.

\item
הסיקו כי
$T$
צירוף לינארי של ארבע העתקות יוניטריות.
\end{enumerate}
\end{exercise}

\begin{solution}
\begin{enumerate}
\item $T$
הרמיטית, לכן נורמלית ולכן לפי משפט הפירוק הספקטרלי קיים בסיס אורתונורמלי
$B = \prs{v_1, \ldots, v_n}$
עבורו
$\brs{T}_B$
אלכסונית עם ערכים עצמיים
$\lambda_1, \ldots, \lambda_n \geq 0$
על האלכסון.
נגדיר
\begin{align*}
S \colon V &\to V \\
v_i &\mapsto \sqrt{\lambda_i} v_i
\end{align*}
ואז
\begin{align*}
\text{.} S\prs{v_i} = \lambda_i v_i
\end{align*}
נקבל
$\brs{S}_B = \brs{T}_B$
ולכן
$S = T$.
$S$
הרמיטית כי
$\overline{\brs{S}_B}^t = \brs{S}_B$
וכי
$B$
בסיס אורתונורמלי.

\item 
כמו מקודם, קיים בסיס אורתונורמלי
$B = \prs{v_1, \ldots, v_n}$
של
$V$
לפיו
$T$
לכסינה.
אבל,
$\brs{T}_B = \overline{\brs{T}_B}^t$
ולכן לכל
$i \in [n]$
מתקיים
$\lambda_i = \overline{\lambda_i}$,
ואז
$\lambda_i$
ממשי.

\item
יהי
$B$
בסיס אורתונורמלי עבורו
$\brs{T}_B$
אלכסונית. אז
\[\brs{\tilde{T}}_B = \frac{1}{c\prs{T}} \brs{T}_B\]
ונקבל כי הערכים העצמיים של
$\tilde{T}$
בקטע
$\brs{0,1}$.
אז גם
$\brs{\id_V - \tilde{T}^2}_B$
אלכסונית עם ערכים עצמיים בקטע
$\brs{0,1}$.
כיוון ש־%
$B$
אורתונורמלי, נקבל כי
$\id_V - \tilde{T}^2$
הרמיטית.

\item נכתוב
\[\text{.} T = \frac{T+T^*}{2} + \frac{T-T^*}{2}\]
מתקיים
\begin{align*}
\prs{\frac{T + T^*}{2}}^* &= \frac{T^* + T^{**}}{2} = \frac{T+T^*}{2} \\
\prs{\frac{T - T^*}{2}}^* &= \frac{T^* - T^{**}}{2} = \frac{T^* - T}{2}
\end{align*}
אבל
\[\prs{\frac{T-T^*}{2i}}^* = i \prs{\frac{T-T^*}{2}}^* = -i {\frac{T-T^*}{2}} = \frac{T - T^*}{2i}\]
ולכן
\begin{align*}
T &= \frac{T+T^*}{2} + i \prs{\frac{T-T^*}{2i}}
\end{align*}
צירוף לינארי של שתי העתקות הרמיטיות.

\item 
מספיק להראות כי
$\tilde{T}$
צירוף של שתי העתקות יוניטריות, לכן נניח
$c\prs{T} = 1$.
נרצה
$f\prs{T}, g\prs{T}$
יוניטריות עבורן
\[\text{.} T = \frac{f\prs{T} + g\prs{T}}{2}\]
כמו מקודם, נכתוב
$f\prs{T} = A + iB$
עבור
$A,B$
הרמיטיות.
אם ניקח
$g\prs{T} = A - iB$
נקבל
\[\text{.} T = \frac{A + iB + A - iB}{2} = A\]
אז
$f\prs{T} = T + iB$.
$T$
הרמיטית ולכן נורמלית ומתחלפת עם
$T^*$.
לכן היא מתחלפת עם
$B$.
מיוניטריות נקבל
\begin{align*}
\id_V &= f\prs{T} f\prs{T}^*
\\&= \prs{T + iB} \prs{T - iB}
\\&= T^2 + B^2
\end{align*}
ואז
\[\text{.} B^2 = \id_V - T^2\]
כיוון שראינו ש־%
$\id_V - T^2$
הרמיטית עם ערכים עצמיים אי־שליליים ולכן קיים לה שורש.
ניקח
$B = \sqrt{\id_V - T^2}$.
אז אכן מתקיים
$\id_V = T^2 + B^2 = f\prs{T} f\prs{T}^*$
ולכן
$f\prs{T}$
יוניטרית.
נשים לב כי
$g\prs{T} = f\prs{T}^*$
ולכן גם
$g\prs{T}$
יוניטרית.
\item ראינו כי 
$T = \alpha H_1 + \beta H_2$
צירוף של שתי העתקות הרמיטיות וכי
$H_1, H_2$
כל אחת צירוף של שתי העתקות יוניטריות, לכן
\[T = \alpha \prs{a U_1 + b U_2} + \beta \prs{c U_3 + d U_4}\]
צירוף לינארי של ארבע העתקות יוניטריות.
\end{enumerate}
\end{solution}

\section{פירוק לערכים סינגולריים}

\subsection{חזרה}

\begin{definition}[ערך סינגולרי]
תהי
$T \in \endo_{\mbb{F}}\prs{V}$.
\emph{הערכים הסינגולריים של
$T$}
הם הערכים העצמיים של
$\sqrt{T^* T}$,
כאשר כל אחד נספר מספר פעמים כריבוי הגיאומטרי שלו.
\end{definition}

\begin{theorem}[פירוק לערכים סינגולריים]
תהי
$T \in \endo_{\mbb{F}}\prs{V}$.
קיימים בסיסים אורתונורמליים
$B,C$
של
$V$
עבורם
$\brs{T}^B_C$
אלכסונית כאשר ערכי האלכסון הם הערכים הסינגולריים של
$T$
כולל כפילויות.
\end{theorem}

\begin{theorem}[פירוק לערכים סינגולריים של מטריצה]
תהי
$A \in M_{n \times m}\prs{\mbb{F}}$.
קיימות
$U \in M_n\prs{\mbb{F}}$
ו־%
$V \in M_m\prs{\mbb{F}}$
אורתוגונליות (יוניטריות) וקיימת מטריצה מלבנית אלכסונית
$\Sigma$
עבורן
\[\text{.} A = U \Sigma V^*\]
\end{theorem}

\subsection{חישוב}

\begin{algorithm}[מציאת \textenglish{SVD}]
כאשר נרצה למצוא פירוק לערכים סינגולריים נבצע את השלבים הבאים.

\begin{enumerate}
\item נמצא בסיס אורתונורמלי של וקטורים עצמיים עבור
$T^* T$. אפשר לעשות את זה למשל לפי גרם־שמידט.
נסדר אותו לפי סדר יורד של הערכים העצמיים.
זה יהיה הבסיס
$B$.

\item עבור כל וקטור
$v$
ב־%
$B$
נסתכל על
$\frac{1}{\sigma} T\prs{v}$
כאשר
$\sigma$
הערך הסינגולרי המתאים לו.
אם
$B = \prs{v_1, \ldots, v_n}$
נקבל קבוצה סדורה אורתונורמלית
\[\text{.} \tilde{C} = \prs{\frac{1}{\sigma_1} T\prs{v_1}, \ldots, \frac{1}{\sigma_n} T\prs{v_n}}\]

\item אם אנו עובדים עם מטריצות, נשלים את
$\tilde{C}$
לבסיס אורתונורמלי
$C$
של
$\prs{\mbb{F}}^m$.
אחרת ניקח
$C = \tilde{C}$.
\end{enumerate}
\end{algorithm}

\begin{example}
תהי
\[\text{.} A = \pmat{1 & -1 \\ 0 & 1 \\ 1 & 0} \in M_{3,2}\prs{\mbb{C}}\]
מתקיים
$A A^* = \pmat{2 & -1 \\ - 1 & 2}$.
הערכים העצמיים של
$A A^*$
הם
$3,1$
ולכן הערכים הסינגולריים של
$A$
הם
$\sqrt{3}, 1$.
$e_1 - e_2$
וקטור עצמי של
$3$
ו־%
$e_1 + e_2$
וקטור עצמי של
$1$.
ננרמל את שני הוקטורים ונקבל בסיס אורתונורמלי
$B = \prs{\frac{e_1 - e_2}{\sqrt{2}}, \frac{e_1 + e_2}{\sqrt{2}}}$
של
$\mbb{C}^2$.

נגדיר
\begin{align*}
\tilde{C} &= \prs{\frac{1}{\sqrt{3}} A \prs{\frac{e_1 - e_2}{\sqrt{2}}}, A\prs{\frac{e_1 + e_2}{\sqrt{2}}}}
\\&= \prs{\frac{1}{\sqrt{6}} \pmat{2\\-1\\1}, \frac{1}{\sqrt{2}} \pmat{0\\1\\1}}
\end{align*}
ונשלים קבוצה סדורה זאת לבסיס אורתונורמלי של
$\mbb{C}^3$.
נסמן
$w = \pmat{a\\b\\c}$
ונפתור את המשוואות
\begin{align*}
2a - b + c &= 0\\
b + c &= 0
\end{align*}
כדי לקבל
$w = \pmat{1\\1\\-1}$.
כעת,
$\norm{w} = 3$
לכן לאחר נרמול נקבל
$v_3 = \frac{1}{\sqrt{3}} \pmat{1\\1\\-1}$.
נגדיר
\begin{align*}
\text{.} C \ceq \tilde{C} * \prs{v_3}
\end{align*}

אז
\[A = \brs{\id_{\mbb{C}^3}}^C_E \pmat{\sqrt{3} & 0 \\ 0 & 1 \\ 0 & 0} \brs{\id_{\mbb{C}^2}}^E_B\]
כאשר מתקיים
\[\brs{\id_{\mbb{C}^2}}^E_B = \prs{\brs{\id_{\mbb{C}^2}}^B_E}^* = \frac{1}{\sqrt{2}} \pmat{1 & - 1 \\ 1 & 1}\]
וגם
\[\text{.} \brs{\id_{\mbb{C}^3}}^C_E = \pmat{\frac{2}{\sqrt{6}} & 0 & \frac{1}{\sqrt{3}} \\ -\frac{1}{\sqrt{6}} & \frac{1}{\sqrt{2}} & \frac{1}{\sqrt{3}} \\ \frac{1}{\sqrt{6}} & \frac{1}{\sqrt{2}} & -\frac{1}{\sqrt{3}}}\]
\end{example}

\subsection{תרגילים}

\begin{exercise}
הוכיחו או הפריכו: הערכים הסינגולריים של
$T^2$
הם ריבועי הערכים הסינגולריים של
$T$.
\end{exercise}

\begin{solution}
הטענה איננה נכונה, למשל עבור
$A = \pmat{0 & 1 \\ 0 & 0}$.
\end{solution}

\section{פירוק פולארי}

\subsection{חזרה}

\begin{definition}[אופרטור מוגדר אי־שלילית]
$T \in \endo_{\mbb{F}}\prs{V}$
צמוד לעצמו (הרמיטי) נקרא
\emph{מוגדר אי־שלילית}
אם
$\trs{Tv, v} \geq 0$
לכל
$v \in V$.
\end{definition}

\begin{theorem}[פירוק פולארי]
תהי
$T \in \endo_{\mbb{F}}\prs{V}$.
יש איזומטריה
$S \in \endo_{\mbb{F}}\prs{V}$
עבורה
$T = S \sqrt{T^* T}$.
\end{theorem}

\begin{remark}
האופרטור
$T^* T$
מוגדר אי־שלילית כי
$\trs{T^* Tv, v} = \trs{Tv, Tv} \geq 0$
לכל
$v \in V$.
ראינו בתרגול הקודם שעבור
$\mbb{F} = \mbb{C}$
יש לו שורש יחיד מוגדר אי־שלילית, וכי שורש זה הרמיטי.
\end{remark}

\begin{algorithm}
כדי למצוא פירוק פולארי ניעזר בפירוק לערכים סינגולריים.
ניתן לכתוב
$A = W \Sigma V^*$.
ניקח
$P = V \Sigma V^*$
ו־%
$U = W V^*$.
אז
$P$
מוגדרת אי־שלילית כמטריצה צמודה ל־%
$\Sigma$,
ו־%
$U$
אורתוגונלית (יוניטרית) כהרכבה של כאלה.
\end{algorithm}

\subsection{תרגילים}

\begin{exercise}
מצאו את הפירוק הפולארי עבור
\begin{align*}
T \colon \mbb{R}^3 &\to \mbb{R}^3 \\
\text{.} \hphantom{lalala} \pmat{x\\y\\z} &\mapsto \pmat{z \\ 2x \\ 3y}
\end{align*}
\end{exercise}

\begin{solution}
מתקיים
\begin{align*}
\brs{T}_E &= \pmat{0 & 0 & 1 \\ 2 & 0 & 0 \\ 0 & 3 & 0} \\
\brs{T^*}_E &= \pmat{0 & 2 & 0 \\ 0 & 0 & 3 \\ 1 & 0 & 0} \\
\brs{T^* T}_E &= \pmat{4 & 0 & 0 \\ 0 & 9 & 0 \\ 0 & 0 & 1}
\end{align*}
ולכן הערכים הסינגולריים של
$T$
הם
$3,2,1$.
יש לנו בסיס אורתונורמלי
$B = \prs{e_2, e_1, e_3}$
של וקטורים עצמיים.
כדי לקבל את הבסיס השני, נפעיל את
$T$
וננרמל.
נקבל
\begin{align*}
\text{.} C &= \pmat{\frac{1}{3} T\prs{e_2}, \frac{1}{2} T\prs{e_1}, T\prs{e_3}}
= \prs{e_3, e_2, e_1}
\end{align*}
אז
\begin{align*}
\text{.} \brs{T}_E = W \Sigma V^*
\end{align*}
עבור
\begin{align*}
\Sigma &= \brs{\sqrt{T^* T}}_B = \pmat{3&0&0\\0&2&0\\0&0&1} \\
V &= \brs{\id_{\mbb{C}^3}}^B_E = \pmat{0&1&0\\1&0&0\\0&0&1}\\
\text{.} W &= \brs{\id_{\mbb{C}^3}}^C_E = \pmat{0&0&1\\0&1&0\\1&0&0}
\end{align*}
מתקיים
\[V \Sigma V^* = P^B_E \brs{\sqrt{T^* T}}_B P^E_B = \brs{\sqrt{T^* T}}_E\]
ולכן
\begin{align*}
\brs{T}_E &= W \Sigma V^*
\\&=
W V^* V \Sigma V^*
\\&=
W V^* \brs{\sqrt{T^* T}}_E
\\&=
\brs{\sqrt{T^* T}}^E_C
\end{align*}
נסמן
$U \ceq W V^*$,
שהינה יוניטרית כמכפלת יוניטריות ונסמן
$S \ceq \prs{\rho^E_E}^{-1}\prs{U}$,
שהינה יוניטרית כי היא מיוצגת על ידי מטריצה יוניטרית בבסיס אורתונורמלי.
נקבל
$T = S \sqrt{T^* T}$.
כאשר
\[\text{.} \brs{S}_E = U = WV^* = \pmat{0 & 0 & 1 \\ 1 & 0 & 0 \\ 0 & 1 & 0}\]
\end{solution}

\begin{remark}
באופן כללי ראינו בתרגיל כי
$\brs{T}_E =  P^C_E P^B_E$
כאשר
$B,C$
הבסיסים מהפירוק לערכים סינגולריים.
\end{remark}

\end{document}