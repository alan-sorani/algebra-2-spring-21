\documentclass[a4paper,10pt,oneside,openany]{article}

\usepackage[lang=hebrew]{maths}
\usepackage{hebrewdoc}
\usepackage{stylish}
\usepackage{lipsum}
\let\bs\blacksquare

\title{
אלגברה ב' (104168) \textenglish{---} אביב 2020-2021
\\
תרגול 3 \textenglish{---} סכומים ישרים, מרחבים שמורים והטלות
}
\author{אלעד צורני}
\date{\today}

\begin{document}
\maketitle

\section{תת־מרחבים שמורים}

\subsection{חזרה}

\begin{definition}[תת־מרחב שמור]
תהי
$T \in \endo_{\mbb{F}}\prs{V}$.
תת־מרחב
$W \leq V$
יקרא
\emph{$T$%
־שמור}
אם
$T\prs{W} \subseteq W$.
\end{definition}

\begin{definition}[צמצום של העתקה]
יהיו
$T \in \endo_{\mbb{F}}\prs{V}$
ו־%
$W \leq V$
מרחב
$T$%
־שמור.
נגדיר את
\emph{הצמצום של
$T$
ל־%
$W$}
על ידי
\begin{align*}
\rest{T}{W} \colon W &\to W \\
\text{.} \hphantom{lalalala} w &\mapsto T\prs{w}
\end{align*}
\end{definition}

\begin{remark}
הגדרנו בעבר צמצום של העתקה כללית
$T \colon V_1 \to V_2$.
בדרך כלל נצמצם העתקות רק לתת־מרחבים שמורים, ואז נתייחס להגדרה הנוכחית.
במקרים אחרים נבהיר במיוחד את הכוונה.
\end{remark}

\begin{definition}[שרשור של בסיסים]
יהיו
\begin{align*}
B_1 &= \prs{v_{1,1}, \ldots, v_{1,n_1}} \\
B_2 &= \prs{v_{2,1}, \ldots, v_{2,n_2}} \\
\vdots \\
B_k &= \prs{v_{k,1}, \ldots, v_{k,n_k}}
\end{align*}
קבוצות סדורות.
נגדיר את
\emph{השרשור שלהן}
על ידי
\[\text{.} B_1 * \ldots * B_k = \prs{v_{1,1}, \ldots, v_{1,n_1}, v_{2,1} \ldots, v_{2,n_2}, \ldots, v_{k,1}, \ldots, v_{k, n_k}}\]
\end{definition}

\subsection{תרגילים}

\begin{exercise}
יהי
$\mbb{C}$
כמרחב וקטורי ממשי ותהי
\begin{align*}
T \colon \mbb{C} &\to \mbb{C} \\
\text{.} \hphantom{lala} z &\mapsto iz
\end{align*}
מצאו את התת־מרחבים ה־%
$T$%
־שמורים של
$\mbb{C}$
והסיקו כי
$T$
אינה לכסינה מעל
$\mbb{R}$.
\end{exercise}

\begin{solution}
$\mbb{C}, \set{0}$
תת־מרחבים
$T$%
־שמורים.

נניח כי
$W \leq \mbb{C}$
מרחב
$T$%
־שמור נוסף. אז
$\dim_{\mbb{R}}\prs{W} = 1$
ולכן יש
$z_0 \in \mbb{C}^* \ceq \set{z \in \mbb{C}}{z \neq 0}$
עבורו
$W = \spn\set{z_0}$.
נקבל
$Tz_0 \in W$
לכן
$Tz_0 = c z_0$
עבור
$c \in \mbb{R}$.
אבל
$c z_0 = i z_0$
גורר
$c = i$
בסתירה.

תת־מרחבים $T$־שמורים של
$\mbb{C}$
מתאימים לוקטורים עצמיים. לכן אין ל־%
$T$
וקטורים עצמיים ממשיים, ולכן אינה לכסינה מעל
$\mbb{R}$.
\end{solution}

\begin{exercise}
תהי
\begin{align*}
T \colon \mbb{F}^4 \to \mbb{F}^4
\end{align*}
עם
\[\text{.} \brs{T}_E = \pmat{0 & 1 & 0 & 0 \\ 1 & 0 & 0 & 0 \\ 0 & 0 & 0 & 1 \\ 0 & 0 & 1 & 0}\]
מצאו בסיס
$B$
של
$\mbb{F}^n$
בו אין ב־%
$\brs{T}_B$
בלוקים לא טריוויאליים.
\end{exercise}

\begin{remark}
כל מטריצה
$A \in M_n\prs{\mbb{F}}$
היא אלכסונית בלוקים עם בלוק מגודל
$n \times n$,
כך שאין משמעות לשאלה האם מטריצה היא אלכסונית בלוקים. יש טעם לשאול האם יש במטריצה בלוקים שאינם טריוויאליים, או האם היא אלכסונית בלוקים עם גדלים מסוימים של בלוקים.
\end{remark}

\begin{solution}
מתקיים
\begin{align*}
T e_1 &= e_2 \\
\text{.} T e_2 &= e_1
\end{align*}
יהי
$B = \prs{e_1, e_4, e_3, e_2}$.
מתקיים
\begin{align*}
\text{.} \brs{T}_B = \pmat{
0 & 0 & 0 & 1 \\
0 & 0 & 1 & 0 \\
0 & 1 & 0 & 0 \\
1 & 0 & 0 & 0
}
\end{align*}
\end{solution}

להעתקה
$T$
הנ"ל יש תת־מרחבים שמורים
$\spn\prs{e_1, e_2}, \spn\prs{e_3, e_4}$,
אבל מצאנו בסיס
$B$
בו
$\brs{T}_B$
אינה אלכסונית בלוקים באופן לא טריוויאלי.
נרצה לתאר באופן כללי את הקשר בין מטריצות אלכסוניות בלוקים לבין תת־מרחבים שמורים.

\begin{exercise}
תהי
$T \in \endo_{\mbb{F}}\prs{V_1 \oplus \ldots \oplus V_k}$
ויהיו
$B_1, \ldots, B_k$
בסיסים עבור
$V_1, \ldots, V_k$
בהתאמה.
יהיו
$n_i \ceq \dim_{\mbb{F}} V_i$
ויהי
$B = B_1 * \ldots * B_n$.
הראו שהתנאים הבאים שקולים.
\begin{enumerate}
\item $\brs{T}_B$
מטריצה
$\prs{n_1, \ldots, n_k}$%
־אלכסונית בלוקים.

\item לכל
$i \in [k]$
המרחב
$V_i$
הוא
$T$%
־שמור.

\item יש העתקות
$T_i \in \endo_{\mbb{F}}\prs{V_i}$
עבורן
$T = \bigoplus_{i \in [k]} T_i$.
\end{enumerate}
\end{exercise}

\begin{solution}
\begin{description}
\item[1 גורר 2:]
נניח כי
$\brs{T}_B$
מטריצה
$\prs{n_1, \ldots, n_k}$%
־אלכסונית בלוקים ויהי
$v \in V_i$.
אז
\begin{align*}
\text{.} \brs{T v}_B &= \brs{T}_B \brs{v}_B \in \spn\set{\brs{T}_B e_j}{j \in [n_1]} \subseteq \spn\set{e_j}{j \in [n_1]}
\end{align*}
מתקיים
\[\rho_B\prs{V_1} = \mbb{F}^{n_1} \times \set{0} \leq \mbb{F}^n\]
כי $\rest{\rho_B}{V_1}$ חד־חד ערכית ומשוויון מימדים.
לכן
\[Tv = \rho_B^{-1}\prs{\brs{Tv}_B} \in \rho_B^{-1}\prs{\spn\set{e_j}{j \in [n_1]}} = V_1\]
ולכן
$V_1$
מרחב
$T$%
־שמור.

\item[2 גורר 3:]
נניח כי כל
$V_i$
הוא
$T$%
־שמור ונגדיר
$T_i = \rest{T}{V_i} \in \endo_{\mbb{F}}\prs{V_i}$.
מתקיים
\[\text{.} T\prs{\sum_{i \in [k]} v_i} = \sum_{i \in [k]} T\prs{v_i} = \sum_{i \in [k]} T_i\prs{v_i} = \bigoplus_{i \in [k]} T_i\prs{v}\]

\item[3 גורר 1:]
עבור
$v \in V_i$
מתקיים
\[T\prs{v} = T_i\prs{v} \in V_i\]
לכן
\[\prs{\brs{T\prs{v}}_B}_{\ell} = 0\]
לכל
$\ell \notin \brs{n_1 + \ldots + n_{i-1} + 1, n_1 + \ldots + n_i}$.
\end{description}
\end{solution}

\begin{remark}
בהמשך הקורס נמצא פירוק של
$V$
לסכום של תת־מרחבים
$T$%
־שמורים
$V_{\lambda}'$
ונמצא עבורם בסיסים שלפיהם קל למצוא תת־מרחבים
$T$%
־שמורים.
נוכל בעזרת כך לתאר בהמשך דרך לחישוב תת־מרחבים
$T$%
־שמורים.
\end{remark}

\section{הטלות}

\subsection{חזרה}

\begin{definition}[הטלה במקביל לתת־מרחב]
יהיו
$V$
מרחב וקטורי מעל
$\mbb{F}$
ו־%
$U,W \leq V$
עבורם
$U \oplus W = V$.
נגדיר את
\emph{ההטלה על
$U$
במקביל ל־%
$W$}
להיות ההעתקה
\begin{align*}
P_U \colon V &\to V \\
\text{.} \hphantom{lala} u+w &\mapsto u
\end{align*}
\end{definition}

\begin{remark}
ההטלה
$P_U$
תלויה ב־%
$W$.
\end{remark}

\begin{example}
יהי
$V = \mbb{R}^2$
ויהיו
\begin{align*}
U &= \spn\prs{e_1} \\
W_1 &= \spn\prs{e_2} \\
W_2 &= \spn\prs{e_1 + e_2}
\end{align*}
ויהיו
$P_1, P_2$
ההטלות על
$U$
במקביל ל־%
$W_1, W_2$
בהתאמה.

מתקיים
\[P_1 \prs{e_1 + e_2} = e_1\]
אבל
\[\text{.} P_2 \prs{e_1 + e_2} = e_1 + e_2\]
\end{example}

\begin{definition}[הטלה]
העתקה
$P \in \endo_{\mbb{F}}\prs{V}$
נקראת הטלה אם קיימים
$U,W \leq V$
עבורם
$P = P_U$.
\end{definition}

\begin{fact}
העתקה
$P \in \endo_{\mbb{F}}\prs{V}$
היא הטלה אם ורק אם
$P^2 = P$.
\end{fact}

\subsection{תרגילים}

\begin{exercise}
תהי
$P \in \endo_{\mbb{F}}\prs{V}$
הטלה.
הראו כי
$\im\prs{P}$
תת־מרחב
$P$%
־שמור וכי
$\rest{P}{\im P} = \id_{\im P}$.
\end{exercise}

\begin{solution}
יהי
$v \in \im P$.
קיים
$u \in V$
עבורו
$v = Pu$.
אז
\[\text{.} Pv = P\prs{P u} = P^2 u = Pu = v \in \im P\]
\end{solution}

\begin{exercise}
יהי
$n \in \mbb{N}$.
מצאו משלים ישר
$W$
של
$\mbb{F}_n\brs{x} \leq \mbb{F}\brs{x}$
ומצאו את ההטלה על
$\mbb{F}_n\brs{x}$
במקביל ל־%
$W$.
\end{exercise}

\begin{solution}
ניקח
\[\text{.} W = \spn\set{x^m}{m > n}\]
אז ההטלה לוקחת פולינום
$p = \sum_{i \in [m]} a_i x^i$
לפולינום
$\sum_{i \in [n]} a_i x^i$.
\end{solution}

\begin{exercise}
יהי
$S \leq M_n\prs{\mbb{F}}$
התת־מרחב של המטריצות הסימטריות.
מצאו הטלה
\[\text{.} P \colon M_n\prs{\mbb{F}} \to S\]
\end{exercise}

\begin{solution}
יהי
$A \leq M_n\prs{\mbb{F}}$
תת־המרחב של המטריצות האנטי־סימטריות.
מטריצה סימטרית ואנטי סימטרית
$X$
מקיימת
\[-X = X^t = X\]
לכן
$X = 0$.
מתקיים
$M_n\prs{\mbb{F}} = S + A$
כי אפשר לכתוב
\[\text{.} X = \frac{1}{2} \prs{X + X^t} + \frac{1}{2} \prs{X - X^t}\]
אז ההטלה על
$S$
במקביל ל־%
$A$
היא
\[\text{.} X \mapsto \frac{1}{2} \prs{X + X^t}\]
\end{solution}

\begin{exercise}
יהי
\[\text{.} U = \spn\set{\pmat{1 \\ 2 \\ 0}, \pmat{1 \\ 0 \\ 1}} \leq \mbb{R}^3\]
מצאו משלים ישר
$W$
של
$U$,
את ההטלה
$P_U$
במקביל ל־%
$W$
ומטריצה מייצגת של
$P$.
\end{exercise}

\begin{solution}
נשלים את
$B \ceq \prs{{\pmat{1 \\ 2 \\ 0}, \pmat{1 \\ 0 \\ 1}}}$
לבסיס
$C = \prs{\pmat{1 \\ 2 \\ 0}, \pmat{1 \\ 0 \\ 1}, \pmat{1 \\ 0 \\ 0}}$
של
$\mbb{R}^3$
ואז
$W \ceq \spn\prs{\pmat{1 \\ 0 \\ 0}}$
משלים ישר של
$U$
כי
$C = B * \prs{\pmat{1\\0\\0}}$
ומתרגיל מהתרגול הקודם.

אז
\begin{align*}
P\pmat{x\\y\\z} &= P\prs{\prs{x-\frac{y}{2}-z}\pmat{1 \\ 0 \\ 0} + \frac{y}{2} \pmat{1 \\ 2 \\ 0} + z \pmat{1 \\ 0 \\ 1}}
\\&=
\prs{x-\frac{y}{2} - z} P\pmat{1\\0\\0} + \frac{y}{2} P\pmat{1\\2\\0} + z P\pmat{1\\0\\1}
\\ \text{.} \hphantom{P\pmat{x\\y\\z}} &= \frac{y}{2} \pmat{1 \\ 2 \\ 0} + z \pmat{1 \\ 0 \\ 1}
\end{align*}

בבסיס
$C$
נקבל
\[\text{.} \brs{P}_C = \pmat{1 & 0 & 0 \\ 0 & 1 & 0 \\ 0 & 0 & 0}\]
\end{solution}

\begin{exercise}
מצאו את כל הערכים העצמיים האפשריים של הטלה.
\end{exercise}

\begin{solution}
יהי
$\lambda$
ערך עצמי של הטלה
$P$
עם וקטור עצמי
$v$.
אז
\[\lambda^2 v = P^2 v = P v = \lambda v\]
לכן
$\lambda \in \set{0,1}$.

אפשר לקחת
$P = \id_V$
או
$P = 0_V$
ולקבל את שתי האופציות האלו.
\end{solution}

\begin{exercise}
תהי
$P \in \endo_{\mbb{F}}\prs{V}$
הטלה.
הראו כי
$P$
הטלה על
$U \leq V$
במקביל ל־%
$W \leq V$
אם ורק אם
$U = \im P$
וגם
$W = \ker P$.
\end{exercise}

\begin{solution}
\begin{itemize}
\item נניח כי
$P$
הטלה על
$U$
במקביל ל־%
$W$.
לכל
$u \in U$
מתקיים
\[P\prs{u} = P\prs{u+0} = u\]
לכן
$U \subseteq \im P$.
להיפך, אם
$v \in \im P$
יש
$u \in U, w \in W$
עבורם
$v = P\prs{u+w} = u \in U$
לכן
$\im P = U$.

יהי
$w \in W$.
מתקיים
\[P\prs{w} = P\prs{0+w} = 0\]
לכן
$W \subseteq \ker\prs{P}$.
להיפך, נניח כי
$v \in \ker P$
ונכתוב
$v = u + w$.
אז
\[0 = Pv = u\]
לכן
$v = w \in W$,
לכן
$\ker P = \im P$.

\item נניח כי
$\ker P = W$
וכי
$\im P = U$.
נניח כי
$v \in U \cap W$.
ראינו כי
$\rest{P}{\im P} = \id_{\im P}$
לכן
$Pv = v$.
אבל גם
$Pv = 0$,
לכן
$v = 0$
ולכן
$U,W$
זרים.
ממשפטי המימדים מתקיים
\begin{align*}
\dim \prs{\im P + \ker P} &=
\dim\prs{\im P} + \dim\prs{\ker P} - \dim\prs{\im P \cap \ker P}
\\&=
\dim\prs{\im P} + \dim\prs{\ker P}
\\ \text{.} \hphantom{\dim \prs{\im P + \ker P}} &= V
\end{align*}
לכן
$V = U \oplus W$.
עבור
$u \in U, w \in W$
נכתוב
$u = P\prs{u'}$.
אז
\begin{align*}
P\prs{u + w} &= P\prs{u} + P\prs{w}
\\&= P\prs{u}
\\&= u
\end{align*}
כאשר השוויון האחרון נכון כי
$\rest{P}{\im P} = \id_{\im P}$.
לכן
$P$
ההטלה על
$\im P$
בניצב ל־%
$\ker P$.
\end{itemize}
\end{solution}

\begin{remark}
בכיוון השני בפתרון הראינו שעבור הטלה
$P$
מתקיים
$V = \im P \oplus \ker P$.
\end{remark}

\begin{exercise}
תהי
$P \in \endo_{\mbb{F}}\prs{V}$
הטלה.
הראו כי
$P$
לכסינה.
\end{exercise}

\begin{solution}
יהי
$B$
בסיס עבור
$\im P$
ויהי
$C$
בסיס עבור
$\ker P$.
ראינו כי
$\rest{P}{\im P} = \id$
וכי
$\rest{P}{\ker P} = 0$
לכן המטריצה המייצגת של
$P$
בבסיס
$B * C$
היא
\begin{align*}
\text{.} \pmat{1 & 0 & & \cdots & & 0 \\ 0 & \ddots & 0 & \cdots & & 0 \\ & & 1 & 0 & \cdots & 0 \\ \vdots & & & 0 & & \vdots \\ & & & & \ddots & \\ 0 & & \cdots & & & 0}
\end{align*}
\end{solution}

\begin{exercise}
תהי
$P \in \endo_{\mbb{F}}\prs{V}$
הטלה ותהי
$T \in \endo_{\mbb{F}}\prs{V}$
כלשהי.
הראו כי
$\ker P$
הוא
$T$%
־שמור אם ורק אם יש
$\hat{T} \colon \ker P \to \ker P$
לינארית המקיימת
\[\text{.} \hat{T} \circ P = P \circ T\]
כלומר, כך שהדיאגרמה
\[\begin{tikzcd}
V \arrow[r, "T"] \arrow[d, swap, "P"] & V \arrow[d, "P"] \\
\im P \arrow[r, swap, "\hat{T}"] & \im P
\end{tikzcd}
\]
מתחלפת.
\end{exercise}

\begin{solution}
\begin{itemize}
\item נניח כי
$W$
הוא
$T$%
־שמור ויהי
$u \in \im P$.
יש
$v \in V$
עבורו
$P\prs{v} = u$,
ואז
$P\prs{u} = P^2\prs{u} = P\prs{v} = u$.
נגדיר
\[\text{.} \hat{T}\prs{u} = P \circ T\prs{u}\]

יהי
$v \in V$,
צריך להראות שמתקיים
\[\text{.} P \circ T \prs{v} = \hat{T} \circ P\prs{v}\]
נכתוב
$v = u+w$
כאשר
$u \in \im P, w \in \ker P$.
אז
\[\hat{T} \circ P \prs{v} = \hat{T} \prs{u} = P \circ T \prs{u}\]
וגם
\[T\prs{w} \in \ker P\]
לכן
\[\text{.} P \circ T \prs{u} = P \circ T \prs{u} + P \circ T \prs{w} = P \circ T \prs{u+w} = P \circ T\prs{v}\]
בסך הכל
\[\text{,} \hat{T} \circ P \prs{v} = P \circ T \prs{u} = P \circ T \prs{v}\]
כנדרש.

\item נניח כי יש העתקה
$\hat{T}$
כמתואר ויהי
$w \in \ker P$.
אז
\[PT\prs{w} = \hat{T}P\prs{w} = \hat{T}\prs{0} = 0\]
ולכן
$T\prs{w} \in \ker P$.
נקבל
$T\prs{\ker P} \subseteq \ker P$.
\end{itemize}
\end{solution}

\end{document}