\documentclass[a4paper,10pt,oneside,openany]{article}

\usepackage[lang=hebrew]{maths}
\usepackage{hebrewdoc}
\usepackage{stylish}
\usepackage{lipsum}
\let\bs\blacksquare
\usepackage{ytableau}

\usepackage{graphicx, txfonts}

\newcommand{\heart}{\ensuremath\varheartsuit}

\title{
אלגברה ב' (104168) \textenglish{---} אביב 2020-2021
\\
תרגול 8 \textenglish{---} מכפלות פנימיות, ניצבות ותהליך גרם־שמידט
}
\author{אלעד צורני}
\date{\today}

\begin{document}
\maketitle

\section{מכפלות פנימיות}

\subsection{חזרה}

נרצה הרבה פעמים לדון במושג של זווית בין וקטורים. לשם כך לא מספיקה ההגדרה של מרחב וקטורי, וצריך להסתכל על מבנה נוסף של מכפלה פנימית על מרחב וקטורי.

\begin{definition}[מכפלה פנימית]
יהי
$V$
מרחב וקטורי מעל
$\mbb{F} \in \set{\mbb{R}, \mbb{C}}$.
\emph{מכפלה פנימית על
$V$}
היא העתקה
\begin{align*}
\trs{\cdot,\cdot} \colon V \times V &\to \mbb{F}
\end{align*}
המקיימת את התכונות הבאות.
\begin{description}
\item[לינאריות ברכיב הראשון:]
לכל
$u,v,w \in V$
ולכל
$\alpha \in \mbb{F}$
מתקיים
\begin{align*}
\text{.} \trs{\alpha u + v, w} &= \alpha\trs{u,w} + \trs{v,w}
\end{align*}
\item[הרמיטיות:]
לכל
$u,v \in V$
מתקיים
\[\text{.} \trs{u,v} = \overline{\trs{v,u}}\]
\item[מוגדרות חיובית:]
לכל
$v \in V \setminus \set{0}$
מתקיים
\[\text{.} \trs{v,v} > 0\]
\end{description}
\end{definition}

נרצה פעמים לדון במושג של אורך של וקטורים במרחב וקטורי. לשם כך צריך מבנה נוסף על המרחב הוקטורי, בשם נורמה.

נניח בדיון על מכפלות פנימיות כי אנו עובדים מעל
$\mbb{F} \in \set{\mbb{R}, \mbb{C}}$.

\subsection{תרגילים}

\begin{exercise}
יהי
$V$
מרחב מכפלה פנימית.
\begin{enumerate}
\item הוכיחו כי אם
$\trs{v,w} = 0$
לכל
$w \in V$
אז
$v = 0$.
\item הוכיחו כי אם
$\trs{v,w} = \trs{u,w}$
לכל
$w \in V$
אז
$v=u$.
\item הוכיחו כי אם
$T,S \in \endo_{\mbb{F}}\prs{V}$
ו־%
$\trs{Tu,v} = \trs{Su,v}$
לכל
$u,v \in V$
אז
$T = S$.
\end{enumerate}
\end{exercise}

\begin{solution}
\begin{enumerate}
\item ניקח
$w = v$
ונקבל
\[\text{.} \trs{v,v} = 0\]
ולכן
$v = 0$.
\item נעביר אגף ונקבל
\[\trs{v-u, w} = 0\]
לכל
$w \in V$.
אז מהסעיף הקודם
$v-u = 0$
ולכן
$v = u$.
\item נעביר אגף ונקבל
\[\text{.} \trs{\prs{T-S}\prs{u},v} = 0\]
לכל
$u,v \in V$.
אז עבור כל
$u \in V$
מתקיים
$\prs{T-S}\prs{u} = 0$,
ולכן
$T\prs{u} = S\prs{u}$.
\end{enumerate}
\end{solution}

\section{ניצבות}

\subsection{חזרה}

\begin{definition}[וקטורים ניצבים]
יהי
$V$
מרחב מכפלה פנימית ויהיו
$u,v \in V$.
נגיד ש־%
\emph{$u,v$
ניצבים}
ונכתוב
$u \perp v$
אם
$\trs{u,v} = 0$.
\end{definition}

\begin{definition}[מרחב ניצב]
יהי
$V$
מרחב מכפלה פנימית ותהי
$S \subseteq V$
תת־קבוצה.
נגדיר את
\emph{המרחב הניצב ל־%
$S$}
על ידי
\[\text{.} S^\perp \ceq \set{v \in V}{\forall u \in S \colon v \perp u}\]
\end{definition}

\begin{remark}
$S^\perp$
תמיד מרחב וקטורי. אם
$v,w \in u$
ו־%
$\alpha \in \mbb{F}$,
גם
$\alpha v + w \perp u$
כי
\[\text{.} \trs{\alpha v + w, u} = \alpha \trs{v,u} + \trs{w,u} = 0\]
\end{remark}

\begin{proposition}
יהי
$V$
מרחב מכפלה פנימית ויהי
$W \leq V$.
אז
\[\text{.} \dim\prs{V} = \dim\prs{W} + \dim \prs{W^\perp}\]
\end{proposition}

\subsection{תרגילים}

\begin{exercise}
מצאו את
$W^\perp$
עבור
\[\text{.} W \ceq \spn\prs{e_1 + e_2} \leq \mbb{R}^2\]
\end{exercise}

\begin{solution}
מתקיים
$v \in W^\perp$
אם ורק אם
$v \perp e_1 + e_2$
אם ורק אם
$v_1 + v_2 = 0$
אם ורק אם
$v_1 = - v_2$.
לכן
\[\text{.} W^\perp = \spn\prs{e_1 - e_2}\]
\end{solution}

\begin{exercise}
יהי
$V$
מרחב מכפלה פנימית ויהי
$W \leq V$.
הראו כי
$\prs{W^\perp}^\perp = W$.
\end{exercise}

\begin{solution}
יהי
$w \in W$
ויהי
$v \in W^\perp$.
מהגדרת
$W^\perp$
מתקיים
$\trs{w,v} = 0$.
לכן
$w \in \prs{W^\perp}^\perp$
ולכן
$W \subseteq \prs{W^\perp}^\perp$.

מתקיים
\begin{align*}
\dim\prs{\prs{W^\perp}^\perp} &= \dim\prs{V} - \dim\prs{W^\perp} = \dim\prs{W}
\end{align*}
ולכן בעצם יש שוויון.
\end{solution}

\begin{exercise}
\begin{enumerate}
\item הראו שאם
$S \subseteq T$
תת־קבוצות במרחב מכפלה פנימית אז
\[\text{.} T^\perp \subseteq S^\perp\]
\item הסיקו כי
\[\text{.} \prs{S^\perp}^\perp = \spn\prs{S}\]
\end{enumerate}
\end{exercise}

\begin{solution}
\begin{enumerate}
\item יהי
$v \in T^\perp$.
אז
$v \perp u$
לכל
$u \in T$.
בפרט
$v \perp u$
לכל
$u \in S$,
ולכן
$v \in S^\perp$.

\item
מתקיים
$S \subseteq \spn\prs{S}$
ולכן
\begin{align*}
\spn\prs{S}^\perp \subseteq S^\perp
\end{align*}
וגם
\begin{align*}
\text{.} \prs{S^\perp}^\perp \subseteq \prs{\spn\prs{S}^\perp}^perp
\end{align*}
אבל,
$\spn\prs{S}$
מרחב וקטורי ולכן
\[\text{.} \prs{\spn\prs{S}^\perp}^perp = \spn\prs{S}\]
ונקבל כי
\[\text{.} \prs{S^\perp}^\perp \subseteq \spn\prs{S}\]
כעת,
$S \subseteq \prs{S^\perp}^\perp$
כמו בתרגיל הקודם, ולכן
$\prs{S^\perp}^\perp$
מרחב וקטורי שמכיל את
$S$
ומוכל ב־%
$\spn\prs{S}$.
ממינימליות
$\spn\prs{S}$
נקבל כי
\[\text{,} \prs{S^\perp}^\perp = \spn\prs{S}\]
כנדרש.
\end{enumerate}
\end{solution}

\section{הטלות אורתוגונליות ותהליך גרם־שמידט}

\subsection{חזרה}

\begin{definition}[בסיס אורתוגונלי]
יהי
$V$
מרחב מכפלה פנימית.
\emph{בסיס אורתוגונלי של
$V$}
הוא בסיס
$\prs{v_1, \ldots, v_n}$
עבורו
\[\trs{v_i, v_j} = 0\]
לכל
$i,j \in [n]$
שונים.

אם גם
$\norm{v_i} = 1$
לכל
$i \in [n]$,
או באופן שקול אם
$\trs{v_i, v_j} = \delta_{i,j}$,
נאמר שזה
\emph{בסיס אורתונורמלי}.
\end{definition}

\begin{definition}[הטלה אורתוגונלית]
יהי
$V$
מרחב מכפלה פנימית ויהי
$W \leq V$
עם בסיס אורתונורמלי
$\prs{w_1, \ldots, w_m}$.
\emph{ההטלה האורתוגונלית על
$W$}
היא
\begin{align*}
P_W \colon V &\to V \\
\text{.} \hphantom{lalala} v &\mapsto \sum_{i \in [m]} \trs{v, w} w
\end{align*}
\end{definition}

\begin{theorem}[תהליך גרם־שמידט]
יהי
$V$
מרחב מכפלה פנימית ויהי
$\prs{v_1, \ldots, v_n}$
בסיס של
$V$.
קיים בסיס
$\prs{w_1, \ldots, w_n}$
אורתונורמלי
של
$V$
עבורו
$\spn\prs{w_1, \ldots, w_i} = \spn\prs{v_1, \ldots, v_i}$
לכל
$i \in [n]$.
\end{theorem}

\subsection{תרגילים}

\begin{exercise}
יהי
\[\text{.} W = \spn\set{\pmat{3\\4\\2}} \leq \mbb{R}^3\]

\begin{enumerate}
\item מצאו את 
$\brs{P_W}_E$.
\item מצאו את
$W^\perp$.
\item תהי
$p = \pmat{5\\5\\5}$.
מצאו את המרחק של
$p$
מ־%
$W$
ונקודה
$q \in W$
עבורה
$\norm{p-q}$
שווה למרחק זה.
\end{enumerate}
\end{exercise}

\begin{solution}
\begin{enumerate}
\item כדי לחשב את
$P_W$
נחפש בסיס אורתונורמלי עבור
$W$.
ננרמל את
$\pmat{3\\4\\0}$
ונקבל
$w = \frac{1}{5} \pmat{3\\4\\0}$
עבורו
$\prs{w}$
בסיס אורתונורמלי של
$W$.
אז
\[\text{.} P_W\prs{v} = \trs{v,w} w\]
נקבל כי
\begin{align*}
P_W\prs{e_1} &= \frac{1}{25} \trs{e_1, \pmat{3\\4\\0}} \pmat{3\\4\\0} = \frac{3}{25} \pmat{3\\4\\0} \\
P_W\prs{e_2} &= \frac{1}{25} \trs{e_2, \pmat{3\\4\\0}} \pmat{3\\4\\0} = \frac{4}{25} \pmat{3\\4\\0} \\
P_W\prs{e_3} &= \frac{1}{25} \trs{e_3, \pmat{3\\4\\0}} \pmat{3\\4\\0} = 0
\end{align*}
ולכן
\begin{align*}
\text{.} \brs{P_W}_E &= \frac{1}{25} \pmat{9 & 12 & 0 \\ 12 & 16 & 0 \\ 0 & 0 & 0}
\end{align*}

\item כדי למצוא את
$W^\perp$
נשלים את
$\pmat{3\\4\\0}$
לבסיס של
$\mbb{R}^3$
ונבצע את תהליך גרם־שמידט כדי לקבל בסיס אורתונורמלי.
יהי
$B = \prs{\pmat{3\\4\\0}, e_2, e_3}$
בסיס ל־%
$\mbb{R}^3$.
ננרמל את הוקטור הראשון ונקבל
$w_1 \ceq w$.
נגדיר
\begin{align*}
\hat{w}_2 &\ceq e_2 - \trs{e_2, w_1} w_1
\\&= e_2 - \frac{1}{25} \trs{e_2, \pmat{3\\4\\0}} \pmat{3\\4\\0}
\\&= e_2 - \frac{4}{25} \pmat{3\\4\\0}
\\&= \pmat{\frac{12}{25} \\ \frac{9}{25} \\ 0}
\end{align*}
ואז
\[\norm{\hat{w}_2} = \frac{1}{25} \sqrt{12^2 + 9^2} = \frac{15}{25} = \frac{3}{5}\]
ונקבל
\[\text{.} w_2 = \frac{\hat{w}_2}{\norm{\hat{w}_2}} = \frac{5}{3} \cdot \frac{1}{25} \pmat{12\\9\\0} = \frac{1}{15} \pmat{12\\9\\0}\]
\end{enumerate}
נגדיר
\begin{align*}
\hat{w}_3 &\ceq e_3 - \cancelto{0}{\trs{e_3, w_2}} w_2 - \cancelto{0}{\trs{e_3, w_1}}w_1
\\&=
e_3
\end{align*}
וגם
\[ \text{.} w_3 = \frac{\hat{w}_3}{\norm{\hat{w}_3}} = \frac{e_3}{1} = e_3\]
אז
\[\text{.} W^\perp = \spn\prs{w_2, w_3} = \spn\prs{\hat{w}_2, \hat{w}_3} = \spn\prs{\frac{1}{15}\pmat{12\\9\\0}, e_3}\]
\item
נמצא קודם נקודה
$q \in W$
כמתואר.
ידוע כי נקודה זאת היא
$P_W\prs{p}$.
מתקיים
\[\text{.} P_W\prs{p} = \frac{1}{25} \pmat{9 & 6 & 0 \\ 6 & 4 & 0 \\ 0 & 0 & 0} \pmat{5\\5\\5} = \frac{1}{25} \pmat{75 \\ 50 \\ 0} = \pmat{3\\2\\0}\]

אז המרחק הוא
\begin{align*}
\norm{\pmat{5\\5\\5} - \pmat{3\\2\\0}} &= \sqrt{\prs{5-3}^2 + \prs{5-2}^2 + 5^2}
\\&= \sqrt{4 + 9 + 25}
\\ \text{.} \hphantom{\norm{\pmat{5\\5\\5} - \pmat{3\\2\\0}}} &= \sqrt{38}
\end{align*}
\end{solution}

\section{האופרטור הצמוד ומשפט ריס}

\subsection{חזרה}

\begin{definition}[האופרטור הצמוד]
יהי
$V$
מרחב מכפלה פנימית ותהי
$T \in \endo_{\mbb{F}}\prs{V}$.
\emph{האופרטור הצמוד של
$T$}
שנסמנו
$T^*$
הוא האופרטור היחיד
$T^* \in \endo_{\mbb{F}}\prs{V}$
עבורו
\[\trs{Tu,v} = \trs{u,T^*v}\]
לכל
$u,v \in V$.
\end{definition}

\begin{theorem}[ריס]
לכל
$f \in V^*$
קיים
$v \in V$
עבורו
$f\prs{u} = \trs{u,v}$
לכל
$u \in V$.
\end{theorem}

\subsection{תרגילים}

\begin{exercise}
יהי
$V = M_2\prs{\mbb{R}}$.
עם המכפלה הפנימית
\[\text{.} \trs{A,B} = \tr\prs{B^t A}\]
נגדיר
\begin{align*}
\text{.} T\prs{\pmat{a & b \\ c & c}} = \pmat{3d & 2c \\ -b & 4a}
\end{align*}
חשבו את
$T^*$
בשתי דרכים שונות.
\end{exercise}

\begin{solution}
\begin{description}
\item[דרך 1:]
נמצא את התנאים על מקדמי
$T^*$
בעזרת המכפלה הפנימית.
מתקיים
\begin{align*}
\trs{T\pmat{a & b \\ c & d}, \pmat{e & f \\ g & h}} &= \trs{\pmat{3d & 2c \\ -b & 4a}, \pmat{e & f \\ g & h}}
\\&= 3de + 2cf - bg + 4ah
\\&= \trs{\pmat{a & b \\ c & d}, \pmat{4h & -g \\ 2f & 3e}}
\end{align*}
ולכן
\[\text{.} T^*\pmat{e & f \\ g & h} = \pmat{4h & -g \\ 2f & 3e}\]
\item[דרך 2:]
נשים לב כי
\[E = \prs{\pmat{1 & 0 \\ 0 & 0}, \pmat{0 & 1 \\ 0 & 0}, \pmat{0 & 0 \\ 1 & 0}, \pmat{0 & 0 \\ 0 & 1}}\]
בסיס אורתונורמלי ל־%
$M_2\prs{\mbb{R}}$.
נחשב:
\[\brs{T}_E = \pmat{0 & 0 & 0 & 3 \\ 0 & 0 & 2 & 0 \\ 0 & -1 & 0 & 0 \\ 4 & 0 & 0 & 0}\]
כיוון ש־%
$E$
אורתונורמלי, מתקיים
\[\brs{T^*}_E = \brs{T}_E^* \ceq \overline{\brs{T}_E}^t\]
ולכן
\begin{align*}
\text{.} \brs{T^*}_E = \brs{T}_E^* = \pmat{0 & 0 & 0 & 4 \\ 0 & 0 & -1 & 0 \\ 0 & 2 & 0 & 0 \\ 3 & 0 & 0 & 0}
\end{align*}
לכן
\[\text{.} T^*\pmat{e & f \\ g & h} = \pmat{4h & -g \\ 2f & 3e}\]
\end{description}
\end{solution}

\begin{exercise}
\begin{enumerate}
\item הוכיחו כי לכל
$n \in \mbb{N}$
קיים
$C > 0$
כך שלכל
$p \in \mbb{R}_n\brs{x}$
מתקיים
\[\text{.} \abs{p\prs{0}} \leq C \prs{\int_{-1}^1 p\prs{x}^2 \diff x}^{\frac{1}{2}}\]

\item חשבו את
$C$
המינימלי עבור
$n=2$.
\end{enumerate}
\end{exercise}

\begin{solution}
\begin{enumerate}
\item נשים לב כי
\[\trs{f,g} = \int_{-1}^1 f\prs{x} g\prs{x} \diff x\]
מכפלה פנימית.
אז
\begin{align*}
\text{.} \prs{\int_{-1}^1 p\prs{x}^2 \diff x}^{\frac{1}{2}} = \pmat{\trs{p,p}}^{\frac{1}{2}} = \norm{p}
\end{align*}
כלומר, עלינו להוכיח
\[\text{.} \abs{p\prs{0}} \leq C\norm{p}\]

ההצבה
\begin{align*}
\ev_0 \colon \mbb{R}_n\brs{x} &\to \mbb{R} \\
f &\mapsto f\prs{0}
\end{align*}
היא פונקציונל לינארי, ולכן ממשפט ריס יש
$g \in \mbb{R}_n\brs{x}$
עבורו
\[\text{.} p\prs{0} = \ev_0\prs{p} = \trs{p,g}\]
עכשיו, מקושי־שוורץ
\[\text{.} \abs{p\prs{0}} = \abs{\trs{p,g}} \leq \norm{p} \norm{g}\]
לכן ניקח
$C = \norm{g}$.

\item
נסמן
$g\prs{x} = ax^2 + bx + c$.
נשים לב כי
כאשר
$p = g$
יש שוויון בקושי־שוורץ ואז
\[\text{.} \abs{p\prs{0}} = \norm{p}\norm{g} \leq C\norm{p}\]
גורר
$C \geq \norm{g}$.
ראינו כי
$C = \norm{g}$
מקיים את הנדרש, ולכן נותר למצוא את
$\norm{g}$.
כעת
\begin{align*}
1 &= 1\prs{0} = \trs{g\prs{x},1} = \int_{-1}^1 g\prs{x} \diff x = \left. \frac{ax^3}{3} + \frac{bx^2}{2} + cx \right|_{x=-1}^1 = \frac{2a}{3} + 2c \\
0 &= x\prs{0} = \trs{g\prs{x}, x} = \int_{-1}^1 g\prs{x} x \diff x = \left. \frac{ax^4}{4} + \frac{bx^3}{3} + \frac{cx^2}{2} \right|_{x=-1}^1 = \frac{2b}{3} \\
\text{.} 0 &= x^2\prs{0} = \trs{g\prs{x}, x^2} = \int_{-1}^1 g\prs{x} x^2 \diff x = \left. \frac{ax^5}{5} + \frac{bx^4}{4} + \frac{cx^3}{3} \right|_{x=-1}^1 = \frac{2a}{5} + \frac{2c}{3}
\end{align*}
מהמשוואה השנייה נקבל
$b = 0$.
מהמשוואה הראשונה פחות
$3$
פעמים השנייה נקבל
\[1 = \frac{2a}{3} - 3 \cdot \frac{2a}{5} + 0 = \frac{10a - 18a}{15}\]
ולכן
$a = -\frac{15}{8}$.
אז מהמשוואה השלישית נקבל
\[c = -\frac{3a}{5} = -\frac{9}{8}\]
ולכן
\[\text{.} g\prs{x} = -\frac{15}{8} x^2 - \frac{9}{8}\]

אז
\[\text{.} C = \norm{g} = \int_{-1}^1 -\frac{15}{8} x^2 - \frac{9}{8} \diff x = -\frac{7}{2}\]
\end{enumerate}
\end{solution}

\end{document}