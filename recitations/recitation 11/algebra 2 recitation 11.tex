\documentclass[a4paper,10pt,oneside,openany]{article}

\usepackage[lang=hebrew]{maths}
\usepackage{hebrewdoc}
\usepackage{stylish}
\usepackage{lipsum}
\let\bs\blacksquare
\usepackage{ytableau}

\usepackage{graphicx, txfonts}

\newcommand{\heart}{\ensuremath\varheartsuit}

\title{
אלגברה ב' (104168) \textenglish{---} אביב 2020-2021
\\
תרגול 11 \textenglish{---}
תבניות בילינאריות
}
\author{אלעד צורני}
\date{\today}

\begin{document}
\maketitle

\section{תבניות בילינאריות}

\subsection{חזרה}

\begin{definition}[תבנית בילינארית]
\emph{תבנית בילינארית}
על מרחב וקטורי
$V$
מעל
$\mbb{F} \in \set{\mbb{R}, \mbb{C}}$
היא העתקה
\[\trs{\cdot,\cdot} \colon V \times V \to \mbb{F}\]
לינארית בשני הרכיבים.
\end{definition}

\begin{remark}
מעל
$\mbb{C}$
בדרך כלל עוסקים בתבניות ססקווילינאריות במקום תבניות בילינאריות, שנדבר עליהן בהמשך.
\end{remark}

\begin{definition}[מטריצות חופפות]
מטריצות
$A,B \in M_n\prs{\mbb{F}}$
נקראות
\emph{חופפות}
אם קיימת
$P \in M_n\prs{\mbb{F}}$
הפיכה
עבורה
$B = P^t A P$.
\end{definition}

\begin{remark}
חפיפת מטריצות היא יחס שקילות.
אם
$B = P^t A P$
וגם
$C = Q^t B Q$
אז
\[\text{.} C = Q^t P^t A P Q = \prs{PQ}^t A \prs{PQ}\]
\end{remark}

\begin{definition}[מטריצה מייצגת לתבנית בילינארית]
עבור העתקה בילינארית
$f \colon V \times V \to \mbb{F}$
ובסיס
$B = \prs{v_i}_{i \in [n]}$
של
$V$
נסמן ב־%
$\brs{f}_B$
את המטריצה עם
$\prs{\brs{f}_B}_{i,j} = f\prs{v_i, v_j}$.
\end{definition}

\begin{remark}
מטריצות
$A,B \in M_n\prs{\mbb{R}}$
חופפות אם ורק אם הן מייצגות את אותה תבנית בילינארית בבסיסים שונים.
\end{remark}

\begin{remark}
ניתן לשחזר את התבנית הבילינארית מהמטריצה המייצגת. עבור
$g$
בילינארית ועבור בסיס
$C$
מתקיים
\[\text{.} g\prs{u,v} = \brs{u}_C^t \brs{g}_C \brs{v}_C\]
למעשה
$g \mapsto \brs{g}_C$
איזומורפיזם בין מרחב התבניות הבילינאריות והמרחב
$M_n\prs{\mbb{R}}$.
\end{remark}

\subsection{תרגילים}

\begin{exercise}
תהיינה
\begin{align*}
f \colon \mbb{R}^2 \times \mbb{R}^2 &\to \mbb{R} \\
\prs{\pmat{x_1 \\ x_2}, \pmat{y_1 \\ y_2}} &\mapsto x_1 y_2 - x_2 y_1 \\
g \colon \mbb{R}^2 \times \mbb{R}^2 &\to \mbb{R} \\
\text{.} \prs{\pmat{x_1 \\ x_2}, \pmat{y_1 \\ y_2}} &\mapsto \prs{x_1 + y_1}^2 - x_2 y_2
\end{align*}
קבעו האם
$f,g$
בילינאריות.
\end{exercise}

\begin{solution}
\begin{itemize}
\item מתקיים
\[f\prs{y,x} = y_1 x_2 - y_2 x_1 = -f\prs{x,y}\]
ולכן מספיק לבדוק לינאריות ברכיב הראשון.
אכן
\[\text{.} f\prs{x + \alpha x', y} = \prs{x_1 + \alpha x_1'} y_2 - \prs{x_2 + \alpha x_2'} y_1 = x_1 y_2 - x_2 y_1 + \alpha\prs{x_1' y_2 - x_2' y_1} = f\prs{x,y} + \alpha f\prs{x',y}\]

\item $g$
אינה תבנית בילינארית.
מתקיים
\[g\prs{\pmat{2\\2}, \pmat{0\\1}} = 2^2 - 2 = 2\]
אבל
\[\text{.} 2g\prs{\pmat{1\\1}, \pmat{0\\1}} = 2\prs{1^1 - 1} = 0\]
\end{itemize}
\end{solution}

\begin{exercise}
תהיינה
$A,B,P \in M_n\prs{\mbb{R}}$
כאשר
$P$
הפיכה ומתקיים
$B = P^t A P$.

הוכיחו או הפריכו את הטענות הבאות.
\begin{enumerate}
\item $\det\prs{A} = \det\prs{B}$.
\item ל־%
$A,B$
יש אותם ערכים עצמיים.
\item $\rank\prs{A} = \rank\prs{B}$.
\item $A$
הפיכה אם ורק אם
$B$
הפיכה, וכאשר זה מתקיים
$A^{-1}, B^{-1}$
חופפות.
\end{enumerate}
\end{exercise}

\begin{solution}
\begin{enumerate}
\item מתקיים
\[\det\prs{B} = \det\prs{P^t A P} = \det\prs{P^t} \det\prs{A} \det\prs{P}\]
ולכן יש שוויון אם ורק אם
$\det\prs{P^t} \det\prs{P} = 1$.
אבל,
$\det\prs{P^t} = \det\prs{P}$
ולכן זה מתקיים אם ורק אם
$\det\prs{P} \in \set{\pm 1}$.
זה לא חייב להיות המצב. למשל, נוכל לקחת
$A = I_n, B = 4I_n, P = 2I_n$.

\item הדוגמא הקודמת מראה שהערכים העצמיים לא נשמרים.

\item הדרגה נשמרת כי
$P^t, P$
הפיכות.

\item נניח כי
$A$
הפיכה. אז
$B$
הפיכה כי הדרגות שלהן שוות.
נשים לב כי
\[B^{-1} = \prs{P^t A P}^{-1} = P^{-1} A^{-1} \prs{P^t}^{-1}\]
ועבור
$\tilde{P} = \prs{P^t}^{-1}$
נקבל
$B^{-1} = \tilde{P}^t A^{-1} \tilde{P}$.
\end{enumerate}
\end{solution}

\section{חוק האינרציה של סילבסטר}

\subsection{חזרה}

\begin{theorem}[סילבסטר]
תהי
$A \in M_n\prs{\mbb{R}}$
סימטרית.
$A$
חופפת למטריצה יחידה מהצורה
$\pmat{I_{\prs{n_+}} & & \\ & - I_{\prs{n_{-}}} & \\ & & 0_{\prs{n_0}}}$.
מטריצה זאת נקראת
\emph{צורת סילבסטר הקנונית של
$A$}.
$n_+$
ו־%
$n_-$
נקראים
\emph{אינדקס האינרציה החיובי והשלילי}
של
$A$,
בהתאמה.
ההפרש
$n_+ - n_-$
נקרא
\emph{הסיגנטורה}
של
$A$.
\end{theorem}

\begin{remark}
משפט סילבסטר אומר באופן שקול שלכל תבנית בילינארית סימטרית
$g$
קיים בסיס
$C$
עבורו
$\brs{g}_C$
בצורת סילבסטר.
\end{remark}

\subsection{תרגילים}

\begin{exercise}
תהי
$A \in M_n\prs{\mbb{R}}$
סימטרית.
\begin{enumerate}
\item הוכיחו כי
$A, A^3$
חופפות.
\item הוכיחו כי אם
$A$
הפיכה, היא חופפת ל־%
$A^{-1}$.
\item תהי
$B \in M_n\prs{\mbb{R}}$
ונניח כי
\begin{align*}
p_A\prs{x} &= x^2 - 2 \\
\text{.} p_B\prs{x} &= x^2 + 2x - 3
\end{align*}
האם
$A,B$
בהכרח חופפות?
\end{enumerate}
\end{exercise}

\begin{solution}
\begin{enumerate}
\item
ל־%
$A,A^3$
יש אותם אינדקסיי אינרציה כי הערכים העצמיים של
$A^3$
הם
$\lambda^3$
עבור
$\lambda$
ערך עצמי של
$A$.
לכן ל־%
$A,A^3$
אותה צורת סילבסטר, ולכן הן חופפות.

\item כמו מקודם, כאשר הערכים העצמיים של
$A^{-1}$
הם
$\frac{1}{\lambda}$
עבור
$\lambda$
ערך עצמי של
$A$.

\item נמצא שורשים של הפולינומים.
\begin{align*}
x^2 - 2 &= \prs{x+\sqrt{2}}\prs{x-\sqrt{2}} \\
x^2 + 2x - 3 &= x^2 + 3x - \prs{x+3} = \prs{x-1}\prs{x+3}
\end{align*}
לכן לכל אחד מהפולינומים ערך עצמי אחד חיובי ואחד שלילי. לכן צורות סילבסטר של
$A,B$
שתיהן
$\pmat{1 & 0 \\ 0 & -1}$
ולכן הן חופפות.
\end{enumerate}
\end{solution}

\begin{exercise}
מצאו את צורת סילבסטר של
\[\text{.} A = \pmat{0 & 1 & 1 & \cdots & 1 \\ 1 & 0 & -1 & \cdots & -1 \\ 1 & -1 & 0 & \cdots & -1 \\ \vdots & \vdots & \vdots & \ddots & \vdots \\ 1 & -1 & -1 & \cdots & 0} \in M_n\prs{\mbb{R}}\]
\end{exercise}

\begin{solution}
כדי למצוא את צורת סילבסטר, נחפש ערכים עצמיים.
נשים לב כי
$1$
ערך עצמי מריבוי
$n-1$.
אז הערך העצמי הנוסף הוא
$\tr\prs{A} -\prs{n-1}\cdot 1 = -n+1$.
אם
$n = 1$
מתקיים
$A = \prs{0}$.
אחרת צורת סילבסטר היא
$\pmat{I_{n-1} & \\ & -1}$.
\end{solution}

\begin{algorithm}
כדי למצוא בסיס
$C$
עבורו
$\brs{g}_C$
בצורת סילבסטר נבצע את השלבים הבאים.

\begin{enumerate}
\item נמצא בסיס
$\tilde{C} = \prs{v_1, \ldots, v_n}$
של
$V$
עבורו
$\brs{g}_{\tilde{C}}$
אלכסונית, בעזרת לכסון אורתוגונלי.

\item
נגדיר
\[\text{.} u_i = \fcases{\frac{v_i}{\sqrt{\abs{\lambda_i}}} & \lambda_i \neq 0 \\ v_i & \lambda_i = 0}\]

\item על ידי בחירת סדר מתאים של ה־%
$u_i$
נקבל בסיס
$C$
המקיים את הנדרש.
\end{enumerate}
\end{algorithm}

\begin{exercise}
תהי
$A = \pmat{2 & 1 & 1 \\ 1 & 2 & 1 \\ 1 & 1 & 2}$.
מצאו מטריצה
$P \in M_3\prs{\mbb{R}}$
עבורה
$P^t A P$
בצורת סילבסטר.
\end{exercise}

\begin{solution}
נתחיל במציאת ערכים עצמיים של
$A$.
נשים לב כי
$1$
ערך עצמי של
$A$
מריבוי
$2$.
הערך העצמי הנוסף הוא
$\tr\prs{A} - 2\cdot 1 = 4$.
בסיס מתאים הוא
$B = \prs{\pmat{1\\1\\1}, \pmat{1\\0\\-1}, \pmat{0\\1\\-1}}$.

נבצע את תהליך גרם־שמידט על כל אחד מהמרחבים העצמיים, כדי לקבל בסיס אורתונורמלי
\[\text{.} \tilde{C} \ceq \prs{v_1, v_2, v_3} = \prs{\pmat{1/\sqrt{3} \\ 1/\sqrt{3} \\ 1/\sqrt{3}}, \pmat{1/\sqrt{2} \\ 0 \\ -1/\sqrt{2}}, \pmat{-\sqrt{6}/6 \\ \sqrt{6}/3 \\ -\sqrt{6}/6}}\]
נחלק את הוקטורים העצמיים ב־%
$\sqrt{\lambda_i}$
ונקבל בסיס
\[\text{.} C \ceq \prs{\pmat{1/2\sqrt{3}\\1/2\sqrt{3}\\1/2\sqrt{3}}, \pmat{1/\sqrt{2} \\ 0 \\ -1/\sqrt{2}}, \pmat{-\sqrt{6}/6 \\ \sqrt{6}/3 \\ -\sqrt{6}/6}}\]
תהי
\[P = \pmat{1/2\sqrt{3} & 1/\sqrt{2} & -\sqrt{6}/6 \\ 1/2\sqrt{3} & 0 & \sqrt{6}/3 \\ 1/2\sqrt{3} & -1/\sqrt{2} & -\sqrt{6}/6}\]
ונקבל כי
\[P^t A P = I_3\]
בצורת סילבסטר.
\end{solution}

\begin{remark}
לפעמים חישוב הערכים העצמיים יכול להיות מסובך. נוכל בעצם למצוא את צורת סילבסטר ואת הבסיס המתאים לה בלי מציאת המרחבים העצמיים.
לשם כך, נשים לב כי אם
$E$
מטריצה אלמנטרית, הכפל
$A \mapsto A E^t$
נקבע על פי אותן פעולות על עמודות
$A$
כמו פעולות
$E$
על השורות. נוכל אם כן לדרג את
$A$
לפי שורה ועמודה במקביל, כאשר בכל שלב נבצע את אותה פעולה על השורות ואז על העמודות (או להיפך). נקבל מטריצה
$P$
כמכפלת המטריצות האלמנטריות, שעבורה
$PAP^t$
תהיה בצורת סילבסטר.
\end{remark}

\begin{exercise}
מצאו
$P \in M_4\prs{\mbb{R}}$
הפיכה עבורה
$P A P^t$
בצורת סילבסטר, כאשר
\[\text{.} A = \pmat{1 & 2 & 3 & 4 \\ 2 & 5 & 6 & 7 \\ 3 & 6 & 8 & 9 \\ 4 & 7 & 9 & 10}\]
\end{exercise}

\begin{solution}
נבצע דירוג שורה ועמודה במקביל.
\begin{align*}
A &\to \pmat{1 & 0 & 0 & 0 \\ 2 & 1 & 0 & -1 \\ 3 & 0 & -1 & -3 \\ 4 & -1 & -3 & -6} \to \pmat{1 & 0 & 0 & 0 \\ 0 & 1 & 0 & -1 \\ 0 & 0 & -1 & -3 \\ 0 & -1 & -3 & -6}
\\&\to
\pmat{1 & 0 & 0 & 0 \\ 0 & 1 & 0 & 0 \\ 0 & 0 & -1 & -3 \\ 0 & -1 & -3 & -7} \to
\pmat{1 & 0 & 0 & 0 \\ 0 & 1 & 0 & 0 \\ 0 & 0 & -1 & -3 \\ 0 & 0 & -3 & -7}
\\&\to
\pmat{1 & 0 & 0 & 0 \\ 0 & 1 & 0 & 0 \\ 0 & 0 & -1 & 0 \\ 0 & 0 & -3 & 2}
\to
\pmat{1 & 0 & 0 & 0 \\ 0 & 1 & 0 & 0 \\ 0 & 0 & -1 & 0 \\ 0 & 0 & 0 & 2}
\\&\to
\pmat{1 & 0 & 0 & 0 \\ 0 & 1 & 0 & 0 \\ 0 & 0 & 2 & 0 \\ 0 & 0 & 0 & -1}
\\&\to
\pmat{1 & 0 & 0 & 0 \\ 0 & 1 & 0 & 0 \\ 0 & 0 & 1 & 0 \\ 0 & 0 & 0 & -1}
\end{align*}
עבור
\begin{align*}
P &= \pmat{1&0&0&0\\0&1&0&0\\0&0&\frac{1}{\sqrt{2}}&0\\0&0&0&1} \pmat{1&0&0&0\\0&1&0&0\\0&0&0&1\\0&0&1&0}\pmat{1&0&0&0\\0&1&0&0\\0&0&1&0\\0&0&-3&1} \pmat{1 & 0 & 0 & 0 \\ 0 & 1 & 0 & 0 \\ 0 & 0 & 1 & 0 \\ 0 & 1 & 0 & 1} \pmat{1 & 0 & 0 & 0 \\ -2 & 1 & 0 & 0 \\ -3 & 0 & 1 & 0 \\ -4 & 0 & 0 & 1}
\end{align*}
נקבל כי
\[PAP^t = \pmat{1 & 0 & 0 & 0 \\ 0 & 1 & 0 & 0 \\ 0 & 0 & 1 & 0 \\ 0 & 0 & 0 & -1}\]
צורת סילבסטר קנונית.
\end{solution}

\end{document}