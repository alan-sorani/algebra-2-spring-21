\documentclass[a4paper,10pt,oneside,openany]{article}

\usepackage[lang=hebrew]{maths}
\usepackage{hebrewdoc}
\usepackage{stylish}
\usepackage{lipsum}
\let\bs\blacksquare

\title{
אלגברה ב' (104168) \textenglish{---} אביב 2020-2021
\\
דף מידע
}
\date{}

\begin{document}
\maketitle

\subsection*{סגל הקורס}

\begin{itemize}
\item
\begin{description}
\item[מרצה:]
ד"ר מקסים גורביץ'
\item[משרד:]
אמאדו 911
\item[מייל:]
\textenglish{\href{mailto:maxg@technion.ac.il}{maxg@technion.ac.il}}
\item[שעות הרצאה:]
יום א' 10:30-12:30 ויום ג' 10:30-12:30
\end{description}
\item
\begin{description}
\item[מתרגל אחראי:]
אלעד צורני
\item[משרד:]
אמאדו 517
\item[מייל:]
\textenglish{\href{mailto:elad.tzorani@campus.technion.ac.il}{elad.tzorani@campus.technion.ac.il}}
\item[שעות תרגול:]
יום ב' 16:30-18:30
\end{description}
\item
\begin{description}
\item[מתרגלת:]
טלי מונדרר
\item[משרד:]
אמאדו 520
\item[מייל:]
\textenglish{\href{mailto:talimon@campus.technion.ac.il}{talimon@campus.technion.ac.il}}
\item[שעות תרגול:]
יום ה' 12:30-14:30
\end{description}
\end{itemize}

\section*{תיאור הקורס}

ברוכים הבאים לאלגברה ב'! הקורס יהוה המשך לקורס ראשון באלגברה לינארית ויציג כלים שמופיעים כמעט בכל רחבי המתמטיקה, בגיאומטריה, אלגברה, אנליזה, קומבינטוריקה ועוד. כתחום כה מרכזי במתמטיקה, קיימים לו גם שימושים רבים בפיזיקה ובמדעי המחשב, שרבים מהם נשענים על חומר הקורס בדרך זאת או אחרת.

\subsection*{סילבוס}

בקורס נעבור על הנושאים הבאים.

\begin{enumerate}
\item \textbf {סיווג העתקות לינאריות ממרחב וקטורי מרוכב לעצמו:}

סכומים שירים שמורים ביחס לאופרטור, מרחבים עצמיים מוכללים, העתקות נילפוטנטיות, צורת ז'ורדן, פולינום מינימלי.

\item \textbf{גיאומטריה אוקלידית על מרחבים וקטוריים ממשיים ומרוכבים:}

מרחבי מכפלה פנימית,, תכונות אוקלידיות במונחים אלגבריים, תהליך גרהם־שמידט, רגרסיה לינארית על ידי הטלות אורתוגונליות, אופרטור צממוד, משפטים ספקטרליים, סיווג העתקות לינאריות באמצעות 
\textenglish{SVD}.

\item \textbf{תבניות בי־לינאריות:}

סיווג תבניות עד כדי איזומטריה, תבניות סימטריות, ליכסון, משפט סילבסטר, תבניות ריבועיות, תבניות לא‏־מנוונות.
\end{enumerate}

תוך כדי הקורס נכיר גם מרחבים דואליים ומרחבי מנה.

\subsection*{אתר הקורס}

\textenglish{\href{https://moodle.technion.ac.il/course/view.php?id=4118}{https://moodle.technion.ac.il/course/view.php?id=4118}}.

\subsection*{דרישות קדם}

דרישות הקדם לקורס הן הקורסים
\href{https://ug3.technion.ac.il/rishum/course/104166}{אלגברה א' (104166)}
ו%
\href{https://ug3.technion.ac.il/rishum/course/104002}{מושגי יסוד במתמטיקה (104002)}.

\subsection*{מבנה הציון}

\begin{itemize}
\item
15\% מהציון יהיה (תקף) עבור הגשת גיליונות תרגילים.
הציון יהיה ממוצע 6 הציונים הגבוהים יותר מבין ציוני 10 גיליונות.
\textbf{חובה להגיש לפחות 2 מבין 4 הגיליונות האחרונים עבור רכיב זה של הציון}.
הגיליונות חשובים ומומלץ מאוד לפתור את כולם גם מעבר לדרישות הציון.

\item
85\% מהציון יהיה עבור מבחן בכתב שאורכו 3 שעות.
\begin{itemize}
\item מועד א' ביום ו' ה־%
\textenglish{09.07.2021}.
\item מועד ב' ביום ב' ה־%
\textenglish{18.10.2021}.
\end{itemize}
\end{itemize}

\subsection*{שעות קבלה}

שעות הקבלה יהיו, אלא אם כן יצוין אחרת ובתיאום עם הסטודנטים, בזום ובשעות המפורטות באתר הקורס או בתיאום מראש.

\subsection*{תקשורת עם סגל הקורס}

ניתן לתקשר עם סגל הקורס במייל בכתובות המופיעות למעלה.

\begin{itemize}
\item בנושאי שעות קבלה ושאלות מתמטיות ניתן לפנות לכל סגל הקורס. עם שאלות מתמטיות עדיף לפנות לשעות הקבלה, במידת האפשר.
\item בכל נושא אחר כגון הגשת גיליונות, מילואים או מחלות יש לפנות למתרגל האחראי.
\end{itemize}

\subsection*{ספרות מומלצת}

הקורס לא יעקוב אחר אף ספר אחד, אך להלן רשימת ספרים מומלצים העוסקים בנושא הקורס.

\begin{description}
\item[אלגברה א'] \textenglish{--} 
עמיצור, ש'

\item[אלגברה לינארית \textenglish{II}] \textenglish{--} 
אורנשטיין, א'

\begin{english}
\item[{Linear Algebra Done Right}] -- Axler, S. -- 3\textsuperscript{rd} Edition -- Springer

\item[{Linear Algebra}] -- Hoffman, K.; Kunze, R. -- 2\textsuperscript{nd} Edition -- Prentice-Hall
\end{english}
\end{description}

\end{document}