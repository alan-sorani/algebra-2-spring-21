\documentclass[a4paper,10pt,twoside,openany]{article}

\usepackage[lang=hebrew]{maths}
\usepackage{hebrewdoc}
\usepackage{stylish}
\usepackage{lipsum}
\let\bs\blacksquare

\setlength{\parindent}{0pt}

%%%%%%%%%%%%
% Styling %
%%%%%%%%%%%%

\usepackage{enumitem}

\renewcommand{\emph}[1]{\textbf{#1}}

%%%%%%%%%%%%%
% Counters  %
%%%%%%%%%%%%%

\setcounter{section}{1}

%%%%%%%%%%
% Title  %
%%%%%%%%%%
\title{
אלגברה ב' - גיליון תרגילי בית 9 \\
פירוק לערכים סינגולריים, והפירוק הפולארי
\\
\small{22.06.2021}
}
\date{}

\begin{document}
\maketitle

\begin{exercise}
יהי
$V$
מרחב וקטורי מעל
$\mbb{R}$
ותהי
$T \in \endo_{\mbb{R}}\prs{V}$.
הראו שקיימת איזומטריה
$S \in \endo_{\mbb{F}}\prs{V}$
עבורה
\[\text{.} T = \sqrt{T T^*} S\]
\end{exercise}

\begin{exercise}
יהי
$V$
מרחב וקטורי מעל
$\mbb{R}$
ותהי
$T \in \endo_{\mbb{R}}\prs{V}$.
יהיו
$s$
ו־%
$S$
הערכים הסינגולריים הכי קטן והכי גדול של
$T$,
בהתאמה.
\begin{enumerate}
\item הראו כי
$s \norm{v} \leq \norm{Tv} \leq S\norm{v}$
לכל
$V$.
\item הראו כי
$s \leq \abs{\lambda} \leq S$
לכל ערך עצמי
$\lambda$
של
$T$.
\end{enumerate}
\end{exercise}

\end{document}