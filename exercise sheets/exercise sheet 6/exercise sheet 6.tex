\documentclass[a4paper,10pt,twoside,openany]{article}

\usepackage[lang=hebrew]{maths}
\usepackage{hebrewdoc}
\usepackage{stylish}
\usepackage{lipsum}
\let\bs\blacksquare

\setlength{\parindent}{0pt}

%%%%%%%%%%%%
% Styling %
%%%%%%%%%%%%

\usepackage{enumitem}

\renewcommand{\emph}[1]{\textbf{#1}}

%%%%%%%%%%%%%
% Counters  %
%%%%%%%%%%%%%

\setcounter{section}{1}

%%%%%%%%%%
% Title  %
%%%%%%%%%%
\title{
אלגברה ב' - גיליון תרגילי בית 6 \\
הפולינום המינימלי וההעתקה המשוכלפת
\\
\small{תאריך הגשה: 25.05.2021}
}
\date{}

\begin{document}
\maketitle

\begin{exercise}
עבור פולינום מתוקן
$p \in \mbb{F}\brs{x}$
נכתוב
\[p\prs{x} = \sum_{i = 0}^n c_i x^i\]
כאשר
$c_n = 1$,
ונגדיר את
\emph{המטריצה המלווה של
$p$}
על ידי
\[\text{.} C\prs{p} \ceq \pmat{0 & 0 & \cdots & 0 & -c_0 \\ 1 & 0 & \cdots & 0 & -c_1 \\ 0 & 1 & \cdots & 0 & -c_2 \\ \vdots & \vdots & \ddots & \vdots & \vdots \\ 0 & 0 & \dots & 1 & -c_{n-1}} \in M_n\prs{\mbb{F}}\]
נתון כי עבור
$p \in \mbb{F}\brs{x}$
הפולינומים האופייני והמינימלי של
$C\prs{p}$
שווים ל־%
$p$
(הוכחה לכך מופיעה ברשימות תרגול 7).

\begin{enumerate}
\item הראו שהתנאים הבאים שקולים עבור
$A \in M_n\prs{\mbb{F}}$.

מותר להניח כי
$\mbb{F}$
סגור אלגברית, למרות שזה לא נחוץ.

\begin{enumerate}[label = (\roman*)]
\item $A$
דומה ל־%
$C\prs{p_A}$.
\item הפולינום האופייני של
$A$
שווה לפולינום המינימלי שלה.
\item יש וקטור
$v \in \mbb{F}^n$
ציקלי עבור
$A$,
במובן ש־%
$\prs{v, Av, \ldots, A^{n-2}v, A^{n-1}v}$
בסיס של
$V$.
\end{enumerate}
\item עבור
$T \in \endo_{\mbb{F}}\prs{V}$
ו־%
$v \in V$
נגדיר את
\emph{הפולינום המינימלי של
$T$
ביחס ל־%
$v$}
להיות הפולינום המתוקן
$m_{T,v}$
מהמעלה הנמוכה ביותר עבורו
$m_{T,v}\prs{T}\prs{v} = 0$.
\\
הראו כי
$\deg_{\mbb{F}}m_{T,v} \geq k$
אם ורק אם
\[ \prs{v, Tv, \ldots, T^{k-1} v}\]
בלתי־תלויה לינארית.

\item
מטריצה מהצורה
\[\pmat{0 & 0 & \cdots & 0 & -c_0 \\ 1 & 0 & \cdots & 0 & -c_1 \\ 0 & 1 & \cdots & 0 & -c_2 \\ \vdots & \vdots & \ddots & \vdots & \vdots \\ 0 & 0 & \dots & 1 & -c_{n-1}} \in M_n\prs{\mbb{F}}\]
נקראת
\emph{בלוק רציונלי}.
מטריצת בלוקים שכל בלוקיה הם בלוקים רציונליים נקראת
\emph{מטריצה רציונלית קנונית}.

תהי
$T \in \endo_{\mbb{F}}\prs{V}$
עבור
$\mbb{F}$
סגור אלגברית.
הראו שקיים בסיס
$B$
של
$V$
עבורו
$\brs{T}_B$
מטריצה רציונלית קנונית.

\emph{רמז:}
השתמשו בבסיס ז'ורדן כדי לבנות תת־מרחבים מהצורה
\[\spn\prs{v, Av, \ldots, A^{n-1}}\]
וכך ש־%
$V$
הוא הסכום הישר שלהם.
\end{enumerate}
\end{exercise}

\begin{exercise}%2
עבור
$T \in \hom_{\mbb{F}}\prs{V,W}$
נגדיר
\begin{align*}
T^t \colon W^* &\to V^* \\
\text{.} \hspace{3em} f &\mapsto f \circ T
\end{align*}

\begin{enumerate}
\item הראו כי עבור
$T \in \endo_{\mbb{F}}\prs{V}$
מתקיים
$T^t \sim T$.
\item 
הראו כי
עבור
$A \in M_{n}\prs{\mbb{F}}$
מתקיים
$L_{A^t} \sim R_A$
וגם
$L_A^t \sim L_{A^t}$.
הסיקו כי מתקיים
\[\text{.} L_A \sim L_{A^t} \sim R_A \sim R_{A^t}\]
\end{enumerate}
\end{exercise}

\end{document}