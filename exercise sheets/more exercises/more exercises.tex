\documentclass[a4paper,10pt,oneside,openany]{article}

\usepackage[lang=hebrew]{maths}
\usepackage{hebrewdoc}
\usepackage{stylish}
\usepackage{lipsum}
\let\bs\blacksquare

\title{
אלגברה ב' (104168) \textenglish{---} אביב 2020-2021
\\
עוד תרגילים
}
\author{אלעד צורני}
\date{\today}

\begin{document}
\maketitle

\setcounter{section}{1}

\section{מרחבי העתקות}

\begin{exercise}
תהי
$S \ceq \set{4}$
ויהי
$V = \mbb{R}^S$
מרחב הפונקציות מ־%
$S$
ל־%
$\mbb{R}$.
תהי
\begin{align*}
f \colon S &\to \mbb{R} \\
n &\mapsto n^2 - 1
\end{align*}
ותהי
\begin{align*}
m_f \colon \mbb{R}^S &\to \mbb{R}^S \\
\text{.} \hphantom{lalalal} g &\mapsto fg
\end{align*}
מצאו בסיס של
$\mbb{R}^S$
ומטריצה מייצגת של $m_f$ לפי אותו בסיס.
\end{exercise}

\begin{solution}
נשים לב כי
$B \ceq \prs{\delta_1, \delta_2, \delta_3, \delta_4}$
בסיס עבור
$\mbb{R}^S$,
כי זאת קבוצה בלתי־תלויה לינארית וכי עבור כל
$g \in \mbb{R}^S$
ו־%
$n \in [4]$
מתקיים
\[\text{.} g\prs{n} = \sum_{i \in [4]} g\prs{i} \delta_i\prs{n}\]

מתקיים
\begin{align*}
m_f\prs{\delta_1} = f\delta_1 = f\prs{1} \delta_1
\end{align*}
ובאותו אופן
$m_f\prs{\delta_i} = f\prs{i} \delta_i$
לכל
$i \in [4]$.
לכן
\begin{align*}
\brs{m_f}_B &= \pmat{f\prs{1} & & & \\ & f\prs{2} & & \\ & & f\prs{3} & \\ & & & f\prs{4}}
\\ \text{.} \hphantom{\brs{m_f}_B} &= \pmat{0 & & & \\ & 3 & & \\ & & 8 & \\ & & & 15}
\end{align*}
\end{solution}

\begin{exercise}
הראו כי
\[\text{.} \dim \hom_{\mbb{F}}\prs{V,W} = \dim_{\mbb{F}}\prs{V} \cdot \dim_{\mbb{F}}\prs{W}\]
\end{exercise}

\begin{solution}
יהי
$B = \prs{v_1, \ldots, v_n}$
בסיס של
$V$
ויהי
$C = \prs{w_1, \ldots, w_m}$
בסיס של
$w$.
עבור
$i \in [n], j \in [m]$
תהי
\begin{align*}
\rho_{i,j} \colon V &\to W \\
\text{.} \sum_{k \in [n]} \alpha_k v_k &\to \alpha_i w_j
\end{align*}
נראה כי
\[B = \prs{\rho_{1,1}, \ldots, \rho_{1,2}, \ldots, \rho_{1,n}, \rho_{2,1}, \ldots, \rho_{2,n}, \ldots, \rho_{n,1}, \ldots, \rho_{n,n}}\]
בסיס של
$\hom_{\mbb{F}}\prs{V,W}$.
מספיק להראות שזאת קבוצה בלתי־תלויה לינארית, ואכן אם
\[\rho_{i,j} = \sum_{\prs{k,\ell} \neq \prs{i,j}} \alpha_{k,\ell} \rho_{k,\ell}\]
נקבל
\[\rho_{i,j}\prs{v_i} = w_j\]
אבל
\[\sum_{\prs{k,\ell} \neq \prs{i,j}} \alpha_{k,\ell} \rho_{k,\ell} \prs{v_i} = \sum_{\ell \neq j} \alpha_{i,\ell} w_\ell\]
וזה שונה מ־%
$w_j$
כי
$\prs{w_1, \ldots, w_m}$
בלתי־תלויה.
\end{solution}

\begin{exercise}
תהי
\begin{align*}
S \colon \mbb{R}^2 &\to \mbb{R}^2 \\
\text{.} \hphantom{la} \prs{x,y} &\mapsto \prs{-2y, x}
\end{align*}
נגדיר
\begin{align*}
T \colon \endo_{\mbb{R}}\prs{\mbb{R}^2} &\to \endo_{\mbb{R}}\prs{\mbb{R}^2} \\
\text{.} \hphantom{lalalalalallaa} U &\mapsto SU
\end{align*}
יהי
\[B = \prs{\rho_{i,j}}{i,j \in [2]}\]
כאשר
$\rho_{i,j}$
מוגדרות כמו בסעיף הקודם לפי הבסיסים הסטנדרטיים על
$\mbb{R}^2$.
\begin{enumerate}
\item מצאו את
$\brs{S}_E$.
\item מצאו את
$\brs{T}_B$.
\end{enumerate}
\end{exercise}

\begin{solution}
\begin{enumerate}
\item 
$S e_1 = e_2$
וגם
$S e_2 = -2 e_1$
לכן
\[\text{.} \brs{S}_E = \pmat{0 & -2 \\ 1 & 0}\]

\item מתקיים
\begin{align*}
T \rho_{1,1} e_1 &= S \rho_{1,1} e_1 = S e_1 = e_2 \\
T \rho_{1,1} e_2 &= S \rho_{1,1} e_2 = S 0 = 0 \\
T \rho_{1,2} e_1 &= S \rho_{1,2} e_1 = S e_2 = -2 e_1 \\
T \rho_{1,2} e_2 &= S \rho_{1,2} e_2 = S 0 = 0 \\
T \rho_{2,1} e_1 &= S \rho_{2,1} e_1 = S 0 = 0 \\
T \rho_{2,1} e_2 &= S \rho_{2,1} e_2 = S e_1 = e_2 \\
T \rho_{2,2} e_1 &= S \rho_{2,2} e_1 = S 0 = 0 \\
T \rho_{2,2} e_2 &= S \rho_{2,2} e_2 = S e_2 = -2 e_1
\end{align*}
לכן
\begin{align*}
T \rho_{1,1} &= \rho_{1,2} \\
T \rho_{1,2} &= -2\rho_{1,1} \\
T \rho_{2,1} &= \rho_{2,2} \\
T \rho_{2,2} &= -2 \rho_{2,1}
\end{align*}
ולכן
\begin{align*}
\text{.} \brs{T}_B &=
\pmat{
0 & -2 & 0 & 0 \\
1 & 0 & 0 & 0 \\
0 & 0 & 0 & -2 \\
0 & 0 & 1 & 0
} = \pmat{S & 0 \\ 0 & S}
\end{align*}
\end{enumerate}
\end{solution}

\section*{סכומים ישרים}

\begin{definition}[מכפלה של מרחבים וקטוריים]
יהי
$\mcal{S}$
אוסף סדור (לאו דווקא סופי או בן־מנייה) של מרחבים וקטוריים מעל שדה
$\mbb{F}$ (עם חזרות).
נגדיר את המכפלה
\[\prod_{V \in \mcal{S}} V\]
להיות המכפלה התורת־קבוצתית עם חיבור וכפל בסקלר רכיב רכיב.

אם
$\mcal{S} = \prs{V_1, \ldots, V_n}$
סופי, נכתוב לפעמים
\[\text{.} V_1 \times \ldots \times V_n \ceq \prod_{V \in \mcal{S}} V\]

אם כל איברי
$\mcal{S}$
שווים לאותו מרחב
$V$
נכתוב לפעמים
\[\text{.} V^{\abs{\mcal{S}}} \ceq \prod_{V \in \mcal{S}} V\]
\end{definition}

\begin{exercise}
תהי
$S \ceq \set{4}$
ויהי
$V = \mbb{R}^S$
מרחב הפונקציות מ־%
$S$
ל־%
$\mbb{R}$.
תהי
\begin{align*}
f \colon S &\to \mbb{R} \\
n &\mapsto n^2 - 1
\end{align*}
ותהי
\begin{align*}
m_f \colon \mbb{R}^S &\to \mbb{R}^S \\
\text{.} \hphantom{lalalal} g &\mapsto fg
\end{align*}
מצאו בסיס של
$\mbb{R}^S$
ומטריצה מייצגת של $m_f$ לפי אותו בסיס.
\end{exercise}

\begin{solution}
נשים לב כי
$B \ceq \prs{\delta_1, \delta_2, \delta_3, \delta_4}$
בסיס עבור
$\mbb{R}^S$,
כי זאת קבוצה בלתי־תלויה לינארית וכי עבור כל
$g \in \mbb{R}^S$
ו־%
$n \in [4]$
מתקיים
\[\text{.} g\prs{n} = \sum_{i \in [4]} g\prs{i} \delta_i\prs{n}\]

מתקיים
\begin{align*}
m_f\prs{\delta_1} = f\delta_1 = f\prs{1} \delta_1
\end{align*}
ובאותו אופן
$m_f\prs{\delta_i} = f\prs{i} \delta_i$
לכל
$i \in [4]$.
לכן
\begin{align*}
\brs{m_f}_B &= \pmat{f\prs{1} & & & \\ & f\prs{2} & & \\ & & f\prs{3} & \\ & & & f\prs{4}}
\\ \text{.} \hphantom{\brs{m_f}_B} &= \pmat{0 & & & \\ & 3 & & \\ & & 8 & \\ & & & 15}
\end{align*}
\end{solution}

\begin{exercise}
\begin{enumerate}
\item הראו שמתקיים
\[\text{.} \dim_{\mbb{F}} \prs{V \times W} = \dim_{\mbb{F}} V + \dim_{\mbb{F}} W\]
\item הראו שמתקיים
\[\text{.} \dim_{\mbb{F}} \hom_{\mbb{F}}\prs{V,W} = \prs{\dim_{\mbb{F}} V}\prs{\dim_{\mbb{F}} W}\]
\item הסיקו שמתקיים
\[\hom_{\mbb{F}}\prs{V, W_1 \times \ldots \times W_n} \cong \hom_{\mbb{F}}\prs{V, W_1} \times \ldots \times \hom_{\mbb{F}}\prs{V, W_n}\]
וגם
\[\text{.} \hom_{\mbb{F}}\prs{V_1 \times \ldots \times V_n, W} \cong \hom_{\mbb{F}}\prs{V_1, W} \times \ldots \times \hom_{\mbb{F}}\prs{V_n, W}\]
\end{enumerate}
\end{exercise}

\begin{solution}
%TODO fill in
\end{solution}

\begin{definition}[מכפלה של העתקות]
תהי
$\mcal{I}$
קבוצה סדורה עם חזרות, יהיו
$\prs{V_i}_{i \in \mcal{I}}, \prs{W_i}_{i \in \mcal{I}}$
מרחבים וקטוריים ויהיו
$\prs{T_i}_{i \in \mcal{I}}$
העתקות לינאריות כך ש־%
\[\text{.} T_i \colon V_i \to W_i\]
נגדיר את המכפלה
\begin{align*}
\prod_{i \in \mcal{I}} T_i \colon \prod_{i \in \mcal{I}} V_i &\to \prod_{i \in \mcal{I}} W_i \\
\text{.} \prs{v_i}_{i \in \mcal{I}} &\mapsto \prs{T_i v_i}_{i \in \mcal{I}}
\end{align*}
אם
$\mcal{I}$
סופית, ניקח בדרך כלל
$\mcal{I} = \brs{n}$
ואז נכתוב לפעמים
\[\text{.} T_1 \times T_2 \times \ldots \times T_n \ceq \prod_{i \in [n]} T_i\]
\end{definition}

\begin{exercise}
\begin{enumerate}
\item יהיו
$V_1 \ceq \mbb{R}^3$
עם הבסיס הסטנדרטי
$B_1 = \prs{e_1, e_2, e_3}$
ו־%
$V_2 \ceq \mbb{R}^2$
עם הבסיס הסטנדרטי
$B_2 = \prs{f_1, f_2}$.

יהי
$B = \prs{e_1, e_2, e_3, f_1, f_2}$
בסיס של
$V_1 \times V_2$,
ותהיינה
\begin{align*}
T_1 \colon \mbb{R}^3 &\to \mbb{R}^3 \\
\pmat{a \\ b \\ c} &\mapsto \pmat{b \\ c \\ a}
\end{align*}
ו־%
\begin{align*}
T_2 \colon \mbb{R}^2 &\to \mbb{R}^2 \\
\text{.} \pmat{x \\ y} &\mapsto \pmat{y \\ -x}
\end{align*}
חשבו את
$\prs{T_1 \times T_2}_B$.

\item יהיו
$\prs{V_i}_{i \in \brs{n}}$
מרחבים וקטוריים עם בסיסים
\[B_i \ceq \prs{v^{\prs{i}}_1, \ldots, v^{\prs{i}}_{k_i}}\]
בהתאמה.
לכל
$i \in [n]$
תהי
$T_i \in \hom_{\mbb{F}}\prs{V_i, V_i}$.
יהי
\[\text{.} B = \prs{v^{\prs{1}}_1, \ldots, v^{\prs{1}}_{k_1}, \ldots, v^{\prs{n}}_1, \ldots, v^{\prs{n}}_{k_n}}\]
הראו כי
\[\brs{T_1 \times \ldots \times T_n}_B\]
מטריצה
$\prs{k_1, \ldots, k_n}$%
אלכסונית בלוקים.
\end{enumerate}
\end{exercise}

\begin{exercise}
תהי
$T \in \endo_{\mbb{F}}\prs{V}$
ויהיו
$\lambda_1, \ldots, \lambda_k \in \mbb{F}$
הערכים העצמיים השונים של
$T$.

נראה בתרגיל זה כי
$T$
לכסינה אם ורק אם
\[\text{.} V = \bigoplus_{i \in [k]} V_\lambda\]

\begin{enumerate}
\item נניח כי
\[\text{.} V = \bigoplus_{i \in [k]} V_\lambda\]
הראו כי יש בסיס של
$V$
שמורכב מוקטורים עצמיים של
$T$
והסיקו כי
$T$
לכסינה.

\item נניח כי
$T$
לכסינה. הראו שניתן לכתוב
\begin{align*}
V = \bigoplus_{i \in [k]} V_{\lambda_i}'
\end{align*}
עבור
$V_{\lambda_i}' \subseteq V_{\lambda_i}$
לכל
$i \in [k]$.

\item הראו תחת הנחות הסעיף הקודם שמתקיים
$V_{\lambda_i}' = V_{\lambda_i}$
והסיקו כי במקרה זה
\[\text{.} V = \bigoplus_{i \in [k]} V_\lambda\]
\end{enumerate}
\end{exercise}

\begin{exercise}
יהיו
$V,W$
מרחבים וקטוריים
$n$%
־מימדיים מעל
$\mbb{F}$.
תהי
$T \in \endo_{\mbb{F}}\prs{V}$
ויהי
$\lambda \in \mbb{F}$
עם מרחב עצמי
$V_\lambda$.
נקרא ל־%
$\dim_{\mbb{F}} V_\lambda$
\emph{הריבוי הגיאומטרי של
$\lambda$
כערך עצמי של
$T$}.
\begin{enumerate}
\item 
תהי
$S \in \endo_{\mbb{F}}\prs{W}$.
ויהי
$P \in \hom_{\mbb{F}}\prs{V,W}$
איזומורפיזם עבורו
$T = P^{-1} S P$.
הראו של־%
$T,S$
אותם ערכים עצמיים מאותם ריבויים גיאומטריים.

\item הסיקו ש־%
$S$
לכסינה אם ורק אם
$T$
לכסינה.

\item 
נניח כי
$T$
לכסינה
ותהי
$S' \in \endo_{\mbb{F}}\prs{W}$
עם אותם ערכים עצמיים וריבויים גיאומטריים כמו של
$T$.
הראו כי
$T \sim S'$.
\end{enumerate}
\end{exercise}

\begin{exercise}[נילפוטנטיות]
העתקה לינארית
$T \colon V \to V$
נקראת
\emph{נילפוטנטית}
אם קיים
$n \in \mbb{N}_+$
עבורו
$T^n = 0$.
$n$
המינימלי המקיים זאת נקרא
\emph{האינדקס של $T$}.

\begin{enumerate}
\item הראו שצירוף לינארי של העתקות נילפוטנטיות הוא נילפוטנטי.
\item הראו שעבור
$T \colon V \to V$
נילפוטנטית מתקיים
$\det\prs{T} = \tr\prs{T} = 0$.
\item 
הראו שעבור
$T \colon V \to V$
נילפוטנטית הערך העצמי היחיד הוא
$0$.
הסיקו שעבור
$T$
נילפוטנטית ולכסינה מתקיים
$T = 0$.
\item תהי
$T$
נילפוטנטית. הראו שההעתקות
$\id_V \pm T$
הפיכות.

\textbf{רמז:}
נסו לחשוב על ההופכי של $1 \pm x$ כאשר
$x$
מספר.

\item 
יהי
$V$
מרחב וקטורי ממימד
$n \in \mbb{N}_+$
ותהי
$T$
נילפוטנטית מאינדקס
$k$.
לכל
$i \in [k]$
נסמן
$n_i \ceq \dim \ker T^i$.
הראו שמתקיים
\[\text{.} 0 < n_1 < n_2 < \ldots < n_{k-1} < n_k = n\]

\item יהי
$V$
מרחב וקטורי ממימד
$n \in \mbb{N}_+$
ותהי
$T \colon V \to V$.
יהי
$B = \prs{v_1, \ldots, v_n}$
בסיס של
$V$
ותהי
\begin{align*}
T \colon V &\to V \\
v_1 &\mapsto 0 \\
\text{.} \forall i > 1 \colon v_i &\mapsto v_{i-1}
\end{align*}
הראו כי
$T$
נילפוטנטית מאינדקס
$n$
וכיתבו את
$\brs{T}_B$.

\item 
נאמר כי
$A\in M_n\prs{\mbb{F}}$
נילפוטנטית אם
$L_A$
נילפוטנטית.

תהי
$A \in M_n\prs{\mbb{F}}$
אלכסונית בלוקים $\prs{m_1, \ldots, m_k}$
כאשר כל בלוק
$A_i$
נילפוטנטי מאינדקס
$n_i$.
הראו ש־%
$A$
נילפוטנטית מאינדקס
$\max_{i \in [k]} \prs{n_i}$.
\item יהי
$V$
מרחב וקטורי ממימד
$n \in \mbb{N}_+$
ותהי
\[0 < n_1 < n_2 < \ldots < n_{k-1} < n_k = n\]
סדרת מספרים.

הראו שיש העתקה
$T \colon V \to V$
כך שמתקיים
$n_i = \dim \ker T^i$
לכל
$i \in [k]$.
\end{enumerate}
\end{exercise}

\begin{exercise}%5
פולינום
$p \in \mbb{F}\brs{x}$
נקרא
\emph{מתוקן}
אם המקדם המוביל שלו שווה
$1$.
עבור פולינום כזה ממעלה
$n \in \mbb{N}$
נכתוב
\[p\prs{x} = \sum_{i = 0}^n c_i x^i\]
כאשר
$c_n = 1$,
ונגדיר את
\emph{המטריצה המלווה של
$p$}
על ידי
\[\text{.} C\prs{p} \ceq \pmat{0 & 0 & \cdots & 0 & -c_0 \\ 1 & 0 & \cdots & 0 & -c_1 \\ 0 & 1 & \cdots & 0 & -c_2 \\ \vdots & \vdots & \ddots & \vdots & \vdots \\ 0 & 0 & \dots & 1 & -c_{n-1}} \in M_n\prs{\mbb{F}}\]

\begin{enumerate}
\item יהי
$p \in \mbb{F}\brs{x}$
ממעלה
$n \in \mbb{N}$.
מצאו את הפולינום המינימלי של
$C\prs{p}$.
\\
\emph{רמז:}
השתמשו בעובדה שהפולינום המינימלי של מטריצה
$A$
שווה ל־%
$\det\prs{xI - A}$
וחשבו את הדטרמיננטה לפי השורה הראשונה.

\item
הראו שהתנאים הבאים שקולים
עבור
$A \in M_n\prs{\mbb{F}}$.
\begin{enumerate}[label = (\roman*)]
\item
$A$
דומה ל־%
$C\prs{p_A}$.
\item
לכל פולינום
$p \in \mbb{F}_{n-1}\brs{x}$
מתקיים
$p\prs{A} \neq 0$.
\item
יש וקטור
$v \in \mbb{F}^n$
\emph{ציקלי עבור
$A$}
במובן שהקבוצה
$\set{v, Av, \ldots, A^{n-1}v}$
היא בסיס של
$V$.
\end{enumerate}
\end{enumerate}
\end{exercise}

\begin{exercise}
יהי
$V$
מרחב וקטורי ממימד
$n \in \mbb{N}_+$
ותהי
\[0 < n_1 < n_2 < \ldots < n_{k-1} < n_k = n\]
סדרת מספרים.

\begin{enumerate}
\item נניח כי
$n_i - n_{i-1}$
מונוטונית יורדת (לאו דווקא ממש).
הראו שיש העתקה
$T \in \endo_{\mbb{F}}\prs{V}$
עבורה
$n_i = \dim\ker\prs{T^i}$
לכל
$i \in [k]$.
\item נניח כי יש
$i \in [k]$
עבורו
$n_{i+1} - n_i > n_i - n_{i-1}$.
הראו שאין העתקה
$T \in \endo_{\mbb{F}}\prs{V}$
עבורה
$n_i = \dim\ker\prs{T^i}$
לכל
$i \in [k]$.
\end{enumerate}
\end{exercise}

\begin{solution}
\begin{enumerate}
\item
\begin{itemize}
\item במקרה
$k = 1$
ניקח בסיס
$B$
של
$V$,
וראינו כי
$T \in \endo_{\mbb{F}}\prs{V}$
עם
\[\brs{T}_B = J_n\prs{0} \ceq \pmat{0 & 1 & 0 & \ldots & 0 \\ 0 & 0 & 1 & \ddots & \vdots \\ \vdots & \ddots & 0 & \ddots & 0 \\ 0 & 0 & \ddots & 0 & 1 \\ 0 & 0 & \cdots & 0 & 0}\]
נילפוטנטית מאינדקס
$n$.

\item  באופן כללי, נרצה להשתמש בדוגמא הנ"ל כדי לבנות את המקרה הכללי.
אם
$V = \bigoplus_{j \in \brs{\ell}} V_j$
ונתונות
$T_j \in \endo_{\mbb{F}}\prs{V_j}$
מתקיים
\[\ker\prs{\bigoplus_{j \in \brs{\ell}} T_j} = \bigoplus_{j \in \brs{\ell}} \ker\prs{T_j}\]
לכן
\begin{align*}
\dim \ker\prs{\prs{\bigoplus_{j \in \brs{\ell}} T_j}^i} &= \dim \ker\prs{\bigoplus_{j \in \brs{\ell}} T_j^i}
\\ &= \dim \bigoplus_{j \in \brs{\ell}} \ker\prs{T_j^i}
\\ \text{.} \hphantom{\ker\prs{\prs{\bigoplus_{j \in \brs{\ell}} T_j}^i}} &= \sum_{j \in \brs{\ell}} \dim \ker\prs{T_j^i}
\end{align*}
נסמן
$n_0 = 0$
וגם
$m_i = n_i - n_{i-1}$.
אז
\begin{align*}
\text{.} \sum_{i \in [k]} m_i = n_k - 0 = n_k = n
\end{align*}
יהי
$B$
בסיס ל־%
$V$
ותהי
$T \in \endo_{\mbb{F}}\prs{V}$
עם
\begin{align*}
\brs{T}_B &= \pmat{A_1 & & 0 \\ & \ddots & \\ 0 & & A_\ell}
\end{align*}
כאשר
$m_1 - m_2$
הבלוקים הראשונים הם
$\prs{0} = J_1\prs{0}$,
כאשר
$m_2 - m_3$
הבלוקים שאחריהם הם
$J_2\prs{0}$
וכאשר באופן כללי יש
$m_i - m_{i+1}$
בלוקים שהם
$J_i\prs{0}$
לכל
$i \in [k-1]$
וגם
$m_k$
בלוקים שהם
$J_k\prs{0}$.
נקבל שסכום הבלוקים הוא
\[\sum_{i \in [k-1]} i\prs{m_i - m_{i+1}} + k m_k = \sum_{i \in [k]} m_i = n\]
לכן באמת אפשר לקחת
$\brs{T}_B$
כזאת.

נקבל
\[V = \bigoplus_{j \in \brs{\ell}} V_j\]
והעתקות
$T_j \in \endo_{\mbb{F}}\prs{V_j}$
עבורן
$\dim \ker\prs{T_j^i} = \min\set{i, \dim V_j}$
וגם
\[\text{.} T = \bigoplus_{j \in \brs{\ell}} T_j\]
עבור
$i < k$
נקבל
\begin{align*}
\dim \ker\prs{T^i} &= \sum_{j \in \brs{\ell}} \dim \ker\prs{T_j^i}
\\&= \sum_{s \in \brs{i}} s\prs{m_s - m_{s+1}} + i\prs{\sum_{s = i+1}^{k-1} \prs{m_s - m_{s+1}} + m_k}
\\&= \sum_{s \in [i]} m_s - i m_{i+1} + i m_{i+1}
\\&= \sum_{s \in [i]} n_s - n_{s-1}
\\&= \sum_{s \in [i]} n_s - \sum_{s = 0}^{i-1} n_s
\\&= n_i - n_0
\\&= n_i
\end{align*}
כנדרש.
\end{itemize}
\item נניח שיש
$T$
כזאת.
מתקיים
\[n_{i+1} - n_i = \dim \ker\prs{T^{i+1}} - \dim \ker\prs{T^{i}}\]
לכן יש
$m_{i+1} \ceq n_{i+1} - n_i$
וקטורים בלתי־תלויים לינארית
$v_1, \ldots, v_{m_{i+1}}$
ב־%
$\ker\prs{T^{i+1}} \setminus \ker\prs{T^i}$.
נראה כי
$T\prs{v_1}, \ldots, T\prs{v_{m_i+1}}$
בלתי־תלויים לינארית.
יהי
\[\sum_{j \in [m_{i+1}]} \alpha_j T\prs{v_j} = 0\]
צירוף לינארי שלהם ששווה
$0$.
אז
\begin{align*}
0 &= \sum_{j \in [m_{i+1}]} \alpha_j T\prs{v_j}
\\&= T\prs{\sum_{j \in [m_{i+1}]} \alpha_j v_j}
\end{align*}
כלומר
\[\sum_{j \in [m_{i+1}]} \alpha_j v_j \in \ker \prs{T} \subseteq \ker\prs{T^i}\]
בסתירה להנחה על ה־%
$v_j$.
כעת,
$T\prs{v_1}, \ldots, T\prs{v_{m_{i + 1}}} \notin \ker\prs{T^{i-1}}$
כי אחרת
\begin{align*}
T^{i}\prs{v_j} = T^{i-1}\prs{T\prs{v_j}} = 0
\end{align*}
בסתירה להנחה.
לכן
$T\prs{v_1}, \ldots, T\prs{v_{m_{i+1}}}$
הם
$m_{i+1}$
וקטורים בלתי־תלויים לינארית ב־%
$\ker\prs{T^i} - \ker\prs{T^{i-1}}$.
לכן
$n_i - n_{i-1} \geq m_{i+1} = n_{i+1} - n_i$
בסתירה לנתון.
\end{enumerate}
\end{solution}

\begin{exercise}
עבור פולינום מתוקן
$p \in \mbb{F}\brs{x}$
נכתוב
\[p\prs{x} = \sum_{i = 0}^n c_i x^i\]
כאשר
$c_n = 1$,
ונגדיר את
\emph{המטריצה המלווה של
$p$}
על ידי
\[\text{.} C\prs{p} \ceq \pmat{0 & 0 & \cdots & 0 & -c_0 \\ 1 & 0 & \cdots & 0 & -c_1 \\ 0 & 1 & \cdots & 0 & -c_2 \\ \vdots & \vdots & \ddots & \vdots & \vdots \\ 0 & 0 & \dots & 1 & -c_{n-1}} \in M_n\prs{\mbb{F}}\]

\begin{enumerate}
\item יהי
$p \in \mbb{F}\brs{x}$
ממעלה
$n$.
מצאו את הפולינום האופייני ואת הפולינום המינימלי של
$C\prs{p}$.
\item הראו שהתנאים הבאים שקולים עבור
$A \in M_n\prs{\mbb{F}}$.
\begin{enumerate}[label = (\roman*)]
\item $A$
דומה ל־%
$C\prs{p_A}$.
\item הפולינום האופייני של
$A$
שווה לפולינום המינימלי שלה.
\item יש וקטור
$v \in \mbb{F}^n$
ציקלי עבור
$A$,
במובן ש־%
$\prs{v, Av, \ldots, A^{n-2}v, A^{n-1}v}$
בסיס של
$V$.
\end{enumerate}
\end{enumerate}
\end{exercise}

\begin{solution}
\begin{enumerate}
\item
\begin{itemize}
\item
מתקיים
\begin{align*}
\text{.} p_{C\prs{p}} &= \det\prs{I - xC\prs{p}}
= \det\pmat{x & 0 & \cdots & 0 & c_0 \\ -1 & x & \cdots & 0 & c_1 \\ 0 & -1 & \cdots & 0 & -c_2 \\ \vdots & \vdots & \ddots & \ddots & \vdots \\ 0 & 0 & \dots & -1 & x + c_{n-1}}
\end{align*}
ניעזר בכך שהטרמיננטה אינווריאנטית תחת הוספת כפולה של שורה לשורה אחרת.
נוסיף את השורה האחרונה כפול
$x$
לזאת שלפניה. לאחר מכן נוסיף את השורה ה־%
$n-1$
כפול
$x$
לזאת שלפניה ונמשיך כך עד שנקבל מטריצה
\begin{align*}
\pmat{0 & 0 & \cdots & 0 & y \\ -1 & 0 & \cdots & 0 & * \\ 0 & -1 & \cdots & 0 & * \\ \vdots & \vdots & \ddots & \ddots & \vdots \\ 0 & 0 & \dots & -1 & *}
\end{align*}
כאשר
\[\text{.} y = a_0 + x\prs{a_1 + x\prs{a_2 + x\prs{\ldots \prs{a_{n-2} + x\prs{a_{n-1} + x}} \ldots}}} = \sum_{i=0}^n c_i x^i = f\]
אז
\[\text{.} p_{C\prs{p}} = \prs{-1}^{n-1} \det \pmat{-1 & x & \cdots & 0 \\ 0 & -1 & \cdots & 0 \\ \vdots & \vdots & \ddots & \vdots \\ 0 & 0 & \cdots & - 1} = \prs{-1}^{n-1} \prs{-1}^{n-1} f = f\]
\item
יהי
$g \in \mbb{F}_{n-1}\brs{x}$
ונכתוב
\[\text{.} g\prs{x} = \sum_{i = 0}^{n-1} b_i x^i\]
אז
\begin{align*}
g\prs{C\prs{p}}\prs{e_1} &= \sum_{i = 0}^{n-1} b_i C\prs{p}^i e_1
\\&= \sum_{i=0}^{n-1} b_i e_{1+i}
\end{align*}
וביטוי זה שונה מאפס כאשר
$g \neq 0$
כי אז יש
$b_i \neq 0$
עבור
$i$
כלשהו.

לכן
$\deg\prs{m_{C\prs{p}}} \geq n$
ולכן
$m_{C\prs{p}} = p_{C\prs{p}}$.
\end{itemize}

\item
\begin{description}
\item[1 גורר 2:]
נניח כי
$A$
דומה ל־%
$C\prs{p_A}$.
ראינו כי
$p_{C\prs{p_A}} = m_{C\prs{p_A}}$
ונקבל
\[p_A = p_{C\prs{p_A}} = m_{C\prs{p_A}} = m_A\]
כנדרש.


\item[2 גורר 3:]
כיוון שהפולינום המינימלי של
$A$
שווה לפולינום האופייני שלה, הוא מדרגה
$n$.
לכן לכל
$v \in V \setminus \set{0}$
הקבוצה הסדורה
$\prs{v, Av, \ldots, A^{n-1} v}$
בלתי־תלויה לינארית. כיוון שהיא מגודל
$n$
היא גם בסיס.

\item[3 גורר 1:]
יהי
\[B \ceq \prs{v, Av, \ldots, A^{n-1} v}\]
הבסיס הנתון.
נרצה להראות כי
\[\text{.} \brs{L_A}_B = C\prs{p_A}\]
ניתן לראות כי
$n-1$
העמודות הראשונות בשתי המטריצות שוות, מהגדרת
$B$.
כעת נכתוב
\[p_A\prs{x} = \sum_{i=0}^n c_i x^i\]
ואז מקיילי־המילטון
\begin{align*}
\text{.} A^n = - \sum_{i=0}^{n-1} c_i A^i
\end{align*}
לכן
\begin{align*}
L_A \prs{A^{n-1} v} &= A^n v = - \sum_{i=0}^{n-1} c_i A^i v
\end{align*}
ולכן העמודה ה־%
$n$
של
$\brs{L_A}_B$
היא
$\pmat{-c_0 \\ \vdots \\ - c_{n-1}}$,
כנדרש.
\end{description}
\end{enumerate}
\end{solution}

\begin{definition}[פולינום מינימלי של וקטור]
תהי
$T \in \endo_{\mbb{F}}\prs{V}$
ויהי
$v \in V$.
\emph{הפולינום המינימלי של
$T$
ביחס ל־%
$v$}
הוא הפולינום המתוקן
$m_{T,v}$
מהמעלה הנמוכה ביותר עבורו
\[\text{.} m_{T,v}\prs{T}\prs{v} = 0\]
\end{definition}

\begin{exercise}
מטריצה מהצורה
\[\pmat{0 & 0 & \cdots & 0 & -c_0 \\ 1 & 0 & \cdots & 0 & -c_1 \\ 0 & 1 & \cdots & 0 & -c_2 \\ \vdots & \vdots & \ddots & \vdots & \vdots \\ 0 & 0 & \dots & 1 & -c_{n-1}} \in M_n\prs{\mbb{F}}\]
נקראת
\emph{בלוק רציונלי}.
מטריצת בלוקים שכל בלוקיה הם בלוקים רציונליים נקראת
\emph{מטריצה רציונלית קנונית}.

תהי
$T \in \endo_{\mbb{F}}\prs{V}$.
הראו שקיים בסיס
$B$
של
$V$
עבורו
$\brs{T}_B$
מטריצה רציונלית קנונית.
\end{exercise}

\begin{solution}
עבור וקטור
$v \in V$
נסמן
\begin{align*}
d &\ceq \deg_{\mbb{F}} m_{T,v}\prs{x} \\
B_v &\ceq \prs{v, Tv, T^2 v, \ldots, T^{d-1} v} \\
\text{.} \trs{v} &\ceq \spn B_v
\end{align*}
מהגדרת
$m_{T,v}$
נקבל כי
$B_v$
בלתי־תלויה לינארית.
נקבל מהתרגיל הקודם כי
$\brs{\rest{T}{\trs{v}}}_{B_v}$
בלוק רציונלי לכל
$v \in V$.

לכן די למצוא וקטורים
$v_1, \ldots, v_k \in V$
עבורם
\[\text{.} V = \bigoplus_{i \in [k]} \trs{v_i}\]
עבור שרשרת ז'ורדן מקסימלית
\[\prs{\prs{T-\lambda \id_V}^{k-1} v, \ldots, \prs{T-\lambda\id_V} v, v}\]
מתקיים
\[\text{.} m_{T,v}\prs{x} = \prs{x-\lambda}^k\]
אז
\[B_v = \prs{v, Tv, T^2v, \ldots, T^{k-1} v}\]
בלתי־תלויה לינארית.
ניקח
\[B = B_{v_1} * \ldots * B_{v_k}\]
עבור
$\prs{v_1, \ldots, v_k}$
אוסף הוקטורים בראש שרשראות ז'ורדן המקסימליות של
$T$
(לפי בסיס כלשהו).
אז
\[V = \bigoplus_{i \in [k]} \trs{v_i}\]
כמסקנה ממשפט ז'ורדן ונקבל כי
$\brs{T}_B$
מטריצה רציונלית קנונית.
\end{solution}

\begin{exercise}[זהות הפולריזציה]
יהי
$V$
מרחב מכפלה פנימית מעל
$\mbb{R}$.
הראו שלכל
$u,v \in V$
מתקיים
\[\text{.} \trs{u,v} = \frac{\norm{u+v}^2 - \norm{u-v}^2}{4}\]
\end{exercise}

\begin{solution}
עבור
$u,v$
מתקיים
\begin{align*}
\frac{\norm{u+v}^2 - \norm{u-v}^2}{4}
&=
\frac{\trs{u+v,u+v} - \trs{u-v,u-v}}{4}
\\&=
\frac{\trs{u,u}+\trs{u,v}+\trs{v,u}+\trs{v,v} - \prs{\trs{u,u} - \trs{u,v} - \trs{v,u} + \trs{v,v}}}{4}
\\&=
\frac{2\trs{u,v} + 2 \trs{v,u}}{4}
\\&= \frac{4 \trs{u,v}}{4}
\\ \text{.} \hphantom{\frac{\norm{u+v}^2 - \norm{u-v}^2}{4}} &= \trs{u,v}
\end{align*}
\end{solution}

\begin{exercise}
תהי
$g$
תבנית בילינארית.
הראו כי יש
$w \in V$
עבורו
$g\prs{w,w} = 0$,
 אם ורק אם יש
$n_0 > 0$
בצורת סילבסטר של
$g$.
\end{exercise}

\begin{solution}
יהי
$C = \prs{u_1, \ldots, u_n}$
בסיס של
$V$
עבורו
$\brs{g}_C$
בצורת סילבסטר ונניח כי
$n_0 > 0$.
אז
$g\prs{u_n, u_n} = 0$.

להיפך, נניח כי יש
$w \in V$
עבורו
$g\prs{w,w} = 0$.
נסמן
$w_1 = w$
ונשלים לבסיס אורתונורמלי
$B = \prs{w_1, \ldots, w_n}$.
ננרמל על ידי
\[\tilde{u}_i \ceq \fcases{\frac{w_i}{\sqrt{g\prs{w_i,w_i}}} & g\prs{w_i, w_i} \neq 0 \\ w_i & g\prs{w_i, w_i} = 0}\]
וניקח שינוי סדר
$u_i$
עבורו
$u_n = \tilde{u}_1$
ועבורו בבסיס
$C = \prs{u_1, \ldots, u_n}$
המטריצה
$\brs{g}_C$
היא בצורת סילבסטר.
\end{solution}

\end{document}