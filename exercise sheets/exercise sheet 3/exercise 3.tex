\documentclass[a4paper,10pt,twoside,openany]{article}

\usepackage[lang=hebrew]{maths}
\usepackage{hebrewdoc}
\usepackage{stylish}
\usepackage{lipsum}
\let\bs\blacksquare

\setlength{\parindent}{0pt}

%%%%%%%%%%%%
% Styling %
%%%%%%%%%%%%

\usepackage{enumitem}

\renewcommand{\emph}[1]{\textbf{#1}}

%%%%%%%%%%%%%
% Counters  %
%%%%%%%%%%%%%

\setcounter{section}{1}

%%%%%%%%%%
% Title  %
%%%%%%%%%%
\title{
אלגברה ב' - גיליון תרגילי בית 3 \\
מרחבים שמורים, מרחבים עצמיים מוכללים ומשפט קיילי־המילטון
\\
\small{תאריך הגשה: 25.4.2021} %TODO change date
}
\date{}

\begin{document}
\maketitle

\begin{exercise}%1
\begin{enumerate}
\item
יהי
$V$
מרחב וקטורי ממימד
$n \in \mbb{N}_+$
מעל שדה
$\mbb{F}$
ותהי
$T \in \endo_{\mbb{F}}\prs{V}$.
יהי
$W \leq V$
תת־מרחב
$T$%
־שמור ויהי
$U \leq V$
משלים ישר של
$W$
ב־%
$V$.
הראו כי
$\rest{T}{W} = T_{W,U}$.
\item
מצאו דוגמא עבור
$T \in \endo_{\mbb{F}}\prs{V}$,
$W \leq V$
כלשהו עם משלימים ישרים
$U_1, U_2 \leq V$,
כך ש־%
$\rest{T}{W, U_1} \neq T_{W,U_2}$.
\end{enumerate}
\end{exercise}

\begin{exercise}%2
יהי
$V$
מרחב וקטורי ממימד
$n \in \mbb{N}_+$
מעל שדה
$\mbb{F}$
ותהי
$T \in \endo_{\mbb{F}}\prs{V}$.
יהי
\[B = \prs{v_1, \ldots, v_n}\]
בסיס של
$V$,
יהי
$k \in [n-1]$
ויהיו
\begin{align*}
B_1 &\ceq \prs{v_1, \ldots, v_k} \\
B_2 &\ceq \prs{v_{k+1}, \ldots, v_n} \\
W &\ceq \spn B_1 \\
\text{.} U &\ceq \spn B_2
\end{align*}

\begin{enumerate}
\item הראו כי
$W$
הוא
$T$%
־שמור אם ורק אם
$\brs{T}_B$
מהצורה
\begin{align*}
\text{.} \brs{T}_B = \pmat{\brs{\rest{T}{W}}{B_1} & * \\ 0 & \brs{T_{U,W}}_{B_2}}
\end{align*}
\item הראו כי
$U$
הוא
$T$%
־שמור אם ורק אם
$\brs{T}_B$
מהצורה
\begin{align*}
\text{.} \brs{T}_B = \pmat{\brs{T_{W,U}}_{B_1} & 0 \\ * & \brs{\rest{T}{U}}_{B_2}}
\end{align*}
\end{enumerate}

\end{exercise}

\begin{exercise}%3
יהי
$V$
מרחב וקטורי ממימד
$n \in \mbb{N}_+$
מעל שדה
$\mbb{F}$
ותהי
$T \in \endo_{\mbb{F}}\prs{V}$.

\begin{enumerate}
\item הראו כי לכל
$i,j \in [n]$
המקיימים
$i < j$,
מתקיים
\[\text{.} \im\prs{T^i} \supseteq \im\prs{T^j}\]
\item יהי
$k \in [n]$
עבורו
$\im\prs{T^k} = \im\prs{T^{k+1}}$.
הראו כי לכל
$m \in \set{k, k+1, \ldots, n}$
מתקיים
$\im\prs{T^k} = \im\prs{T^m}$.
\end{enumerate}
\end{exercise}

\begin{exercise}%4
יהי
$V$
מרחב וקטורי ממימד
$n \in \mbb{N}_+$
מעל שדה
$\mbb{F}$
ותהי
$T \in \endo_{\mbb{F}}\prs{V}$.
נזכיר כי משפט קיילי המילטון קובע כי
$p_T\prs{T} = 0$
כאשר
\[p_T\prs{x} \ceq \prod_{i \in [k]} \prs{x-\lambda}^{r_{a,T}\prs{\lambda}}\]
הפולינום האופייני של
$T$
וכאשר
$\prs{\lambda_i}_{i \in [k]}$
הערכים העצמיים השונים של
$T$.

\begin{enumerate}
\item
יהי
\[\text{.} \mbb{F}\brs{T} \ceq \set{p\prs{T}}{p \in \mbb{F}\brs{x}} \leq \endo_{\mbb{F}}\prs{V}\]
 הראו כי
\[\text{.} \mbb{F}\brs{T} = \spn\prs{I, T, \ldots, T^{n-1}}\]

\item
מצאו דוגמא עבור
$S \in \endo_{\mbb{F}}\prs{V}$
עבורה
\[\set{I, S, \ldots, S^{n-1}}\]
קבוצה בלתי־תלויה לינארית. הסיקו שבמקרה זה
\[\text{.} \dim_{\mbb{F}} \mbb{F}\brs{S} = n\]

\item
נניח כי
$T$
הפיכה. הראו שמתקיים
$T^{-1} \in \mbb{F}\brs{T}$.

\item נניח כי
$T$
הפיכה ונסמן
\[\text{.} \mbb{F}\prs{T} \ceq \set{\sum_{i=-m}^m a_i T^i}{\substack{m \in \mbb{N} \\ \forall i: a_i \in \mbb{F}}} \leq \endo_{\mbb{F}}\prs{V}\]
הראו כי לכל
$S_1, S_2 \in \mbb{F}\brs{T}$
מתקיים
$S_1 \circ S_2 \in \mbb{F}\brs{T}$
והסיקו שמתקיים
$\mbb{F}\brs{T} = \mbb{F}\prs{T}$.
\end{enumerate}
\end{exercise}

\begin{exercise}
\begin{enumerate}
\item תהי
$T \in \endo_{\mbb{C}}\prs{\mbb{C}^4}$
שהערכים העצמיים שלה הם
$3,5,8$.
הראו כי
\[\text{.} \prs{T - 3I}^2\prs{T - 5I}^2 \prs{T - 8I}^2 = 0\]
\item תהי
$T \in \endo_{\mbb{C}}\prs{V}$
הטלה ויהיו
\[\text{.} m \ceq \dim \ker \prs{T}, \quad n \ceq \dim \im \prs{T}\]
הראו כי
\[\text{.} p_T\prs{x} = x^m \prs{x-1}^n\]
\end{enumerate}
\end{exercise}

\end{document}