\documentclass[a4paper,10pt,twoside,openany]{article}

\usepackage[lang=hebrew]{maths}
\usepackage{hebrewdoc}
\usepackage{stylish}
\usepackage{lipsum}
\let\bs\blacksquare

\setlength{\parindent}{0pt}

%%%%%%%%%%%%
% Styling %
%%%%%%%%%%%%

\usepackage{enumitem}

\renewcommand{\emph}[1]{\textbf{#1}}

%%%%%%%%%%%%%
% Counters  %
%%%%%%%%%%%%%

\setcounter{section}{1}

%%%%%%%%%%
% Title  %
%%%%%%%%%%
\title{
אלגברה ב' - גיליון תרגילי בית 1 \\
מטריצות מייצגות וכלל קרמר
\\
\vspace{1cm}
\large{תאריך הגשה: 6.4.2021}
}
\date{}

\begin{document}
\maketitle

\begin{exercise}
יהי
$V$
מרחב וקטורי מעל
$\mbb{R}$
ותהי
$T \colon V \to V$
לינארית המקיימת
$T^2 = -5 \id_V$.
\begin{enumerate}
\item הוכיחו כי לכל
$v \in V \setminus \set{0}$
הקבוצה
$\set{v, Tv}$
בלתי־תלויה לינארית.
\item נתון גם כי
$\dim V = 2$.
הוכיחו כי קיים בסיס בו
$T$
מיוצגת על ידי
$\pmat{0 & 1 \\ -5 & 0}$.
\end{enumerate}
\end{exercise}

\begin{exercise}
יהי
$B = \prs{v_i}_{i \in [n]}$
בסיס למרחב וקטורי
$V$.
נתונה
$T \colon V \to V$
הפיכה המקיימת
\[\text{.} T\prs{v_1 + 2 v_2} = \sum_{i \in [n]} v_i\]
מצאו את סכום איברי
$\brs{T^{-1}}_B$.
\end{exercise}

\begin{exercise}
יהי
$V$
מרחב וקטורי ממימד
$n \in \mbb{N}_+$
מעל שדה
$\mbb{F}$
ויהי
$B$
בסיס של
$V$.

\begin{enumerate}
\item 
תהי
\begin{align*}
\rho_B \colon V &\to \mbb{F}^n \\
\text{.} v &\mapsto \brs{v}_B
\end{align*}
הראו ש־%
$\rho_B$
חד־חד ערכית ועל.

\item תהיינה
$T,S \colon V \to V$
העתקות לינאריות עבורן
$\brs{T}_B = \brs{S}_B$.
הסיקו שמתקיים
$T = S$.
\end{enumerate}
\end{exercise}

\begin{exercise}[כלל קרמר]
תהי
$Ax = b$
מערכת משוואות כאשר
$A \in M_n\prs{\mbb{F}}$
הפיכה.

לכל
$i \in [n]$
תהי
\begin{align*}
K_{A,i} \colon \mbb{F}^n &\to \mbb{F} \\
b &\mapsto \frac{\det\pmat{\vert & & \vert & \vert & \vert & & \vert \\ A_1 & \cdots & A_{i-1} & b & A_{i+1} & \cdots & A_n \\ \vert & & \vert & \vert & \vert & & \vert}}{\det\prs{A}}
\end{align*}
ותהי
\begin{align*}
K_A \colon \mbb{F}^n &\to \mbb{F}^n \\
\text{.} \hphantom{lalala} b &\mapsto \pmat{K_{A,1}\prs{b} \\ K_{A,2}\prs{b} \\ \vdots \\ K_{A, n-1}\prs{b} \\ K_{A,n}\prs{b}}
\end{align*}

נראה שהפתרון היחיד למערכת נתון על ידי
$x = K_{A}\prs{b}$.

\begin{enumerate}
\item הראו שלכל
$i \in [n]$
ההעתקה
$K_{A,i}$
לינארית.
\item
הסיקו ש־%
$K_A$
העתקה לינארית.
\item
תהי
\begin{align*}
L_A \colon \mbb{F}^n &\to \mbb{F}^n \\
\text{.} \hphantom{lalala} v &\mapsto Av
\end{align*}

הראו ש־%
$K_A \circ L_A = \id_{\mbb{F}^n}$
על ידי בדיקה על הבסיס הסטנדרטי, והסיקו שמתקיים
$K_A = \prs{L_A}^{-1}$.
\item הסיקו שמתקיים
$\brs{K_A}_E = A^{-1}$
כאשר
$E$
הבסיס הסטנדרטי של
$\mbb{F}^n$.
\end{enumerate}
\end{exercise}

\begin{exercise}
תהי
$T \colon V \to V$
העתקה לינארית ויהיו
$B,C$
בסיסים של
$V$.
הראו כי
\[\tr\prs{\brs{T}_B} = \tr\prs{\brs{T}_C}\]
וגם
\[\text{.} \det\prs{\brs{T}_B} = \det\prs{\brs{T}_C}\]

אז נוכל להגדיר
\begin{align*}
\tr T &\ceq \tr\prs{\brs{T}_B} \\
\text{.} \det T &\ceq \det\prs{\brs{T}_B}
\end{align*}
\end{exercise}

\begin{exercise}[מטריצות בלוקים]
מטריצה
$A \in M_n\prs{\mbb{F}}$
נקראת
\emph{אלכסונית בלוקים
$\prs{n_1, \ldots, n_k}$
}
אם היא מהצורה
\[A = \pmat{A_{1} & 0 & \cdots & & 0 \\ 0 & A_{2} &  & 0 & \vdots \\ \vdots & & \ddots & & \\ & 0 & & A_{k-1} & 0 \\ 0 & \cdots & & 0 & A_k} \]
כאשר
$A_i \in M_{n_i}\prs{\mbb{F}}$
לכל
$i \in [k]$.

תהי
$A \in M_n\prs{\mbb{F}}$
אלכסונית בלוקים
$\prs{n_1, \ldots, n_k}$
עם בלוקים
$A_i \in M_{n_i}\prs{\mbb{F}}$.

\begin{enumerate}
\item
הראו שמתקיים
\[\text{.} \det\prs{A} = \prod_{i \in [k]} \det\prs{A_i}\]
\item הראו שמתקיים
\[\text{.} \tr\prs{A} = \sum_{i \in [k]} \tr\prs{A_i}\]
\item
תהי
$B \in M_n\prs{\mbb{F}}$
אלכסונית בלוקים
$\prs{n_1, \ldots, n_k}$
עם בלוקים
$B_i \in M_{n_i}\prs{\mbb{F}}$.
הראו שמתקיים
\begin{align*}
\text{.} AB &= \pmat{A_1 B_1 & 0 & \cdots & & 0 \\ 0 & A_2 B_2 &  & 0 & \vdots \\ \vdots & & \ddots & & \\ & 0 & & A_{k-1} B_{k-1} & 0 \\ 0 & \cdots & & 0 & A_k B_k} 
\end{align*}

\item הסיקו שלכל
$m \in \mbb{N}$
מתקיים
\[\text{.} A^m = \pmat{A_{1}^m & 0 & \cdots & & 0 \\ 0 & A_{2}^m &  & 0 & \vdots \\ \vdots & & \ddots & & \\ & 0 & & A_{k-1}^m & 0 \\ 0 & \cdots & & 0 & A_k^m} \]

\item הסיקו שאם
$\prs{P_i}_{i \in [k]}$
מטריצות הפיכות כך ש־%
$P_i \in M_{n_i}\prs{\mbb{F}}$
לכל
$i \in [k]$
ו־%
$P \in M_n\prs{\mbb{F}}$
מטריצה אלכסונית בלוקים
$\prs{n_1, \ldots, n_k}$
עם בלוקים
$P_i$,
אז
$P$
הפיכה ומתקיים
\[\text{.} P A P^{-1} = \pmat{P_1 A_1 P_1^{-1} & 0 & \cdots & & 0 \\ 0 & P_2 A_{2} P_2^{-1} &  & 0 & \vdots \\ \vdots & & \ddots & & \\ & 0 & & P_{k-1} A_{k-1} P_{k-1}^{-1} & 0 \\ 0 & \cdots & & 0 & P_k A_k P_k^{-1}} \]
\end{enumerate}
\end{exercise}

\end{document}