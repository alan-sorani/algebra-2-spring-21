\documentclass[a4paper,10pt,twoside,openany]{article}

\usepackage[lang=hebrew]{maths}
\usepackage{hebrewdoc}
\usepackage{stylish}
\usepackage{lipsum}
\let\bs\blacksquare

\setlength{\parindent}{0pt}

%%%%%%%%%%%%
% Styling %
%%%%%%%%%%%%

\usepackage{enumitem}

\renewcommand{\emph}[1]{\textbf{#1}}

%%%%%%%%%%%%%
% Counters  %
%%%%%%%%%%%%%

\setcounter{section}{1}

%%%%%%%%%%
% Title  %
%%%%%%%%%%
\title{
אלגברה ב' - גיליון תרגילי בית 7 \\
מכפלות פנימיות וגרם־שמידט
\\
\small{תאריך הגשה: 02.06.2021}
}
\date{}

\begin{document}
\maketitle

\begin{exercise}%1
יהי
$V$
מרחב מכפלה פנימית ותהי
$T \colon V \to V$
לינארית. הוכיחו:
\begin{enumerate}
\item אם
$T^* T = 0$
אז
$T = 0$.
\item אם
$T, T^*$
מתחלפות,
$\ker T = \ker T^*$.
\item $\prs{T^*}^{-1} = \prs{T^{-1}}^*$.
\item $\ker \prs{T^*} = \prs{\im\prs{T}}^\perp$.
\item $\im \prs{T^*} = \prs{\ker\prs{T}}^\perp$.
\end{enumerate}
\end{exercise}

\begin{exercise}
על
$\mbb{R}_3\brs{x}$
נגדיר מכפלה פנימית לפי
\[\text{.} \trs{p,q} = \int_0^1 p\prs{x} q\prs{x} \diff x\]
מצאו בסיס אורתונורמלי של
$\mbb{R}_3\brs{x}$
לפיו להעתקת הגזירה יש צורה משולשת עליונה.
\end{exercise}

\begin{exercise}%3
יהי
\[\text{.} W = \spn\prs{\pmat{1\\2\\3}, \pmat{4\\5\\6}} \leq \mbb{R}^3\]
מצאו את המרחק של
$\pmat{7\\8\\9}$
מ־%
$W$.
\end{exercise}

\begin{exercise}
יהי
$V$
מרחב מכפלה פנימית עם בסיס אורתונורמלי
$\prs{v_1, \ldots, v_n}$,
ויהיו
$u_1, \ldots, u_n \in V$
עבורם
\[\text{.} \forall i \in \brs{n} \colon \norm{v_i - u_i} < \frac{1}{\sqrt{n}}\]
הוכיחו כי
$\prs{u_1, \ldots, u_n}$
בסיס של
$V$.
\end{exercise}

\end{document}